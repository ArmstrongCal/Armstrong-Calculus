\section{Integration by Substitution} \label{S:4.6.Substitution}

\begin{goals}
\item How can we begin to find algebraic formulas for antiderivatives of more complicated algebraic functions?
\item How does the technique of $u$-substitution work to help us evaluate certain indefinite and definite integrals, and how does this process rely on identifying function-derivative pairs?
\end{goals}

%----------------------------------
% SUBSECTION INTRODUCTION
%----------------------------------
\subsection*{Introduction}

In Section~\ref{S:4.4.FTC}, we learned the key role that antiderivatives play in the the process of evaluating definite integrals exactly.  In particular, we know that if $F$ is any antiderivative of $f$, then
\[ \int_a^b f(x) \ dx = F(b) - F(a). \]
Furthermore, we realized that each elementary derivative rule developed in Chapter~\ref{CH:2} leads to a corresponding elementary antiderivative, as summarized in Table~\ref{T:4.4.Act2}.  Thus, if we wish to evaluate an integral such as 
\[ \int_0^1 \left(x^3 - \sqrt{x} + 5^x \right) \ dx, \]
it is straightforward to do so, since we can easily antidifferentiate $f(x) = x^3 - \sqrt{x} + 5^x.$ 

Because an algebraic formula for an antiderivative of $f$ enables us to evaluate the definite integral $\int_a^b f(x) \, dx$ exactly, we have a natural interest in being able to find such algebraic antiderivatives.\sidenote{Note that we emphasize \emph{algebraic} antiderivatives, as opposed to any antiderivative, since we know by the Fundamental Theorem of Calculus that $G(x) = \int_a^x f(t) \, dt$ is indeed an antiderivative of the given function $f$, but one that still involves a definite integral.}  

However, suppose we wish to evaluate the following indefinite integral.
\[ \int 2x \sqrt{1 + x^2} \ dx \]
Referencing Table~\ref{T:4.4.Act2}, notice that none of the antiderivatives involve a linear function of $x$ multiplied by a root function of $x$, which might tell us that none of the derivative rules we learned in Chapter~\ref{CH:2} give us a way to antidifferentiate the above integrand.  But in fact,
\[ \int 2x \sqrt{1 + x^2} \ dx = \frac{2}{3}\left( 1 + x^2 \right)^{3/2} + C.  \]

We know that we can easily check the evaluation of any indefinite integral simply by differentiating.  Doing so, we have
\begin{align*}
\frac{d}{dx}\left[ \frac{2}{3}\left( 1 + x^2 \right)^{3/2} + C \right] & = \frac{2}{3} \cdot \frac{3}{2} \left( 1 + x^2 \right)^{1/2} \cdot (2x) + 0 \\
& = \left( 1 + x^2 \right)^{1/2} \cdot (2x) \\
& = 2x \sqrt{1 + x^2}.
\end{align*}
While differentiating, we used a very specific rule from Chapter~\ref{CH:2}---the Chain Rule!  In fact, in every problem we encounter in this section, if we wish to check our integration by differentiating, we must use the Chain Rule.  Our goal in this section is to develop a method of integrating such that we ``undo'' the Chain Rule for differentiation.

\begin{pa} \label{PA:4.6}
In Section~\ref{S:2.6.Chain}, we learned the Chain Rule and how it can be applied to find the derivative of a composite function.  In particular, if $g$ is a differentiable function of $x$, and $f$ is a differentiable function of $g(x)$, then
$$\frac{d}{dx} \left[ f(g(x))  \right] = f'(g(x)) \cdot g'(x).$$
In words, we say that the derivative of a composite function $c(x) = f(g(x))$, where $f$ is considered the ``outer'' function and $g(x)$ the ``inner'' function, is ``the derivative of the outer function, evaluated at the inner function, times the derivative of the inner function.''  

For each of the following functions, use the Chain Rule to find the function's derivative.
\ba
\item $f(x) = e^{3x^2}$
\item $h(x) = \sin(3x + 4)$
\item $p(x) = \arctan\left( \sqrt{x} \right)$
\item $q(x) = (2-7x)^4$
\item $r(x) = \ln \left( 4-11x^3 \right)$
\ea
\end{pa} 
\afterpa % PREVIEW ACTIVITY

%----------------------------------------
% SUBSECTION U-SUBSTITUTION
%---------------------------------------
\subsection*{Reversing the Chain Rule: $u$-substitution} \index{$u$-substitution}

It is important to explicitly remember that differentiation and antidifferentiation are essentially inverse processes; that they are not quite inverse processes is due to the $+C$ that arises when antidifferentiating.  This close relationship enables us to take any known derivative rule and translate it to a corresponding rule for an indefinite integral.  

Restating the relationship defined by the Chain Rule given in Preview Activity~\ref{PA:4.6} in terms of an indefinite integral, we have
\begin{equation} \label{E:usubst} % EQUATION
\int f'(g(x)) g'(x) \, dx = f(g(x))+C.
\end{equation}
Hence, Equation~(\ref{E:usubst}) tells us that if we can take a given function and view its algebraic structure as $f'(g(x)) g'(x)$ for some appropriate choices of $f$ and $g$, then we can antidifferentiate the function by reversing the Chain Rule.  It is especially notable that both $g(x)$ and $g'(x)$ appear in the form of $f'(g(x)) g'(x)$; we will sometimes say that we seek to \emph{identify a function-derivative pair}\index{function-derivative pair} when trying to apply the rule in Equation~(\ref{E:usubst}).

In the situation where we can identify a function-derivative pair, we will introduce a new variable $u$ to represent the function $g(x)$.  Observing that with $u = g(x)$, it follows in Leibniz notation that $\frac{du}{dx} = g'(x)$, so that in terms of differentials\footnote{If we recall from the definition of the derivative that $\frac{du}{dx} \approx \frac{\triangle{u}}{\triangle{x}}$ and use the fact that $\frac{du}{dx} = g'(x)$, then we see that $g'(x) \approx \frac{\triangle{u}}{\triangle{x}}$.  Solving for $\triangle u$, $\triangle u \approx g'(x) \triangle x$.  It is this last relationship that, when expressed in ``differential'' notation enables us to write $du = g'(x) \, dx$ in the change of variable formula.}, $du = g'(x)\, dx$.  Now converting the indefinite integral of interest to a new one in terms of $u$, we have  
\[ \int f'(g(x)) g'(x) \ dx = \int f'(u) \ du. \]
To emphasize the importance of this concept, we restate it.
 
\concept{Integration by Substitution} %CONCEPT	
{Let $f$ and $g$ be differentiable functions, where the range of $g$ is an interval $I$ and the domain of $f$ is contained in $I$. Then \index{integration!by substitution}
\[ \int f'(g(x))g'(x)\ dx = f(g(x)) + C. \]
If $u = g(x)$, then $du = g'(x) \ dx$ and 
\[ \int f'(g(x))g'(x)\ dx = \int f'(u)\ du = f(u)+C = f(g(x))+C. \]
} % end concept

When $f'$ is an elementary function whose antiderivative is known, we can easily evaluate the indefinite integral in $u$, and then go on to determine the desired overall antiderivative of $f'(g(x)) g'(x)$.  We call this process \textbf{$u$-substitution}.  To see $u$-substitution at work, we consider the following example.

\begin{example} \label{Ex:4.6.usub1} % EXAMPLE
Evaluate the indefinite integral
\[ \int 4x^3 \cdot \sin (x^4 + 3) \ dx \]
and check the result by differentiating.

\solution
We can make several key algebraic observations regarding the integrand, $4x^3 \cdot \sin (x^4 + 3)$.  

First, $\sin (x^4 + 3)$ is a composite function; as such, we know we'll need a more sophisticated approach to antidifferentiating.  

Second, $4x^3$ is the derivative of $(x^4 + 3)$.  Thus, $4x^3$ and $(x^4 + 3)$ are a \emph{function-derivative} pair. 

Furthermore, we know the antiderivative of $f(u) = \sin(u)$.  The combination of these observations suggests that we can evaluate the given indefinite integral by reversing the chain rule through $u$-substitution.

Let $u = x^4 + 3$, hence $\frac{du}{dx} = 4x^3.$  In differential notation, it follows that 
\[ du = 4x^3 \, dx. \]
Observe the original indefinite integral may now be written 
\[ \int \sin (x^4 + 3) \cdot 4x^3 \ dx, \]
and by substituting the expressions in terms of $u$ for the expressions in terms of $x$ (specifically $u$ for $x^4 + 3$ and $du$ for $4x^3 \ dx$), it follows that
\[ \int \sin (x^4 + 3) \cdot 4x^3 \ dx = \int \sin(u) \cdot \ du. \]
Now we may evaluate the original integral by first evaluating the easier integral in $u$, followed by replacing $u$ by the expression $x^4 + 3$.  Doing so, we find
\begin{eqnarray*}
\int \sin (x^4 + 3) \cdot 4x^3 \, dx & = & \int \sin(u) \cdot \ du \\
& = & -\cos(u) + C \\
& = & -\cos(x^4 + 3) + C.
\end{eqnarray*}
To check our work, we observe by the Chain Rule that
\begin{eqnarray*}
\frac{d}{dx} \left[ -\cos(x^4 + 3) + C \right]  & = & -( -\sin(x^4 + 3) \cdot 4x^3) \\
& = & \sin(x^4 + 3) \cdot 4x^3,
\end{eqnarray*}
which is indeed the original integrand.
\end{example} % EXAMPLE

An essential observation about our work in Example~\ref{Ex:4.6.usub1} is that the $u$-substitution only worked because the function multiplying $\sin (x^4 + 3)$ was $4x^3$.  If instead that function was $x^2$ or $x^4$, the substitution process may not (and likely would not) have worked.  This is one of the primary challenges of antidifferentiation: slight changes in the integrand make tremendous differences.  For instance, we can use $u$-substitution with $u = x^2$ and $du = 2x \ dx$ to find that
\begin{eqnarray*}
\int xe^{x^2} \ dx & = & \int e^u \cdot \frac{1}{2} \ du \\
& = & \frac{1}{2} \int e^u \, du \\
& = & \frac{1}{2} e^u + C \\
& = & \frac{1}{2} e^{x^2} + C.
\end{eqnarray*}
If, however, we consider the similar indefinite integral
\[ \int e^{x^2} \ dx, \]
the missing $x$ to multiply $e^{x^2}$ makes the $u$-substitution $u = x^2$ no longer possible.  Hence, part of the lesson of $u$-substitution is just how specialized the process is: it only applies to situations where, up to a missing constant, the integrand that is present is the result of applying the Chain Rule to a different, related function.

Let's look at some more examples of $u$-substitution.

\begin{example} \label{Ex:4.6.usub2}
Evaluate $\ds \int \frac{7}{-3x+1}\ dx$.

\solution
First notice this integrand is a composition of 
\[ f(x) = \frac{7}{x} \quad \mbox{and} \quad g(x) = -3x+1. \]
Therefore, we begin our substitution by letting $\ds u = -3x+1$, which implies
\[ \frac{du}{dx} = -3 \quad \Rightarrow \quad du = -3 \ dx. \]
The integrand lacks a $-3$; hence divide the previous equation by $-3$ to obtain 
\[ - \frac{1}{3} du = dx.\]
We can now evaluate the integral through substitution.
\begin{align*}
\int \frac{7}{-3x+1}\ dx & = \int \frac{7}{u} \cdot -\frac{1}{3} du \\
&= -\frac{7}{3} \int \frac{1}{u} \ du \\
&= -\frac{7}{3} \ln |u| + C\\
&= -\frac{7}{3} \ln|-3x+1| + C.
\end{align*}
\end{example} % EXAMPLE

Not all integrals that benefit from substitution have a clear ``inside'' function. Several of the following examples will demonstrate ways in which this occurs.

\begin{example} \label{eg:4.6.usub3} 
Evaluate $\ds \int \sin(x) \cos(x) \ dx$.

\solution
There is not a composition of functions here to exploit; rather, just a product of functions. 

In this example, let $u = \sin(x)$. Then $du = \cos(x) \ dx$, which we have as part of the integrand! The substitution becomes very straightforward:
\begin{align*}
\int \sin(x) \cos(x) \ dx & = \int u \ du \\
&= \frac{1}{2}u^2+ C \\
&= \frac{1}{2} \sin^2(x) + C.
\end{align*}
One would do well to ask ``What would happen if we let $u = \cos(x)$?'' The answer: the result is just as easy to find, yet looks very different. The challenge to the reader is to evaluate the integral letting $u = \cos(x)$ and discovering why the answer is the same, yet looks different.
\end{example} % EXAMPLE

Do not be afraid to experiment; when given an integral to evaluate, it is often beneficial to think ``If I let $u$ be \textit{this}, then $du$ must be \textit{that} \ldots'' and see if this helps simplify the integral at all.

\begin{example} \label{Ex:4.6.ussub4} % EXAMPLE
Evaluate $\ds \int \frac{1}{x \ln(x)} \ dx$.

\solution
This is another example where there does not seem to be an obvious composition of functions. Again: choose something for $u$ and consider what this implies $du$ must be. If $u$ can be chosen such that $du$ also appears in the integrand, then we have chosen well.

Suppose
\[ u = \frac{1}{x} \quad \mbox{which implies} \quad du = -\frac{1}{x^2} \ dx;\]
however, that does not seem helpful. Suppose we let
\[ u = \ln x \quad \mbox{which implies} \quad du = \frac{1}{x} \ dx;\]
now that is part of the integrand. Thus:
\begin{align*}
\int \frac1{x\ln x}\ dx 	&=	\int \frac{1}{\underbrace{\ln x}_{1/u}}\underbrace{\frac1x\ dx}_{du} \\
&= \int \frac1u\ du \\
&= \ln |u| + C \\
&= \ln | \ln x| + C.
\end{align*}
The final answer is interesting; the natural log of the natural log. Take the derivative to confirm this answer is indeed correct.
\end{example} % EXAMPLE

The next two examples will help fill in some missing pieces of our antiderivative knowledge. We know the antiderivatives of the sine and cosine functions; what about the other standard functions tangent, cotangent, secant and cosecant? We discover these next.

\begin{example} \label{eg:4.6.usub5} % EXAMPLE
Evaluate $\ds \int \tan(x) \ dx.$

\solution
The previous paragraph established that we did not know the antiderivatives of tangent, hence we must assume that we have learned something in this section that  can help us evaluate this indefinite integral. 

Rewrite $\tan(x)$ as $\frac{\sin(x)}{\cos(x)}$. While the presence of a composition of functions may not be immediately obvious, recognize that $\cos(x)$ is ``inside'' the $\frac{1}{x}$ function. Therefore, we see if setting $u = \cos(x)$ returns usable results. We have that $du = -\sin(x) \ dx$, hence $-du = \sin(x) \ dx$. We can integrate:
\begin{align*}
\int \tan(x) \ dx &= \int \frac{\sin(x)}{\cos(x)}\ dx \\
&= \int \frac1{\underbrace{\cos(x)}_u}\underbrace{\sin(x) \ dx}_{-du} \\
&= \int \frac {-1}u \ du\\
&= -\ln |u| + C \\
&= -\ln |\cos(x)| + C.
\end{align*}
Some texts prefer to bring the $-1$ inside the logarithm as a power of $\cos(x)$, as in:
\begin{align*}
-\ln |\cos(x)| + C &= \ln |(\cos(x))^{-1}| + C\\
&= \ln \left| \frac{1}{\cos(x)}\right| + C\\
&= \ln |\sec(x)| + C.
\end{align*}
Thus the result they give is $\int \tan x \ dx = \ln|\sec(x)| + C$. These two answers are equivalent.
\end{example} % EXAMPLE

\begin{example} \label{eg:4.6.usub6} % EXAMPLE
Evaluate $\ds\int \sec(x) \ dx$.

\solution
This example employs a wonderful trick: multiply the integrand by ``$1$'' so that we see how to integrate more clearly. In this case, we write ``$1$'' as
\[ 1 = \frac{\sec x + \tan x}{\sec x + \tan x}. \]
This may seem like it came out of left field, but it works beautifully. Consider:
\begin{align*}
\int \sec x\ dx	&=	\int \sec x\cdot \frac{\sec x + \tan x}{\sec x + \tan x}\ dx \\
&= \int \frac{\sec^2 x + \sec x\tan x}{\sec x + \tan x}\ dx.\\
\intertext{Now let $u = \sec x+\tan x$; this means $du = (\sec x\tan x+ \sec^2 x)\ dx$, which is our numerator. Thus:}
&= \int \frac{du}{u} \\
&= \ln |u| + C \\
&= \ln |\sec x+\tan x| + C.
\end{align*}
\end{example} % EXAMPLE

We can use similar techniques to those used in Examples \ref{eg:4.6.usub5} and \ref{eg:4.6.usub6} to find antiderivatives of $\cot x$ and $\csc x$ (which the reader can explore in the exercises.)

\input{activities/4.6.Act1} % ACTIVITY

%-----------------------------------------------------
% SUBSECTION U-SUBSTITUTION WITH DEFINITE INTEGRALS
%-----------------------------------------------------
\subsection*{Evaluating Definite Integrals via $u$-substitution}

This section has focused on $u$-substitution as a means to evaluate indefinite integrals of functions that can be written, up to a constant multiple, in the form $f(g(x))g'(x)$.  However, much of the time integration is used in the context of definite integrals involving such functions. We need to be careful with the corresponding limits of integration.  Consider, for instance, the definite integral
\[ \int_2^5 xe^{x^2} \ dx. \]
Whenever we write a definite integral, it is implicit that the limits of integration correspond to the variable of integration.  To be more explicit, observe that
\[ \int_2^5 xe^{x^2} \ dx = \int_{x=2}^{x=5} xe^{x^2} \ dx. \]
When we execute a $u$-substitution, we change the \emph{variable} of integration; it is essential to note that this also changes the \emph{limits} of integration.  For instance, with the substitution $u = x^2$ and $du = 2x \, dx$, it also follows that when $x = 2$, $u = (2)^2 = 4$, and when $x = 5$, $u = (5)^2 = 25.$  Thus, under the change of variables of $u$-substitution, we now have
\begin{eqnarray*}
\int_{x=2}^{x=5} xe^{x^2} \ dx & = & \int_{u=4}^{u=25} e^{u} \cdot \frac{1}{2} \ du \\
& = & \left. \frac{1}{2}e^u \right|_{u=4}^{u=25} \\
& = & \frac{1}{2}e^{25} - \frac{1}{2}e^4.
\end{eqnarray*}

Alternatively, we could consider the related indefinite integral $\int_2^5 xe^{x^2} \, dx,$ find the antiderivative $\frac{1}{2}e^{x^2}$ through $u$-substitution, and then evaluate the original definite integral.  From that perspective, we'd have
\begin{eqnarray*}
\int_{2}^{5} xe^{x^2} \, dx & = & \left. \frac{1}{2}e^{x^2} \right|_{2}^{5} \\
& = & \frac{1}{2}e^{25} - \frac{1}{2}e^4,
\end{eqnarray*}
which is, of course, the same result.

\concept{Substitution with Definite Integrals} %CONCEPT
{Let $f$ and $g$ be differentiable functions, where the range of $g$ is an interval $I$ that contains the domain of $f$. Then \index{integration!definite!and substitution}\index{definite integral!and substitution}
\[ \int_a^b f'\big(g(x)\big)g'(x)\ dx = \int_{g(a)}^{g(b)} f'(u)\ du. \]
} % end concept

Let's look at more examples.

\begin{example} \label{Ex:4.6.sub7}
Evaluate $\ds\int_0^2 \cos(3x-1)\ dx$.

\solution
In this example, we begin with
\[ u = 3x - 1 \quad \mbox{and} \quad \frac{du}{dx} = 3 \Rightarrow \frac{1}{3}du = dx. \]
Rewriting the bounds of integration with respect to $u$, we have $u(2) = 3 \cdot 2 - 1 = 5$ and $u(0) = 3 \cdot 0 - 1 = -1$.   We now evaluate the definite integral:
\begin{align*}
\int_1^2 \cos(3x-1) \ dx &=	\int_{-1}^5 \cos(u) \cdot \frac{1}{3} \ du \\
&= \frac{1}{3} \sin(u) \Big|_{-1}^5 \\
&= \frac{1}{3}\big( \sin(5) - \sin(-1) \big)\\
&\approx -0.039.
\end{align*}
Notice how once we converted the integral to be in terms of $u$, we never went back to using $x$.

%The graphs in Figure \ref{fig:subst12} tell more of the story. In (a) the area defined by the original integrand is shaded, whereas in (b) the area defined by the new integrand is shaded. In this particular situation, the areas look very similar; the new region is ``shorter'' but ``wider,'' giving the same area.
\end{example} % EXAMPLE

\begin{example} \label{eg:4.6.sub8} 
Evaluate $\ds \int_0^{\pi/2} \sin(x) \cos(x) \ dx$.

\solution
We saw the corresponding indefinite integral in Example \ref{eg:4.6.usub3}. In that example we set $u = \sin(x)$ but stated that we could have let $u = \cos(x)$. For variety, we do the latter here.

Let $u = g(x) = \cos(x)$, giving $du = -\sin(x) \ dx$. The new upper bound is $g(\pi/2) = 0$; the new lower bound is $g(0) = 1$. Note how the lower bound is actually larger than the upper bound now. We have
\begin{align*}
\int_0^{\pi/2} \sin(x) \cos(x)\ dx &= \int_1^0 u \ (-1) du \\
&= \int_1^0 -u \ du \quad \text{\scriptsize (switch bounds \& change sign)}\\
&=	\int_0^1 u \ du\\
&= \frac{1}{2} u^2 \Big|_0^1\\
&= 1/2.
\end{align*}
\end{example} % EXAMPLE

\begin{activity} \label{A:4.6.2}  Evaluate each of the following definite integrals exactly through an appropriate $u$-substitution.

\ba
	\item $\ds \int_1^2 \frac{x}{1 + 4x^2} \, dx$
	\item $\ds \int_0^1 e^{-x} (2e^{-x}+3)^{9} \, dx$
	\item $\ds \int_{2/\pi}^{4/\pi} \frac{\cos\left(\frac{1}{x}\right)}{x^{2}} \,dx$
\ea
\end{activity}
\begin{smallhint}
\ba
	\item Small hints for each of the prompts above.
\ea
\end{smallhint}
\begin{bighint}
\ba
	\item Big hints for each of the prompts above.
\ea
\end{bighint}
\begin{activitySolution}
\ba
	\item Solutions for each of the prompts above.
\ea
\end{activitySolution}
\aftera % ACTIVITY

\newpage

%-------------
% SUMMARY
%-------------
\begin{summary}
\item To begin to find algebraic formulas for antiderivatives of more complicated algebraic functions, we need to think carefully about how we can reverse known differentiation rules.  To that end, it is essential that we understand and recall known derivatives of basic functions, as well as the standard derivative rules.

\item The indefinite integral provides notation for antiderivatives.  When we write ``$\int f(x) \, dx$,'' we mean ``the general antiderivative of $f$.''  In particular, if we have functions $f$ and $F$ such that $F' = f$, the following two statements say the exact thing:
\[ \frac{d}{dx}[F(x)] = f(x) \ \mbox{and} \ \int f(x) \, dx = F(x) + C. \]
That is, $f$ is the derivative of $F$, and $F$ is an antiderivative of $f$.

\item The technique of $u$-substitution helps us evaluate indefinite integrals of the form 
\[\int f(g(x))g'(x) \, dx\]
through the substitutions $u = g(x)$ and $du = g'(x) \, dx$, so that
\[ \int f(g(x))g'(x) \, dx = \int f(u) \, du. \]
A key part of choosing the expression in $x$ represented by $u$ is the identification of a function-derivative pair.  To do so, we often look for an ``inner'' function $g(x)$ that is part of a composite function, while investigating whether $g'(x)$ (or a constant multiple of $g'(x)$) is present as a multiplying factor of the integrand.
\end{summary}

\clearpage

%--------------
% EXERCISES
%--------------
\begin{adjustwidth*}{}{-2.25in}
\textbf{{\large Exercises}}
\setlength{\columnsep}{25pt}
\begin{multicols*}{2}
\noindent Terms and Concepts \small
\begin{enumerate}[1)]
\item Substitution ``undoes'' what derivative rule?
\item T/F: One can use algebra to rewrite the integrand of an integral to make it easier to evaluate.
\end{enumerate} 

\noindent {\normalsize Problems} \small

\noindent{\bf In exercises 3--14, evaluate the indefinite integral to develop an understanding of Substitution.}

\begin{enumerate}[1),resume]
\item $\ds \int 3 x^2 \left(x^3-5\right)^7 dx $
\item $\ds \int (2 x-5) \left(x^2-5 x+7\right)^3 dx $
\item $\ds \int x \left(x^2+1\right)^8 dx $
\item $\ds \int (12 x+14) \left(3 x^2+7 x-1\right)^5 dx $
\item $\ds \int \frac{1}{2 x+7} dx $
\item $\ds \int \frac{1}{\sqrt{2 x+3}} dx $
\item $\ds \int \frac{x}{\sqrt{x+3}} dx $
\item $\ds \int \frac{x^3-x}{\sqrt{x}} dx $
\item $\ds \int \frac{e^{\sqrt{x}}}{\sqrt{x}} dx $
\item $\ds \int \frac{x^4}{\sqrt{x^5+1}} dx $
\item $\ds \int \frac{\frac{1}{x}+1}{x^2} dx $
\item $\ds \int \frac{\ln(x)}{x} dx $
\end{enumerate}

\noindent{\bf In Exercises 15--21, use Substitution to evaluate the indefinite integral involving trigonometric functions.}

\begin{enumerate}[1),resume]
\item $\ds \int \sin ^2(x) \cos (x) dx $
\item $\ds \int \cos (3-6 x) dx $
\item $\ds \int \sec ^2(4-x) dx $
\item $\ds \int \sec (2 x) dx $
\item $\ds \int \tan ^2(x) \sec ^2(x) dx $
\item $\ds \int x \cos \left(x^2\right) dx $
\item $\ds \int \tan ^2(x) dx $
\end{enumerate}

\columnbreak

\noindent{\bf In Exercises 22--28, use Substitution to evaluate the indefinite integral involving exponential functions.}

\begin{enumerate}[1),resume]
\item $\ds \int e^{3 x-1} dx $
\item $\ds \int e^{x^3} x^2 dx $
\item $\ds \int e^{x^2-2 x+1} (x-1) dx $
\item $\ds \int \frac{e^x+1}{e^x} dx $
\item $\ds \int \frac{e^x-e^{-x}}{e^{2x}} dx $
\item $\ds \int 3^{3 x} dx $
\item $\ds \int 4^{2 x} dx $
\end{enumerate}

\noindent{\bf In Exercises 29--31, use Substitution to evaluate the indefinite integral involving logarithmic functions.}

\begin{enumerate}[1),resume]
\item $\ds \int \frac{\ln ^2(x)}{x} dx $
\item $\ds \int \frac{\ln \left(x^3\right)}{x} dx $
\item $\ds \int \frac{1}{x \ln \left(x^2\right)} dx $
\end{enumerate}

\noindent{\bf In Exercises 32--37, use Substitution to evaluate the indefinite integral involving rational functions.}

\begin{enumerate}[1),resume]
\item $\ds \int \frac{x^2 + 3x + 1}{x} dx $
\item $\ds \int \frac{x^3+x^2+x+1}{x} dx $
\item $\ds \int \frac{x^3-1}{x+1} dx $
\item $\ds \int \frac{x^2+2 x-5}{x-3} dx $
\item $\ds \int \frac{3 x^2-5 x+7}{x+1} dx $
\item $\ds \int \frac{x^2+2 x+1}{x^3+3 x^2+3 x} dx $
\end{enumerate}

\noindent{\bf In Exercises 38--47, use Substitution to evaluate the indefinite integral involving inverse trigonometric functions.}

\begin{enumerate}[1),resume]
\item $\ds \int \frac{7}{x^2+7} dx $
\item $\ds \int \frac{3}{\sqrt{9-x^2}} dx $
\item $\ds \int \frac{14}{\sqrt{5-x^2}} dx $
\item $\ds \int \frac{2}{x \sqrt{x^2-9}} dx $
\end{enumerate}

%------------------------------------------
% END OF EXERCISES ON FIRST PAGE
%------------------------------------------
\end{multicols*}
\end{adjustwidth*}

\clearpage

\begin{adjustwidth*}{}{-2.25in}
\setlength{\columnsep}{25pt}
\begin{multicols*}{2}\small

\begin{enumerate}[1),start=42]
\item $\ds \int \frac{5}{\sqrt{x^4-16 x^2}} dx $
\item $\ds \int \frac{x}{\sqrt{1-x^4}} dx $
\item $\ds \int \frac{1}{x^2-2 x+8} dx $
\item $\ds \int \frac{2}{\sqrt{-x^2+6 x+7}} dx $
\item $\ds \int \frac{3}{\sqrt{-x^2+8 x+9}} dx $
\item $\ds \int \frac{5}{x^2+6 x+34} dx $
\end{enumerate}

\vspace{.25cm}

\noindent{\bf In Exercises 48--72, use Substitution to evaluate the indefinite integral. }

\begin{enumerate}[1),resume]
\item $\ds \int \frac{x^2}{\left(x^3+3\right)^2} dx $
\item $\ds \int \left(3 x^2+2 x\right) \left(5 x^3+5 x^2+2\right)^8 dx $
\item $\ds \int \frac{x}{\sqrt{1-x^2}} dx $
\item $\ds \int x^2 \csc ^2\left(x^3+1\right) dx $
\item $\ds \int \sin (x) \sqrt{\cos (x)} dx $
\item $\ds \int \frac{1}{x-5} dx $
\item $\ds \int \frac{7}{3 x+2} dx $
\item $\ds \int \frac{3 x^3+4 x^2+2 x-22}{x^2+3 x+5} dx $
\item $\ds \int \frac{2 x+7}{x^2+7 x+3} dx $
\item $\ds \int \frac{9 (2 x+3)}{3 x^2+9 x+7} dx $
\item $\ds \int \frac{-x^3+14 x^2-46 x-7}{x^2-7 x+1} dx $
\item $\ds \int \frac{x}{x^4+81} dx $
\item $\ds \int \frac{2}{4 x^2+1} dx $
\item $\ds \int \frac{1}{x \sqrt{4 x^2-1}} dx $
\item $\ds \int \frac{1}{\sqrt{16-9 x^2}} dx $

\item $\ds \int \frac{3 x-2}{x^2-2 x+10} dx $
\item $\ds \int \frac{7-2 x}{x^2+12 x+61} dx $
\item $\ds \int \frac{x^2+5 x-2}{x^2-10 x+32} dx $
\item $\ds \int \frac{x^3}{x^2+9} dx $
\item $\ds \int \frac{x^3-x}{x^2+4 x+9} dx $
\item $\ds \int \frac{\sin (x)}{\cos ^2(x)+1} dx $
\item $\ds \int \frac{\cos (x)}{\sin ^2(x)+1} dx $
\item $\ds \int \frac{\cos (x)}{1-\sin ^2(x)} dx $
\item $\ds \int \frac{3 x-3}{\sqrt{x^2-2 x-6}} dx $
\item $\ds \int \frac{x-3}{\sqrt{x^2-6 x+8}} dx $
\end{enumerate}

\noindent{\bf In Exercises 73--80, use Substitution to evaluate the definite integral. }

\begin{enumerate}[1),resume]
\item $\ds \int_1^3 \frac{1}{x-5} dx $
\item $\ds \int_2^6 x\sqrt{x-2} dx $
\item $\ds \int_{-\pi/2}^{\pi/2} \sin^2(x) \cos (x)\ dx $
\item $\ds \int_{0}^{1} 2x(1-x^2)^4\ dx $
\item $\ds \int_{-2}^{-1} (x+1)e^{x^2+2x+1}\ dx $
\item $\ds \int_{-1}^{1} \frac{1}{1+x^2}\ dx $
\item $\ds \int_{2}^{4} \frac{1}{x^2-6x+10}\ dx $
\item $\ds \int_{1}^{\sqrt{3}} \frac{1}{\sqrt{4-x^2}}\ dx $

  \item For the town of Mathland, MI, residential power consumption has shown certain trends over recent years.  Based on data reflecting average usage, engineers at the power company have modeled the town's rate of energy consumption by the function
 $$r(t) = 4 + \sin(0.263t + 4.7) + \cos(0.526t+9.4).$$
Here, $t$ measures time in hours after midnight on a typical weekday, and $r$ is the rate of consumption in megawatts at time $t$. %{\em Note: The unit megawatt is itself a rate, which measures energy consumption per unit time.  A megawatt-hour is the total amount of energy that is equivalent to a constant stream of 1 megawatt of power being sustained for 1 hour.}
Units are critical throughout this problem.
	\ba
		\item Sketch a carefully labeled graph of $r(t)$ on the interval $[0,24]$ and explain its meaning.  Why is this a reasonable model of power consumption?
		\item Without calculating its value, explain the meaning of $\int_0^{24} r(t) \, dt$.   Include appropriate units on your answer.
		
  		\item Determine the exact amount of power Mathland consumes in a typical day.  
		\item What is Mathland's average rate of energy consumption in a given $24$-hour period?  What are the units on this quantity?
	\ea
\end{enumerate}

%---------------------------------------------
% END OF EXERCISES ON SECOND PAGE
%---------------------------------------------
\end{multicols*}
\end{adjustwidth*}

\afterexercises 

\cleardoublepage