\section{Integration by Substitution} \label{S:5.3.Substitution}

\begin{goals}
\item How can we begin to find algebraic formulas for antiderivatives of more complicated algebraic functions?
\item How does the technique of $u$-substitution work to help us evaluate certain indefinite integrals, and how does this process rely on identifying function-derivative pairs?
\end{goals}

%----------------------------------
% SUBSECTION INTRODUCTION
%----------------------------------
\subsection*{Introduction}

In Section~\ref{S:4.4.FTC}, we learned the key role that antiderivatives play in the the process of evaluating definite integrals exactly.  In particular, we know that if $F$ is any antiderivative of $f$, then
\[ \int_a^b f(x) \ dx = F(b) - F(a). \]
Furthermore, we realized that each elementary derivative rule developed in Chapter~\ref{C:2} leads to a corresponding elementary antiderivative, as summarized in Table~\ref{T:4.4.Act2}.  Thus, if we wish to evaluate an integral such as 
\[ \int_0^1 \left(x^3 - \sqrt{x} + 5^x \right) \ dx, \]
it is straightforward to do so, since we can easily antidifferentiate $f(x) = x^3 - \sqrt{x} + 5^x.$ 

%In particular, since a function $F$ whose derivative is $f$ is given by $F(x) = \frac{1}{4}x^4 - \frac{2}{3}x^{3/2} + \frac{1}{\ln(5)}5^x$, the Fundamental Theorem of Calculus tells us that
%\begin{eqnarray*}
%\int_0^1 \left(x^3 - \sqrt{x} + 5^x\right) \,dx & = & \left. \frac{1}{4}x^4 - \frac{2}{3}x^{3/2} + \frac{1}{\ln(5)}5^x\right|_0^1 \\
%& = & \left( \frac{1}{4}(1)^4 - \frac{2}{3}(1)^{3/2} + \frac{1}{\ln(5)}5^1 \right) - \left( \frac{1}{4}(0)^4 - \frac{2}{3}(0)^{3/2} + \frac{1}{\ln(5)}5^0 \right) \\
%& = & \frac{1}{4} - \frac{2}{3} + \frac{5}{\ln(5)} - \frac{1}{\ln(5)} \\
%& = & -\frac{5}{12} + \frac{4}{\ln(5)}.
%\end{eqnarray*}

Because an algebraic formula for an antiderivative of $f$ enables us to evaluate the definite integral $\int_a^b f(x) \, dx$ exactly, we have a natural interest in being able to find such algebraic antiderivatives.\sidenote{Note that we emphasize \emph{algebraic} antiderivatives, as opposed to any antiderivative, since we know by the Fundamental Theorem of Calculus that $G(x) = \int_a^x f(t) \, dt$ is indeed an antiderivative of the given function $f$, but one that still involves a definite integral.}  One of our main goals in this section is to develop understanding, in select circumstances, of how to ``undo'' the process of differentiation in order to find an algebraic antiderivative for a given function.

%\begin{pa} \label{PA:5.3}
In Section~\ref{S:2.8.Inverse}, we discussed derivatives of inverse functions including the inverse trigonometric functions $\arcsin (x)$, $\arccos (x)$, and $\arctan (x)$. We will need the techniques to find the derivatives to evaluate integrals of a certain form. Let's review the technique.
Suppose $f(x) = \arcsin (x)$ for $-1<x<1$ and $g(x)=\arctan(x)$ for $-1<x<1$
	\be
		\item Rewrite $f(x)$ as $\theta$. Label the right triangle that corresponds to $\theta=\arcsin(x)$. 
		\item Use the completed triangle to show $$ \tan (\arcsin (x)) = \frac{x}{1-x^2} $$
		\item Rewrite $g(x)$ as $\phi$. Label the right triangle that corresponds to $\phi=\arctan(x)$.
		\item Use the completed triangle to show $$ \cos (\arctan (x)) = \frac{1}{\sqrt{1+x^2}} $$
	\ee
\end{pa} 
\afterpa

%add two triangles like figure 2.41 % Change number to 4.6, change references % PREVIEW ACTIVITY

%-----------------------------------------
% SUBSECTION reversing chain rule
%-----------------------------------------
\subsection*{Reversing the Chain Rule: First Steps} 

In Preview Activity~\ref{PA:5.3}, we saw that it is usually straightforward to antidifferentiate a function of the form
\[ h(x) = f(u(x)), \]
whenever $f$ is a familiar function whose antiderivative is known and $u(x)$ is a linear function.  For example, if we consider
\[ h(x) = (5x-3)^6, \]
in this context the outer function $f$ is $f(u) = u^6$, while the inner function is $u(x) = 5x - 3$.  Since the antiderivative of $f$ is $F(u) = \frac{1}{7}u^7+C$, we 
see that the antiderivative of $h$ is
\[ H(x) = \frac{1}{7} (5x-3)^7 \cdot \frac{1}{5} + C = \frac{1}{35} (5x-3)^7 + C. \]
The inclusion of the constant $\frac{1}{5}$ is essential precisely because the derivative of the inner function is $u'(x) = 5$.  Indeed, if we now compute $H'(x)$, we find by the Chain Rule (and Constant Multiple Rule) that
\[ H'(x) = \frac{1}{35} \cdot 7(5x-3)^6 \cdot 5 = (5x-3)^6 = h(x), \]
and thus $H$ is indeed the general antiderivative of $h$.

% removed the part on linear special case 

\begin{activity} \label{A:5.3.1}  Evaluate each of the following indefinite integrals.  
\bmtwo\ba
	\item $\ds \int \frac{\sqrt{4-x^2}}{x^2}\ dx$
	\item $\ds \int \sqrt{5+4x-x^2} \, dx$
	\item $\ds \int \frac{1}{\sqrt{x^2-6x+13}} \, dx$
	\item $\ds \int \frac{1}{x^3\sqrt{x^2-1}}\, dx$
\ea\emtwo
\end{activity}
\begin{smallhint}
\ba
	\item Small hints for each of the prompts above.
\ea
\end{smallhint}
\begin{bighint}
\ba
	\item Big hints for each of the prompts above.
\ea
\end{bighint}
\begin{activitySolution}
\ba
	\item We use Key Idea \ref{idea:trigsub}(a) with $a=2$, $x=2\sin \theta$, $dx = 2\cos \theta$ and hence $\sqrt{4-x^2} = 2\cos\theta$. This gives
\begin{align*}
\int \frac{\sqrt{4-x^2}}{x^2}\ dx &= \int \frac{2\cos\theta}{4\sin^2\theta}(2\cos\theta)\ d\theta\\
		&= \int \cot^2\theta\ d\theta\\
		&=	\int (\csc^2\theta -1)\ d\theta\\
		&= -\cot\theta -\theta + C.
\end{align*}
We need to rewrite our answer in terms of $x$. Using the reference triangle found in Key Idea \ref{idea:trigsub}(a), we have $\cot\theta = \sqrt{4-x^2}/x$ and $\theta = \sin^{-1}(x/2)$. Thus
$$\int \frac{\sqrt{4-x^2}}{x^2}\ dx = -\frac{\sqrt{4-x^2}}x-\sin^{-1}\left(\frac x2\right) + C.$$
%Solutions for each of the prompts above.
	\item 

\ea
\end{activitySolution}
\aftera % change to function/derivative pairs some being composition functions
 
%----------------------------------------
% SUBSECTION U-SUBSTITUTION
%---------------------------------------

\subsection*{Reversing the Chain Rule: $u$-substitution} \index{$u$-substitution}

Of course, a natural question  arises from our recent work: what happens when the inner function is not a linear function?  For example, can we find antiderivatives of such functions as 
\[ g(x) = x e^{x^2} \ \mbox{and} \ h(x) = e^{x^2}? \]

It is important to explicitly remember that differentiation and antidifferentiation are essentially inverse processes; that they are not quite inverse processes is due to the $+C$ that arises when antidifferentiating.  This close relationship enables us to take any known derivative rule and translate it to a corresponding rule for an indefinite integral.  For example, since
\[ \frac{d}{dx} \left[x^5\right] = 5x^4, \]
we can equivalently write
\[ \int 5x^4 \, dx = x^5 + C. \]

Recall that the Chain Rule states that
\[ \frac{d}{dx} \left[ f(g(x)) \right] = f'(g(x)) \cdot g'(x). \]
Restating this relationship in terms of an indefinite integral,
\begin{equation} \label{E:usubst} % EQUATION
\int f'(g(x)) g'(x) \, dx = f(g(x))+C.
\end{equation}
Hence, Equation~(\ref{E:usubst}) tells us that if we can take a given function and view its algebraic structure as $f'(g(x)) g'(x)$ for some appropriate choices of $f$ and $g$, then we can antidifferentiate the function by reversing the Chain Rule.  It is especially notable that both $g(x)$ and $g'(x)$ appear in the form of $f'(g(x)) g'(x)$; we will sometimes say that we seek to \emph{identify a function-derivative pair}\index{function-derivative pair} when trying to apply the rule in Equation~(\ref{E:usubst}).

In the situation where we can identify a function-derivative pair, we will introduce a new variable $u$ to represent the function $g(x)$.  Observing that with $u = g(x)$, it follows in Leibniz notation that $\frac{du}{dx} = g'(x)$, so that in terms of differentials\footnote{If we recall from the definition of the derivative that $\frac{du}{dx} \approx \frac{\triangle{u}}{\triangle{x}}$ and use the fact that $\frac{du}{dx} = g'(x)$, then we see that $g'(x) \approx \frac{\triangle{u}}{\triangle{x}}$.  Solving for $\triangle u$, $\triangle u \approx g'(x) \triangle x$.  It is this last relationship that, when expressed in ``differential'' notation enables us to write $du = g'(x) \, dx$ in the change of variable formula.}, $du = g'(x)\, dx$.  Now converting the indefinite integral of interest to a new one in terms of $u$, we have  
\[ \int f'(g(x)) g'(x) \ dx = \int f'(u) \ du. \]
To emphasize the importance of this concept, we restate it.
 
\concept{Integration by Substitution} %CONCEPT	
{Let $F$ and $g$ be differentiable functions, where the range of $g$ is an interval $I$ and the domain of $F$ is contained in $I$. Then \index{integration!by substitution}
\[ \int F'(g(x))g'(x)\ dx = F(g(x)) + C. \]
If $u = g(x)$, then $du = g'(x)dx$ and 
\[ \int F'(g(x))g'(x)\ dx = \int F'(u)\ du = F(u)+C = F(g(x))+C. \]
} % end concept

When $f'$ is an elementary function whose antiderivative is known, we can easily evaluate the indefinite integral in $u$, and then go on to determine the desired overall antiderivative of $f'(g(x)) g'(x)$.  We call this process \textbf{$u$-substitution}.  To see $u$-substitution at work, we consider the following example.

\begin{example} \label{Ex:4.6.usub1} % EXAMPLE
Evaluate the indefinite integral
\[ \int 4x^3 \cdot \sin (x^4 + 3) \ dx \]
and check the result by differentiating.

\solution
We can make several key algebraic observations regarding the integrand, $4x^3 \cdot \sin (x^4 + 3)$.  

First, $\sin (x^4 + 3)$ is a composite function; as such, we know we'll need a more sophisticated approach to antidifferentiating.  

Second, $4x^3$ is the derivative of $(x^4 + 3)$.  Thus, $4x^3$ and $(x^4 + 3)$ are a \emph{function-derivative} pair. 

Furthermore, we know the antiderivative of $f(u) = \sin(u)$.  The combination of these observations suggests that we can evaluate the given indefinite integral by reversing the chain rule through $u$-substitution.

Let $u = x^4 + 3$, hence $\frac{du}{dx} = 4x^3.$  In differential notation, it follows that 
\[ du = 4x^3 \, dx. \]
Observe the original indefinite integral may now be written 
\[ \int \sin (x^4 + 3) \cdot 4x^3 \ dx, \]
and by substituting the expressions in terms of $u$ for the expressions in terms of $x$ (specifically $u$ for $x^4 + 3$ and $du$ for $4x^3 \ dx$), it follows that
\[ \int \sin (x^4 + 3) \cdot 4x^3 \ dx = \int \sin(u) \cdot \ du. \]
Now we may evaluate the original integral by first evaluating the easier integral in $u$, followed by replacing $u$ by the expression $x^4 + 3$.  Doing so, we find
\begin{eqnarray*}
\int \sin (x^4 + 3) \cdot 4x^3 \, dx & = & \int \sin(u) \cdot \ du \\
& = & -\cos(u) + C \\
& = & -\cos(x^4 + 3) + C.
\end{eqnarray*}
To check our work, we observe by the Chain Rule that
\begin{eqnarray*}
\frac{d}{dx} \left[ -\cos(x^4 + 3) + C \right]  & = & -( -\sin(x^4 + 3) \cdot 4x^3) \\
& = & \sin(x^4 + 3) \cdot 4x^3,
\end{eqnarray*}
which is indeed the original integrand.
\end{example} % EXAMPLE

An essential observation about our work in Example~\ref{Ex:4.6.usub1} is that the $u$-substitution only worked because the function multiplying $\sin (7x^4 + 3)$ was $x^3$.  If instead that function was $x^2$ or $x^4$, the substitution process may not (and likely would not) have worked.  This is one of the primary challenges of antidifferentiation: slight changes in the integrand make tremendous differences.  For instance, we can use $u$-substitution with $u = x^2$ and $du = 2xdx$ to find that
\begin{eqnarray*}
\int xe^{x^2} \ dx & = & \int e^u \cdot \frac{1}{2} \ du \\
& = & \frac{1}{2} \int e^u \, du \\
& = & \frac{1}{2} e^u + C \\
& = & \frac{1}{2} e^{x^2} + C.
\end{eqnarray*}
If, however, we consider the similar indefinite integral
\[ \int e^{x^2} \ dx, \]
the missing $x$ to multiply $e^{x^2}$ makes the $u$-substitution $u = x^2$ no longer possible.  Hence, part of the lesson of $u$-substitution is just how specialized the process is: it only applies to situations where, up to a missing constant, the integrand that is present is the result of applying the Chain Rule to a different, related function.

Let's look at some more examples of $u$-substitution.

\begin{example} \label{Ex:4.6.usub2}
Evaluate $\ds \int \frac{7}{-3x+1}\ dx$.

\solution
First notice this integrand is a composition of 
\[ f(x) = \frac{7}{x} \quad \mbox{and} \quad g(x) = -3x+1. \]
Therefore, we begin our substitution by letting $\ds u = -3x+1$, which implies
\[ \frac{du}{dx} = -3 \quad \Rightarrow \quad du = -3 \ dx. \]
The integrand lacks a $-3$; hence divide the previous equation by $-3$ to obtain 
\[ - \frac{1}{3} du = dx.\]
We can now evaluate the integral through substitution.
\begin{align*}
\int \frac{7}{-3x+1}\ dx & = \int \frac{7}{u} \cdot -\frac{1}{3} du \\
&= -\frac{7}{3} \int \frac{1}{u} \ du \\
&= -\frac{7}{3} \ln |u| + C\\
&= -\frac{7}{3} \ln|-3x+1| + C.
\end{align*}
\end{example} % EXAMPLE

Not all integrals that benefit from substitution have a clear ``inside'' function. Several of the following examples will demonstrate ways in which this occurs.

\begin{example} \label{eg:4.6.usub3} 
Evaluate $\ds \int \sin(x) \cos(x) \ dx$.

\solution
There is not a composition of functions here to exploit; rather, just a product of functions. 

In this example, let $u = \sin(x)$. Then $du = \cos(x) \ dx$, which we have as part of the integrand! The substitution becomes very straightforward:
\begin{align*}
\int \sin(x) \cos(x) \ dx & = \int u \ du \\
&= \frac{1}{2}u^2+ C \\
&= \frac{1}{2} \sin^2(x) + C.
\end{align*}
One would do well to ask ``What would happen if we let $u = \cos(x)$?'' The answer: the result is just as easy to find, yet looks very different. The challenge to the reader is to evaluate the integral letting $u = \cos(x)$ and discovering why the answer is the same, yet looks different.
\end{example} % EXAMPLE

Do not be afraid to experiment; when given an integral to evaluate, it is often beneficial to think ``If I let $u$ be \textit{this}, then $du$ must be \textit{that} \ldots'' and see if this helps simplify the integral at all.

\begin{example} \label{Ex:4.6.ussub4} % EXAMPLE
Evaluate $\ds \int \frac{1}{x \ln(x)} \ dx$.

\solution
This is another example where there does not seem to be an obvious composition of functions. Again: choose something for $u$ and consider what this implies $du$ must be. If $u$ can be chosen such that $du$ also appears in the integrand, then we have chosen well.

Suppose
\[ u = \frac{1}{x} \quad \mbox{which implies} \quad du = -\frac{1}{x^2} \ dx;\]
however, that does not seem helpful. Suppose we let
\[ u = \ln x \quad \mbox{which implies} \quad du = \frac{1}{x} \ dx;\]
now that is part of the integrand. Thus:
\begin{align*}
\int \frac1{x\ln x}\ dx 	&=	\int \frac{1}{\underbrace{\ln x}_{1/u}}\underbrace{\frac1x\ dx}_{du} \\
&= \int \frac1u\ du \\
&= \ln |u| + C \\
&= \ln | \ln x| + C.
\end{align*}
The final answer is interesting; the natural log of the natural log. Take the derivative to confirm this answer is indeed correct.
\end{example} % EXAMPLE

The next two examples will help fill in some missing pieces of our antiderivative knowledge. We know the antiderivatives of the sine and cosine functions; what about the other standard functions tangent, cotangent, secant and cosecant? We discover these next.

\begin{example} \label{eg:4.6.usub5} % EXAMPLE
Evaluate $\ds \int \tan(x) \ dx.$

\solution
The previous paragraph established that we did not know the antiderivatives of tangent, hence we must assume that we have learned something in this section that  can help us evaluate this indefinite integral. 

Rewrite $\tan(x)$ as $\frac{\sin(x)}{\cos(x)}$. While the presence of a composition of functions may not be immediately obvious, recognize that $\cos(x)$ is ``inside'' the $\frac{1}{x}$ function. Therefore, we see if setting $u = \cos(x)$ returns usable results. We have that $du = -\sin(x) \ dx$, hence $-du = \sin(x) \ dx$. We can integrate:
\begin{align*}
\int \tan(x) \ dx &= \int \frac{\sin(x)}{\cos(x)}\ dx \\
&= \int \frac1{\underbrace{\cos(x)}_u}\underbrace{\sin(x) \ dx}_{-du} \\
&= \int \frac {-1}u \ du\\
&= -\ln |u| + C \\
&= -\ln |\cos(x)| + C.
\end{align*}
Some texts prefer to bring the $-1$ inside the logarithm as a power of $\cos(x)$, as in:
\begin{align*}
-\ln |\cos(x)| + C &= \ln |(\cos(x))^{-1}| + C\\
&= \ln \left| \frac{1}{\cos(x)}\right| + C\\
&= \ln |\sec(x)| + C.
\end{align*}
Thus the result they give is $\int \tan x \ dx = \ln|\sec(x)| + C$. These two answers are equivalent.
\end{example} % EXAMPLE

\begin{example} \label{eg:4.6.usub6} % EXAMPLE
Evaluate $\ds\int \sec(x) \ dx$.

\solution
This example employs a wonderful trick: multiply the integrand by ``$1$'' so that we see how to integrate more clearly. In this case, we write ``$1$'' as
\[ 1 = \frac{\sec x + \tan x}{\sec x + \tan x}. \]
This may seem like it came out of left field, but it works beautifully. Consider:
\begin{align*}
\int \sec x\ dx	&=	\int \sec x\cdot \frac{\sec x + \tan x}{\sec x + \tan x}\ dx \\
&= \int \frac{\sec^2 x + \sec x\tan x}{\sec x + \tan x}\ dx.\\
\intertext{Now let $u = \sec x+\tan x$; this means $du = (\sec x\tan x+ \sec^2 x)\ dx$, which is our numerator. Thus:}
&= \int \frac{du}{u} \\
&= \ln |u| + C \\
&= \ln |\sec x+\tan x| + C.
\end{align*}
\end{example} % EXAMPLE

We can use similar techniques to those used in Examples \ref{ex_sub6} and \ref{ex_sub7} to find antiderivatives of $\cot x$ and $\csc x$ (which the reader can explore in the exercises.) We summarize our results here.

%\concept{Antiderivatives of Trigonometric Functions}
%{\begin{minipage}{.45\specialboxlength}\small\index{integration!of trig. functions}
%\begin{enumerate}
%\item $\ds \int \sin x \ dx = -\cos x +C$
%\item $\ds\int \cos x\ dx = \sin x + C$
%\item $\ds \int \tan x\ dx = -\ln|\cos x|+C$
%\end{enumerate}
%\end{minipage}
%\begin{minipage}{.55\specialboxlength}\small
%\begin{enumerate}\addtocounter{enumi}{3}
%\item $\ds \int \csc x \ dx = -\ln|\csc x+\cot x| +C$
%\item $\ds\int \sec x\ dx = \ln|\sec x+\tan x| + C$
%\item $\ds \int \cot x\ dx = \ln|\sin x|+C$
%\end{enumerate}
%\end{minipage}
%}

%%\begin{activity} \label{A:5.3.2}  Evaluate each of the following definite integrals by using Trigonometric substitution.
\ba
	\item $\ds \int_0^4 \left(16-x^2 \right)^5 \, dx$
	\item $\ds \int_0^{3/2} x \sqrt{4x^2+9} \, dx$
	\item $\ds \int_0^{\sqrt{2}/3} \frac{1}{\sqrt{9x^2-2}} \,dx$
\ea
\end{activity}
\begin{smallhint}
\ba
	\item Small hints for each of the prompts above.
\ea
\end{smallhint}
\begin{bighint}
\ba
	\item Big hints for each of the prompts above.
\ea
\end{bighint}
\begin{activitySolution}
\ba
	\item Solutions for each of the prompts above.
\ea
\end{activitySolution}
\aftera % ACTIVITY

%%%%%%%%%%%%%%%%%%%%%%%%%%%%%%%%%%%%%%%%%%%%%%%%%
%%***In the same way that definite integrals are linear operators, so are indefinite integrals ***

%-----------------------------------------------------
% SUBSECTION U-SUBSTITUTION WITH DEFINITE INTEGRALS
%-----------------------------------------------------

\subsection*{Evaluating Definite Integrals via $u$-substitution}

This section has focused on $u$-substitution as a means to evaluate indefinite integrals of functions that can be written, up to a constant multiple, in the form $f(g(x))g'(x)$.  However, much of the time integration is used in the context of definite integrals involving such functions. We need to be careful with the corresponding limits of integration.  Consider, for instance, the definite integral
\[ \int_2^5 xe^{x^2} \ dx. \]
Whenever we write a definite integral, it is implicit that the limits of integration correspond to the variable of integration.  To be more explicit, observe that
\[ \int_2^5 xe^{x^2} \ dx = \int_{x=2}^{x=5} xe^{x^2} \ dx. \]
When we execute a $u$-substitution, we change the \emph{variable} of integration; it is essential to note that this also changes the \emph{limits} of integration.  For instance, with the substitution $u = x^2$ and $du = 2x \, dx$, it also follows that when $x = 2$, $u = (2)^2 = 4$, and when $x = 5$, $u = (5)^2 = 25.$  Thus, under the change of variables of $u$-substitution, we now have
\begin{eqnarray*}
\int_{x=2}^{x=5} xe^{x^2} \ dx & = & \int_{u=4}^{u=25} e^{u} \cdot \frac{1}{2} \ du \\
& = & \left. \frac{1}{2}e^u \right|_{u=4}^{u=25} \\
& = & \frac{1}{2}e^{25} - \frac{1}{2}e^4.
\end{eqnarray*}

Alternatively, we could consider the related indefinite integral $\int_2^5 xe^{x^2} \, dx,$ find the antiderivative $\frac{1}{2}e^{x^2}$ through $u$-substitution, and then evaluate the original definite integral.  From that perspective, we'd have
\begin{eqnarray*}
\int_{2}^{5} xe^{x^2} \, dx & = & \left. \frac{1}{2}e^{x^2} \right|_{2}^{5} \\
& = & \frac{1}{2}e^{25} - \frac{1}{2}e^4,
\end{eqnarray*}
which is, of course, the same result.

Using the notation of \ref{usub}, we can restate this as the following concept.

\concept{Substitution with Definite Integrals} %CONCEPT
{Let $f$ and $g$ be differentiable functions, where the range of $g$ is an interval $I$ that contains the domain of $F$. Then \index{integration!definite!and substitution}\index{definite integral!and substitution}
\[ \int_a^b F'\big(g(x)\big)g'(x)\ dx = \int_{g(a)}^{g(b)} F'(u)\ du. \]
} % end concept

Let's look at more examples.

\begin{example} \label{Ex:4.6.sub7}
Evaluate $\ds\int_0^2 \cos(3x-1)\ dx$.

\solution
In this example, we begin with
\[ u = 3x - 1 \quad \mbox{and} \quad \frac{du}{dx} = 3 \Rightarrow \frac{1}{3}du = dx. \]
Rewriting the bounds of integration with respect to $u$, we have $u(2) = 3 \cdot 2 - 1 = 5$ and $u(0) = 3 \cdot 0 - 1 = -1$.   We now evaluate the definite integral:
\begin{align*}
\int_1^2 \cos(3x-1) \ dx &=	\int_{-1}^5 \cos(u) \cdot \frac{1}{3} \ du \\
&= \frac{1}{3} \sin(u) \Big|_{-1}^5 \\
&= \frac{1}{3}\big( \sin(5) - \sin(-1) \big)\\
&\approx -0.039.
\end{align*}
Notice how once we converted the integral to be in terms of $u$, we never went back to using $x$.

%The graphs in Figure \ref{fig:subst12} tell more of the story. In (a) the area defined by the original integrand is shaded, whereas in (b) the area defined by the new integrand is shaded. In this particular situation, the areas look very similar; the new region is ``shorter'' but ``wider,'' giving the same area.
\end{example} % EXAMPLE

\begin{example} \label{eg:4.6.sub8} 
Evaluate $\ds \int_0^{\pi/2} \sin(x) \cos(x) \ dx$.

\solution
We saw the corresponding indefinite integral in Example \ref{eg:4.6.usub3}. In that example we set $u = \sin(x)$ but stated that we could have let $u = \cos(x)$. For variety, we do the latter here.

Let $u = g(x) = \cos(x)$, giving $du = -\sin(x) \ dx$. The new upper bound is $g(\pi/2) = 0$; the new lower bound is $g(0) = 1$. Note how the lower bound is actually larger than the upper bound now. We have
\begin{align*}
\int_0^{\pi/2} \sin(x) \cos(x)\ dx &= \int_1^0 u \ (-1) du \\
&= \int_1^0 -u \ du \quad \text{\scriptsize (switch bounds \& change sign)}\\
&=	\int_0^1 u \ du\\
&= \frac{1}{2} u^2 \Big|_0^1\\
&= 1/2.
\end{align*}
\end{example} % EXAMPLE

%%\input{activities/5.3.Act3} % ACTIVITY

\begin{summary}
\item To begin to find algebraic formulas for antiderivatives of more complicated algebraic functions, we need to think carefully about how we can reverse known differentiation rules.  To that end, it is essential that we understand and recall known derivatives of basic functions, as well as the standard derivative rules.

\item The indefinite integral provides notation for antiderivatives.  When we write ``$\int f(x) \, dx$,'' we mean ``the general antiderivative of $f$.''  In particular, if we have functions $f$ and $F$ such that $F' = f$, the following two statements say the exact thing:
\[ \frac{d}{dx}[F(x)] = f(x) \ \mbox{and} \ \int f(x) \, dx = F(x) + C. \]
That is, $f$ is the derivative of $F$, and $F$ is an antiderivative of $f$.

\item The technique of $u$-substitution helps us evaluate indefinite integrals of the form $\int f(g(x))g'(x) \, dx$ through the substitutions $u = g(x)$ and $du = g'(x) \, dx$, so that
\[ \int f(g(x))g'(x) \, dx = \int f(u) \, du. \]
A key part of choosing the expression in $x$ to let be represented by $u$ is the identification of a function-derivative pair.  To do so, we often look for an ``inner'' function $g(x)$ that is part of a composite function, while investigating whether $g'(x)$ (or a constant multiple of $g'(x)$) is present as a multiplying factor of the integrand.
\end{summary}

%\input{exercises/5.3.Substitution(Ex)} 

\cleardoublepage