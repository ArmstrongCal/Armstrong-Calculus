
%\chapter{Integration}

%\section{Area under a curve}
%blah
%
%\subsection*{Recall Chapter 0 and the area under piece-wise linear curves}
%
%\subsection*{Use rectangles to approximate area--$L_n$, $R_n$ and $M_n$}
%
%\subsection*{Introduce summation notation}
%	
%\section{Riemann Sums}
%blah
%
%\subsection*{Net area}
%	
%\section{Definite Integrals}
%blah
%
%\subsection*{Definition of the definite integral}
%
%\subsection*{Properties of definite integrals}
%
%\subsection*{Using geometry and limits to evaluate definite integrals}
%
%blah
%	
%\section{Fundamental Theorem of Calculus, part 1}
%blah
%
%\subsection*{Area functions}
%	
%\section{Antiderivatives and Fundamental Theorem of Calculus, part 2}
%blah
%
%\section{Substitution}
%blah
%
%\section{Integration by parts}
%blah
%	
%\section{Trigonometric Integrals}
%blah
%	
%\section{Trigonometric Substitution}
%blah
%	
%\section{Partial Fractions}
%blah
%
%\section{Numerical Integration}
%blah
%	
%\section{Improper Integrals}
%blah
