\begin{activity} \label{8.3.Act8} Consider the series defined by
\begin{equation} \label{eq:8.3_ratio_example}
\sum_{k=1}^{\infty} \frac{2^k}{3^k-k}.
\end{equation}
This series is not a geometric series, but this activity will illustrate how we might compare this series to a geometric one. Recall that a series $\sum a_k$ is geometric if the ratio $\frac{a_{k+1}}{a_k}$ is always the same. For the series in (\ref{eq:8.3_ratio_example}), note that $a_k = \frac{2^k}{3^k-k}$.
\ba
\item  To see if $\sum \frac{2^k}{3^k-k}$ is comparable to a geometric series, we analyze the ratios of successive terms in the series. Complete Table \ref{T:8.3.3_ratio_test}, listing your calculations to at least 8 decimal places.
\begin{table}[ht]
\begin{center}
\renewcommand{\arraystretch}{1.5}
\begin{tabular}{c|p{2in}}
$k$   & $\frac{a_{k+1}}{a_k}$ \\
5   & \\
10   &  \\
20   &  \\
21   &  \\
22   &  \\
23  & \\
24   &  \\
25   &  \\
\end{tabular}
\label{T:8.3.3_ratio_test}
\caption{Ratios of successive terms in the series $\sum \frac{2^k}{3^k-k}$}
\end{center}
\end{table}

\item Based on your calculations in Table \ref{T:8.3.3_ratio_test}, what can we say about the ratio $\frac{a_{k+1}}{a_k}$ if $k$ is large?

\item Do you agree or disagree with the statement: ``the series $\sum \frac{2^k}{3^k-k}$ is approximately geometric when $k$ is large''? If not, why not? If so, do you think the series $\sum \frac{2^k}{3^k-k}$ converges or diverges? Explain.



\ea
\end{activity}

\begin{smallhint}
\ba
	\item Small hints for each of the prompts above.
\ea
\end{smallhint}
\begin{bighint}
\ba
	\item Big hints for each of the prompts above.
\ea
\end{bighint}
\begin{activitySolution}
\ba
	\item Ratios of successive summands in the series are shown (to 10 decimal places) in the following table. 
%\begin{table}[ht]
\begin{center}
\renewcommand{\arraystretch}{1.5}
\begin{tabular}{c|p{2in}}
$k$   & $\frac{a_{k+1}}{a_k}$ \\
5   & 0.6583679115 \\
10   & 0.6665951585 \\
20   & 0.6666666642 \\
21   & 0.6666666658 \\
22   & 0.6666666664  \\
23  & 0.6666666666 \\
24   & 0.6666666666 \\
25   & .6666666667 \\
\end{tabular}
%\label{T:8.3.3_ratio_test_b}
%\caption{Ratios of successive summands in the series $\sum \frac{2^k}{3^k-k}$}
\end{center}
%\end{table}
    \item The calculations in the table in part (a) seem to indicate that the ratio $\frac{a_{k+1}}{a_k}$ is roughly $\frac{2}{3}$ when $k$ is large.
    \item Since $\frac{a_{k+1}}{a_k} \approx \frac{2}{3}$ for large $k$, the series $\sum \frac{2^k}{3^k-k}$ is approximately the same as $\sum \left(\frac{2}{3}\right)^k$ when $k$ is large. So the series $\sum \frac{2^k}{3^k-k}$ is approximately geometric with ratio $\frac{2}{3}$ when $k$ is large. Since the series $\sum \left(\frac{2}{3}\right)^k$ converges because the ratio is less than 1, we expect that the series $\sum \frac{2^k}{3^k-k}$ will also converge.
\ea
\end{activitySolution}
\aftera 