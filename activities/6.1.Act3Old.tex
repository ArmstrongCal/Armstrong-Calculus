\begin{activity} \label{A:6.1.3}  Each of the following questions somehow involves the arc length along a curve.
\ba
	
	\item Use the definition and appropriate computational technology to determine the arc length along $y = x^2$ from $x = -1$ to $x = 1$.
	\item Find the arc length of $y = \sqrt{4-x^2}$ on the interval $0 \le x \le 4$.  Find this value in two different ways: (a) by using a definite integral, and (b) by using a familiar property of the curve.
	\item Determine the arc length of $y = xe^{3x}$ on the interval $[0,1]$.
	\item Will the integrals that arise calculating arc length typically be ones that we can evaluate exactly using the First FTC, or ones that we need to approximate?  Why?
	\item A moving particle is traveling along the curve given by $y = f(x) = 0.1x^2 + 1$, and does so at a constant rate of 7 cm/sec, where both $x$ and $y$ are measured in cm (that is, the curve $y = f(x)$ is the path along which the object actually travels; the curve is not a ``position function'').  Find the position of the particle when $t = 4$ sec, assuming that when $t = 0$, the particle's location is $(0,f(0))$.
\ea

\end{activity}
\begin{smallhint}
\ba
	\item Small hints for each of the prompts above.
\ea
\end{smallhint}
\begin{bighint}
\ba
	\item Big hints for each of the prompts above.
\ea
\end{bighint}
\begin{activitySolution}
\ba
	\item Solutions for each of the prompts above.
\ea
\end{activitySolution}
\aftera