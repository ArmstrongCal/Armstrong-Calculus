\begin{activity} \label{A:2.5.7}  Answer each of the following questions.  Where a derivative is requested, be sure to label the derivative function with its name using proper notation.
\ba
	\item  Let $f(x) = 5 \sec(x) - 2\csc(x)$.  Find the slope of the tangent line to $f$ at the point where $x =\frac{\pi}{3}$.
	\item Let $p(z) = z^2\sec(z) - z\cot(z)$.  Find the instantaneous rate of change of $p$ at the point where $z = \frac{\pi}{4}$.
        \item Let $h(t) = \displaystyle \frac{\tan (t)}{t^2+1} - 2e^t \cos(t)$.  Find $h'(t)$.
        \item Let $g(r) = \displaystyle \frac{r \sec(r) }{5^r}$.  Find $g'(r)$.
        \item When a mass hangs from a spring and is set in motion, the object's position oscillates in a way that the size of the oscillations decrease.  This is usually called a \emph{damped oscillation}.  Suppose that for a particular object, its displacement from equilibrium (where the object sits at rest) is modeled by the function $$s(t) = \frac{15 \sin(t)}{e^t}.$$
Assume that $s$ is measured in inches and $t$ in seconds.  Sketch a graph of this function for $t \ge 0$ to see how it represents the situation described.  Then compute $ds/dt$, state the units on this function, and explain what it tells you about the object's motion.  Finally, compute and interpret $s'(2)$.
\ea
\end{activity}
\begin{smallhint}
\ba
	\item  What rule(s) can help you determine $f'(x)$?
	\item Note that $p$ is a sum of two functions.  What rule is needed to differentiate each term in the sum?
	\item Observe that $h$ is a sum of two functions; the first term in the sum is a quotient, while the second is a product.
        \item What is the overall structure of $g$?  What is the algebraic structure of the numerator of $g$?
        \item Keep in mind that the derivative of position tells us the instantaneous velocity.
\ea
\end{smallhint}
\begin{bighint}
\ba
	\item Use the sum and constant multiple rules help you determine $f'(x)$.
	\item Note that $p$ is a sum of two functions and that each term in the sum is a product.
	\item Observe that $h$ is a sum of two functions; the first term in the sum is a quotient, while the second is a product.
        \item What is the overall structure of $g$?  What is the algebraic structure of the numerator of $g$?
        \item Keep in mind that the derivative of position tells us the instantaneous velocity, and observe that differentiating $s$ will require the quotient rule.
\ea
\end{bighint}
\begin{activitySolution}
\ba
	\item Using the sum and constant multiple rules along with the formulas for the derivatives of $\sec(x)$ and $\csc(x)$, we find that
	$$f'(x) = 5 \sec(x)\tan(x) + 2\csc(x)\cot(x).$$  Therefore, the slope of the tangent line to $f$ at the point where $x =\frac{\pi}{3}$ is given by $m = f'(\frac{\pi}{3}) =  5 \sec(\frac{\pi}{3})\tan(\frac{\pi}{3}) + 2\csc(\frac{\pi}{3})\cot(\frac{\pi}{3}) = 5 \cdot 2 \cdot \sqrt{3} + 2 \cdot \frac{2}{\sqrt{3}} \cdot \frac{1}{\sqrt{3}} = 10\sqrt{3} + \frac{4}{3}.$
	\item By the sum rule and two applications of the product rule, we have
	\begin{eqnarray*}
	p'(z) & = & \frac{d}{dz}[z^2\sec(z)] - \frac{d}{dz}[z\cot(z)] \\  
	       & = &  [z^2 \sec(z) \tan(z) + \sec(z) \cdot 2z] - [z(-\csc^2(z)) + \cot(z) \cdot 1] \\
	       & = &  z^2 \sec(z) \tan(z) + 2z\sec(z) + z\csc^2(z) - \cot(z).
	\end{eqnarray*}
	Thus, the instantaneous rate of change of $p$ at the point where $z = \frac{\pi}{4}$ is
	\begin{eqnarray*}
	p'(\frac{\pi}{4}) & = &  (\frac{\pi}{4})^2 \sec(\frac{\pi}{4}) \tan(\frac{\pi}{4}) + 2\frac{\pi}{4}\sec(\frac{\pi}{4}) + \frac{\pi}{4}\csc^2(\frac{\pi}{4}) - \cot(\frac{\pi}{4}) \\
	 		     & = &  \frac{\pi^2}{16} \sqrt{2} + 2\frac{\pi}{4}\sqrt{2} + \frac{\pi}{4}2 - 1 \\
			     & = &  \frac{\pi^2}{16} \sqrt{2} + \frac{\sqrt{2}\pi}{2} + \frac{\pi}{2} - 1 
	\end{eqnarray*}
        \item Using the sum and constant multiple rules, followed by the quotient rule on the first term and the product rule on the second, we find that
        \begin{eqnarray*}
        		h'(t) & = & \frac{d}{dt}\left[ \frac{\tan (t)}{t^2+1} \right] - 2\frac{d}{dt}\left[e^t \cos(t)\right] \\
		       & = & \frac{(t^2+1) \sec^2(t) - \tan(t) (2t)}{(t^2 + 1)^2} - 2(e^t(-\sin(t)) + \cos(t)e^t) \\
		       & = &  \frac{(t^2+1) \sec^2(t) - 2t \tan(t)}{(t^2 + 1)^2} + 2e^t \sin(t) - 2 e^t\cos(t)
	\end{eqnarray*} 
        \item Note that $g$ is fundamentally a quotient, so we need to use the quotient rule.  But the numerator of $g$ is a product, so the product rule will be required to compute the derivative of the top function.  Executing the quotient rule and proceeding, we find that
        \begin{eqnarray*}
        g'(r) & = & \frac{5^r \frac{d}{dr}[r \sec(r)] - r\sec(r) \cdot 5^r \ln(5) }{(5^r)^2}\\
      		& = & \frac{5^r [r \sec(r) \tan(r) + \sec(r) \cdot 1] - r\sec(r) \cdot 5^r \ln(5) }{(5^r)^2}\\
		& = & \frac{r \sec(r) \tan(r) + \sec(r) - r 5^r \sec(r)}{5^r}
	\end{eqnarray*}
        \item By the quotient rule, 
        $$\frac{ds}{dt} = \frac{e^t \cdot 15 \cos(t) - 15\sin(t) \cdot e^t}{(e^t)^2} = \frac{15\cos(t) - 15\sin(t)}{e^t}.$$
The function $\frac{ds}{dt} = s'(t)$ measures the instantaneous vertical velocity of the mass that is attached to the spring.  In particular, $s'(2) = \frac{15\cos(2) - 15\sin(2)}{e^2} \approx -2.69$ inches per second, which tells us at the instant $t = 2$, the mass is moving downward at an instantaneous rate of 2.69 inches per second.
\ea
\end{activitySolution}
\aftera