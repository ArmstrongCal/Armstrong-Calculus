\begin{activity}  \label{A:0.2.2}
Each of the following questions concern  $s(t) = 64 - 16(t-1)^2$, the position function from Preview Activity \ref{PA:0.2}.
\ba
	\item Compute the average velocity of the ball on the time interval $[1.5,2]$.  What is different between this value and the average velocity on the interval $[0,0.5]$?
	\item Use appropriate computing technology to estimate the instantaneous velocity of the ball at $t = 1.5$.  Likewise, estimate the instantaneous velocity of the ball at $t = 2$.  Which value is greater?
	\item How is the sign of the instantaneous velocity of the ball related to its behavior at a given point in time?  That is, what does positive instantaneous velocity tell you the ball is doing?  Negative instantaneous velocity?
	\item Without doing any computations, what do you expect to be the instantaneous velocity of the ball at $t = 1$?  Why?
\ea
\end{activity}
\begin{smallhint}
\ba
	\item Remember to use the formula for average velocity from above:  $AV_{[a,b]} = \frac{s(b)-s(a)}{b-a}$.  Think carefully about whether certain quantities are positive or negative.
	\item To estimate the instantaneous velocity at $t = 1.5$, consider average velocities on the intervals $[1.499,1.5]$ and $[1.5,1.501]$.
	\item Think about whether the ball is rising or falling.
	\item What is the average velocity of the ball on small intervals that contain $t = 0$?
\ea
\end{smallhint}
\begin{bighint}
\ba
	\item $AV_{[1.5,2]} = \frac{s(2)-s(1.5)}{2-1.5}$.  Your result should be negative since $s(2) < s(1.5)$.
	\item To estimate the instantaneous velocity at $t = 1.5$, consider average velocities on the intervals $[1.499,1.5]$ and $[1.5,1.501]$.
	\item You should find in (a) that the instantaneous velocity at $t = 1.5$ is negative, while earlier we found that the instantaneous velocity at $t = 0.5$ is positive.  How are these signs connected to whether the ball is rising or falling?
	\item Think about the line through the points $(0.999,s(0.999))$ and $(1,s(1))$ will look like given the ``special'' role of $(1,s(1))$ on the graph of $s(t)$.
\ea
\end{bighint}
\begin{activitySolution}
\ba
	\item $AV_{[1.5,2]} = \frac{s(2)-s(1.5)}{2-1.5} = -24$ ft/sec.  We note that this average velocity is negative, and in fact is the opposite of the average velocity of 24 ft/sec on the interval $[0,0.5]$.
	\item Since $AV_{[1.499,1.5]} = -15.984$ and $AV_{[1.5, 1.501]} = -16.016$, it appears that the instantaneous velocity of the ball at $t = 1.5$ is approximately $-16$ ft/sec.  Similar computations show that at $t = 2$, it appears the instantaneous velocity is about $-32$ ft/sec.  Note that $-16>-32$, so the instantaneous velocity at $t = 1.5$ is greater because it is ``less negative.'' Asking which number is ``greater'' is different from asking which number is ``more negative.''
	\item When the ball is rising, its instantaneous velocity is positive, while when the ball is falling, its instantaneous velocity is negative.
	\item Note that $(1,s(1))$ is the vertex of the parabola given by $s(t)$.  At this point, the ball is neither rising nor falling.  On intervals of the form $[a,1]$, where $a < 1$, the average velocity of the ball is positive; on intervals of form $[1,b]$, where $b > 1$, the average velocity is positive.  Hence we expect the instantaneous velocity of the ball at the moment $t = 1$ to be zero.
\ea
\end{activitySolution}
\aftera
