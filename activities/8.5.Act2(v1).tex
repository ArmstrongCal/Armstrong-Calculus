\begin{activity} \label{8.5.Act2} We have just seen that the $n$th order Taylor polynomial centered at $a = 0$ for the exponential function $e^x$ is
 \[\sum_{k=0}^{n} \frac{x^k}{k!}.\]
In this activity, we determine the $n$th order Taylor polynomials for several other familiar functions. 
\ba
\item Let $f(x) = \frac{1}{1-x}$.
    \begin{itemize}
    \item[(i)] Calculate the first four derivatives of $f(x)$ at $x=0$. Then find a general formula for $f^{(k)}(0)$.

    \item[(ii)]  Determine the $n$th order Taylor polynomial for $f(x) = \frac{1}{1-x}$ centered at $x=0$.

    \end{itemize}


\item Let $f(x) = \cos(x)$.
    \begin{itemize}
    \item[(i)] Calculate the first four derivatives of $f(x)$ at $x=0$. Then find a general formula for $f^{(k)}(0)$.  (Think about how $k$ being even or odd affects the value of the $k$th derivative.)

    \item[(ii)] Determine the $n$th order Taylor polynomial for $f(x) = \cos(x)$ centered at $x=0$. 

    \end{itemize}

\item Let $f(x) = \sin(x)$.
    \begin{itemize}
    \item[(i)] Calculate the first four derivatives of $f(x)$ at $x=0$. Then find a general formula for $f^{(k)}(0)$.  (Think about how $k$ being even or odd affects the value of the $k$th derivative.)

    \item[(ii)] Determine the $n$th order  Taylor polynomial for $f(x) = \sin(x)$ centered at $x=0$.  Hint: Your answer should depend on whether $n$ is even or odd.

    \end{itemize}


 \ea


\end{activity}

\begin{smallhint}
\ba
	\item Small hints for each of the prompts above.
\ea
\end{smallhint}
\begin{bighint}
\ba
	\item Big hints for each of the prompts above.
\ea
\end{bighint}
\begin{activitySolution}
\ba
	\item
    \btl
    \item The first four derivatives of $f(x)$ at $x=0$ are
    \begin{align*}
f(x) &= \cos(x) & \hspace{0.5in} & f(0) &= 1 \\
f'(x) &= -\sin(x) & \hspace{0.5in} & f'(0) &= 0 \\
f''(x) &= -\cos(x) & \hspace{0.5in} & f''(0) &= -1 \\
f^{(3)}(x) &= \sin(x) & \hspace{0.5in} & f^{(3)}(0) &= 0 \\
f^{(4)}(x) &= \cos(x) & \hspace{0.5in} & f^{(4)}(0) &= 1. \\
\end{align*}
It appears that the odd derivatives of $f(x)$ are all plus or minus $\sin(x)$ and so have values of 0 at $x=0$ and the even derivatives are $\pm \cos(x)$ and have alternating values of 1 and $-1$ at $x-0$. Since the even numbers can be represented in the form $2k$ where $k$ is an integer we have
\[f^{k}(0) = 0 \text{ if } k \text{ is odd } \ \ \ \text{ and } \ \ \ f^{2k}(0) = (-1)^k.\]


    \item Based on the previous part of this problem the $n$th order Taylor polynomial for $\cos(x)$ is
    \[1 - \frac{x^2}{2} + \frac{x^4}{4!} - \frac{x^6}{6!} + \cdots + (-1)^{n/2}\frac{x^n}{n!}\]
    if $n$ is even and
    \[1 - \frac{x^2}{2} + \frac{x^4}{4!} - \frac{x^6}{6!} + \cdots + (-1)^{(n-1)/2}\frac{x^(n-1)}{(n-1)!}\]
    if $n$ is odd.

    \etl

\item
    \btl
    \item The first four derivatives of $f(x)$ at $x=0$ are
    \begin{align*}
f(x) &= \sin(x) & \hspace{0.5in} & f(0) &= 0 \\
f'(x) &= \cos(x) & \hspace{0.5in} & f'(0) &= 1 \\
f''(x) &= -\sin(x) & \hspace{0.5in} & f''(0) &= 0 \\
f^{(3)}(x) &= -\cos(x) & \hspace{0.5in} & f^{(3)}(0) &= -1 \\
f^{(4)}(x) &= \sin(x) & \hspace{0.5in} & f^{(4)}(0) &= 0. \\
\end{align*}
It appears that the even derivatives of $f(x)$ are all plus or minus $\sin(x)$ and so have values of 0 at $x=0$ and the odd derivatives are $\pm \cos(x)$ and have alternating values of 1 and $-1$ at $x-0$. Since the odd numbers can be represented in the form $2k+1$ where $k$ is an integer we have
\[f^{k}(0) = 0 \text{ if } k \text{ is even } \ \ \ \text{ and } \ \ \ f^{2k+1}(0) = (-1)^k.\]

    \item Based on the previous part of this problem the $n$th order Taylor polynomial for $\sin(x)$ is
    \[x - \frac{x^3}{3!} + \frac{x^5}{5!} - \frac{x^7}{7!} + \cdots + (-1)^{(n-1)/2}\frac{x^n}{n!}\]
    if $n$ is odd and
    \[x - \frac{x^3}{3!} + \frac{x^5}{5!} - \frac{x^7}{7!} + \cdots + (-1)^{n/2+1}\frac{x^{n-1}}{(n-1)!}\]
    if $n$ is even.

    \etl

\item
    \btl
    \item The first four derivatives of $f(x)$ at $x=0$ are
    \begin{align*}
f(x) &= \frac{1}{1-x} & \hspace{0.5in} & f(0) &= 1 \\
f'(x) &= \frac{1}{(1-x)^2} & \hspace{0.5in} & f'(0) &= 1 \\
f''(x) &= \frac{2}{(1-x)^3} & \hspace{0.5in} & f''(0) &= 2 \\
f^{(3)}(x) &= \frac{3!}{(1-x)^4} & \hspace{0.5in} & f^{(3)}(0) &= 3! \\
f^{(4)}(x) &= \frac{4!}{(1-x)^5} & \hspace{0.5in} & f^{(4)}(0) &= 4!. \\
\end{align*}
It appears that the pattern is
\[f^{(k)}(0) = k!.\]

    \item  Determine the $n$ order Taylor polynomial for $f(x) = \frac{1}{1-x}$ centered at $x=0$.

The $n$th order Taylor polynomial for $f$ at $x=0$ is
\[\sum_{k=0}^n \frac{f^{(k)}}{k!} x^k = \sum_{k=0}^n \frac{k!}{k!} x^k = \sum_{k=0}^n  x^k.\]
This makes sense since $f(x)$ is the sum of the geometric series with ratio $x$, so the $n$th order Taylor polynomial should just be the $n$th partial sum of this geometric series.

    \etl
\ea
\end{activitySolution}
\aftera 