\begin{activity} \label{A:2.4.Act5}
Let $f(x) = a^x$.  The goal of this problem is to explore how the value of $a$ affects the derivative of $f(x)$, without assuming we know the rule for $\frac{d}{dx}[a^x]$ that we have stated and used in earlier work in this section.
\ba
	\item Use the limit definition of the derivative to show that
	$$f'(x) = \lim_{h \to 0} \frac{a^x \cdot a^h - a^x}{h}.$$
	\item Explain why it is also true that
	$$f'(x) = a^x \cdot \lim_{h \to 0} \frac{a^h - 1}{h}.$$
	\item Use computing technology and small values of $h$ to estimate the value of 
	$$L = \lim_{h \to 0} \frac{a^h - 1}{h}$$
	when $a = 2$.  Do likewise when $a = 3$.
	\item Note that it would be ideal if the value of the limit $L$ was $1$, for then $f$ would be a particularly special function:  its derivative would be simply $a^x$, which would mean that its derivative is itself.  By experimenting with different values of $a$ between $2$ and $3$, try to find a value for $a$ for which 
	$$L = \lim_{h \to 0} \frac{a^h - 1}{h} = 1.$$
	\item Compute $\ln(2)$ and $\ln(3)$.  What does your work in (b) and (c) suggest is true about $\frac{d}{dx}[2^x]$ and $\frac{d}{dx}[2^x]$.
	\item How do your investigations in (d) lead to a particularly important fact about the number $e$?
\ea

\end{activity} 