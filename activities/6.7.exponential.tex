\begin{activity} \label{A:7.6.1}  
  Our first model will be based on the following assumption:
  \begin{quote}{\em
      The rate of change of the population is proportional to the
      population.  
    }
  \end{quote}

  On the face of it, this seems pretty reasonable.  When there is a
  relatively small number of people, there will be fewer births and
  deaths so the rate of change will be small.  When there is a larger
  number of people, there will be more births and deaths so we expect
  a larger rate of change.

  If $P(t)$ is the population $t$ years after the year $2000$, we may
  express this assumption as
  $$
  \frac{dP}{dt} = kP
  $$
  where $k$ is a constant of proportionality.

\ba
\item Use the data in the table to estimate the derivative $P'(0)$
  using a central difference.  Assume that $t=0$ corresponds to the
  year $2000$.

\item What is the population $P(0)$?

\item Use these two facts to estimate the constant of proportionality
  $k$ in the differential equation.

\item Now that we know the value of $k$, we have the initial
  value problem
  $$
    \frac{dP}{dt} = kP, \ P(0) = 6.084.
  $$
  Find the solution to this initial value problem.

\item What does your solution predict for the population in the year
  $2010$?  Is this close to the actual population given in the table?

\item When does your solution predict that the population will reach
  $12$ billion?

\item What does your solution predict for the population in the year
  $2500$? 

\item Do you think this is a reasonable model for the earth's
  population?  Why or why not?  Explain your thinking using a couple
  of complete sentences. 

\ea
\end{activity}
\begin{smallhint}
\ba
	\item Small hints for each of the prompts above.
\ea
\end{smallhint}
\begin{bighint}
\ba
	\item Big hints for each of the prompts above.
\ea
\end{bighint}
\begin{activitySolution}
\ba
	\item Solutions for each of the prompts above.
\ea
\end{activitySolution}
\aftera