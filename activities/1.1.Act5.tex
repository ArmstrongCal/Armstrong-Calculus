\begin{activity}  \label{A:1.1.5}
In this activity, we prove $\ds \lim_{\theta \to 0} \frac{\sin(\theta)}{\theta} = 1$ using the squeeze theorem.  
\ba
	\item Draw the part of the unit circle that lies in the first quadrant. Place a point on the circle in the first quadrant and label it $(x,y)$.
	\item Draw the line connecting the origin to the point $(x,y)$ on the same graph. Label the angle between the line and the $x-axis$ as $\theta$.
	\item Find the area of the sector of the unit circle subtended by $\theta$ that you drew above.
	\item Draw a vertical line from the point $(x,y)$ to the $x$-axis. This will form a right triangle. \label{small}
	\item Find the area of the triangle in (\ref{small}) in terms of $\theta$. You will need to use trig functions. 
	\item Draw a line on the picture connecting the points $(1,0)$ and $(1,1)$. This will also form a right triangle. \label{large}
	\item Find the area of the triangle in (\ref{large}). You will need to use trig functions.
	\item What is the relationship between the three areas you found? Write this as an inequality.
	\item Write this inequality so that it involves the function $\frac{\sin(\theta)}{\theta}$.
	\item Now apply the squeeze theorem to get the result.
\ea
\end{activity}

%\begin{marginfigure}
%\margingraphics{figs/1/figSqueeze1.pdf}
%\caption{The unit circle and related triangles.}\label{fig:1-1_Eg4}
%\end{marginfigure}

\begin{activitySolution}
	We begin by considering the unit circle. Each point on the unit circle has coordinates $(\cos (\theta),\sin (\theta))$ for some angle $\theta$ as shown in Figure~\ref{fig:1-1_Eg4}. Using similar triangles, we can extend the line from the origin through the point to the point $(1,\tan (\theta))$, as shown. (Here we are assuming that $0 \leq \theta \leq \pi/2$. Later we will show that we can also consider $\theta \leq 0$.)

The area of the large triangle is $\frac{1}{2} \tan(\theta)$; the area of the sector is $\theta/2$; the area of the triangle contained inside the sector is $\frac{1}{2} \sin(\theta)$. It is then clear from the diagram that 
\[ \frac{\tan (\theta)}{2} \geq \frac{\theta}{2} \geq \frac{\sin (\theta)}{2}.\]

Multiply all terms by $\ds \frac{2}{\sin(\theta)}$, giving 
\[ \frac{1}{\cos(\theta)} \geq \frac{\theta}{\sin (\theta)} \geq 1.\]

Taking reciprocals reverses the inequalities, giving 
\[ \cos (\theta) \leq \frac{\sin(\theta)}{\theta} \leq 1.\]
These inequalities hold for all values of $\theta$ near 0, even negative values, since $\cos (-\theta) = \cos(\theta)$ and $\sin (-\theta) = -\sin (\theta)$.

Now take the limit of everything as $\theta \to 0$.

\[ \lim_{\theta \to 0} \cos (\theta) \leq \lim_{\theta \to 0} \frac{\sin(\theta)}{\theta} \leq \lim_{\theta \to 0}  1 \]
\[ \cos 0 \leq \lim_{\theta \to 0} \frac{\sin(\theta)}{\theta} \leq  1 \]
\[ 1 \leq \lim_{\theta \to 0} \frac{\sin(\theta)}{\theta} \leq  1 \]

By the Squeeze Theorem, $\ds \lim_{\theta \to 0} \frac{\sin(\theta)}{\theta}=1$.
\end{activitySolution}
\aftera
