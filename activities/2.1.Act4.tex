\begin{marginfigure}[8cm]
\margingraphics{figures/1_5_PA1.eps}
\caption{The graph of $y = s(t)$, the position of the car along highway $46$, which tells its distance in miles from Gackle, ND, at time $t$ in minutes.} \label{fig:2.1.Act4}
\end{marginfigure}

\begin{activity}  \label{A:2.1.4}
One of the longest stretches of straight (and flat) road in North America can be found on the Great Plains in the state of North Dakota on state highway $46$, which lies just south of the interstate highway I-$94$ and runs through the town of Gackle.  A car leaves town (at time $t = 0$) and heads east on highway $46$; its position in miles from Gackle at time $t$ in minutes is given by the graph of the function in Figure~\ref{fig:2.1.Act4}.  Three important points are labeled on the graph; where the curve looks linear, assume that it is indeed a straight line.

\ba
	\item In everyday language, describe the behavior of the car over the provided time interval.  In particular, discuss what is happening on the time intervals $[57,68]$ and $[68,104]$.
	\item Find the slope of the line between the points $(57,63.8)$ and $(104,106.8)$.  What are the units on this slope?  What does the slope represent?
	\item Find the average rate of change of the car's position on the interval $[68,104]$.  Include units on your answer.
	\item Estimate the instantaneous rate of change of the car's position at the moment $t = 80$.  Write a sentence to explain your reasoning and the meaning of this value.
\ea
\end{activity}