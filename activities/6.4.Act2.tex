\begin{activity} \label{A:6.4.2} For quantities of equal weight, such as two children on a teeter-totter, the balancing point is found by taking the average of their locations.  When the weights of the quantities differ, we use a weighted average of their respective locations to find the balancing point. 
\ba
	\item Suppose that a shelf is 6 feet long, with its left end situated at $x = 0$.  If one book of weight 1 lb is placed at $x_1 = 0$, and another book of weight 1 lb is placed at $x_2 = 6$, what is the location of $\overline{x}$, the point at which the shelf would (theoretically) balance on a fulcrum?
	\item Now, say that we place four books on the shelf, each weighing 1 lb:  at $x_1 = 0$, at $x_2 = 2$, at $x_3 = 4$, and at $x_4 = 6$.  Find $\overline{x}$, the balancing point of the shelf.
	\item How does $\overline{x}$ change if we change the location of the third book?  Say the locations of the 1-lb books are  $x_1 = 0$, $x_2 = 2$, $x_3 = 3$, and $x_4 = 6$.  
	\item Next, suppose that we place four books on the shelf, but of varying weights:  at $x_1 = 0$ a 2-lb book, at $x_2 = 2$ a 3-lb book, and $x_3 = 4$ a 1-lb book, and at $x_4 = 6$ a 1-lb book.  Use a weighted average of the locations to find $\overline{x}$, the balancing point of the shelf.  How does the balancing point in this scenario compare to that found in (b)?
	\item What happens if we change the location of one of the books?  Say that we keep everything the same in (d), except that $x_3 = 5$.  How does $\overline{x}$ change?  
	\item What happens if we change the weight of one of the books?  Say that we keep everything the same in (d), except that the book at $x_3 = 4$ now weighs 2 lbs.  How does $\overline{x}$ change?
	\item Experiment with a couple of different scenarios of your choosing where you move the location of one of the books to the left, or you decrease the weight of one of the books.
	\item Write a couple of sentences to explain how adjusting the location of one of the books or the weight of one of the books affects the location of the balancing point of the shelf.  Think carefully here about how your changes should be considered relative to the location of the balancing point $\overline{x}$ of the current scenario.
\ea

\end{activity}
\begin{smallhint}
\ba
	\item Small hints for each of the prompts above.
\ea
\end{smallhint}
\begin{bighint}
\ba
	\item Big hints for each of the prompts above.
\ea
\end{bighint}
\begin{activitySolution}
\ba
	\item Solutions for each of the prompts above.
\ea
\end{activitySolution}
\aftera