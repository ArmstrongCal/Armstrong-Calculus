\begin{activity} \label{8.1.Act2}
\ba
\item Recall our earlier work with limits involving infinity in Section~\ref{S:2.8.LHR}.  State clearly what it means for a continuous function $f$ to have a limit $L$ as $x \to \infty$.

\item Given that an infinite sequence of real numbers is a function from the integers to the real numbers, apply the idea from part (a) to explain what you think it means for a sequence $\{s_n\}$ to have a limit as $n \to \infty$.  

\item Based on your response to (b), decide if the sequence $\left\{ \frac{1+n}{2+n}\right\}$ has a limit as $n \to \infty$. If so, what is the limit? If not, why not?

\ea
\end{activity}

\begin{smallhint}
\ba
	\item Small hints for each of the prompts above.
\ea
\end{smallhint}
\begin{bighint}
\ba
	\item Big hints for each of the prompts above.
\ea
\end{bighint}
\begin{activitySolution}
\ba
	\item A continuous function $f$ has a limit $L$ as the independent variable $x$ goes to infinity if we can make the values of $f(x)$ as close to $L$ as we want by choosing $x$ as large as we need. 
    \item We expect that a sequence $\{s_n\}$ will have a limit $L$ as $n$ goes to infinity if we can make the entries $s_n$ in the sequence as close to $L$ as we want by choosing $n$ as large as we need.
    \item As $n$ gets large, the constant terms become infinitesimally small compared to $n$ and so $\frac{1+n}{2+n}$ looks like $\frac{n}{n}$ or 1 for large $n$. So the sequence $\left\{ \frac{1+n}{2+n}\right\}$ has a limit of 1 at infinity.   
\ea
\end{activitySolution}
\aftera 