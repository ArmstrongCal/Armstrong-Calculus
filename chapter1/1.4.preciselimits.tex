\section{Epsilon-Delta Definition of a Limit} \label{S:1.4.precise}

\begin{goals}
\item What do we mean by saying "arbitrarily close"?
\item What is the precise definition of a \emph{limit}?
\item What are $\epsilon$ and $\delta$ and how do they help in determining the value of the limit of a function at a point?
\end{goals}

%-----------------------------------
% SUBSECTION INTRODUCTION
%-----------------------------------
\subsection*{Introduction}

Recall our definition of a limit of a function from Section~\ref{S:1.1.Limits}.

\definition{The Limit of a Function}{ %DEFINITION 
Given a function $f$, a fixed input $x = a$, and a real number $L$, we say that \emph{$f$ has limit\index{limit!definition} $L$ as $x$ approaches $a$}, and write
\[ \lim_{x \to a} f(x) = L \]
provided that we can make $f(x)$ as close to $L$ as we like by taking $x$ sufficiently close (but not equal) to $a$.  If we cannot make $f(x)$ as close to a single value as we would like as $x$ approaches $a$, then we say that \emph{$f$ does not have a limit as $x$ approaches $a$.}
} % end definition

The problem with this definition is that the words "approaches" and "close" are not exact.  In what way does the variable $x$ approach $a$? How "close" do $x$ and $y$ have to be to $a$ and $L$, respectively? Finally, what determines "sufficiently close"? The precise definition we describe in this section comes from formalizing our original definition.  A quick restatement gets us closer to what we want:

\begin{description}
\item ``If $x$ is within a certain \textit{tolerance level} of $a$, then the corresponding value $y=f(x)$ is within a certain \textit{tolerance level} of $L$.''
\end{description}

The accepted notation for the $x$-tolerance is the lowercase Greek letter delta, or $\delta$, and the $y$-tolerance is lowercase epsilon, or $\epsilon$. One more rephrasing nearly gets us to the actual definition:

\begin{description}
\item ``If $x$ is within $\delta$ units of $a$, then the corresponding value of $y$ is within $\epsilon$ units of $L$.''
\end{description}

Note that this means (let the "$\longrightarrow$" represent the word "implies"):
$\ds a - \delta < x < a + \delta \longrightarrow L - \epsilon < y < L + \epsilon$ or  $\ds |x - a| < \delta \longrightarrow  |y - L| < \epsilon$. The point is that $\delta$ and $\epsilon$, being tolerances, can be any positive (but typically small) values.  Finally, we have the formal definition of the limit with the notation  seen in the previous section.

\definition{The Limit of a Function $f$} %DEFINITION 
{Let $f$ be a function defined on an open interval containing $a$. The notation $$\displaystyle \lim_{x\rightarrow a} f(x) = L,$$ read as ``the limit of $f(x)$, as $x$ approaches $a$, is $L$,'' 
means that given any $\epsilon > 0$, there exists $\delta > 0$ such that 
whenever $|x - a| < \delta$, we have $|f(x) - L| < \epsilon$.\index{limit!definition}
}% end definition

Mathematicians often enjoy writing ideas without using any words. Here is the wordless definition of the limit:
{\small \[ \ds \lim_{x \to a} f(x) = L \iff \forall \ \epsilon > 0, \exists \ \delta > 0 \ \mbox{s.t.} \ |x - a| < \delta \longrightarrow |f(x) - L| < \epsilon. \]}
There is an emphasis here that we may have passed over before.  In the definition, the $y$-tolerance $\epsilon$ is given \textit{first} and then the limit will exist {\em if} we can find an $x$-tolerance $\delta$ that works.  

It is time for an example.  Note that the explanation is long, but it will take you through all steps necessary to understand the ideas.

%\ifthenelse{\boolean{longpage}}%
%{\mtable{.4}{Illustrating the $\epsilon-\delta$ process.}{fig:choose_e_d}{%
		%\begin{tabular}{cc} 
		%\myincludegraphics{figures/figLimitProof1a}&
		%\myincludegraphics{figures/figLimitProof1b}
		%\end{tabular}
		%\vskip \baselineskip
		%\parbox{200pt}{\centering With $\epsilon=0.5$, we pick any $\delta < 1.75$.}
		%}
%}
%\mtable{.45}{Illustrating the $\epsilon-\delta$ process.}{fig:choose_e_d}{%
%		\begin{tabular}{c} 
%		\myincludegraphics{figures/figLimitProof1a}\\
%		\myincludegraphics{figures/figLimitProof1b}\\
%		\noindent\parbox{200pt}{With $\epsilon=0.5$, we pick any $\delta < 1.75$.}
%		\end{tabular}
%	}

\begin{marginfigure}[7cm] % MARGIN FIGURE
\captionsetup[subfigure]{labelformat=empty}
\subfloat{\margingraphics{figures/figLimitProof1a}}

\subfloat{\margingraphics{figures/figLimitProof1b}}
\caption{Illustrating the $\epsilon-\delta$ process. With $\epsilon=0.5$, we pick any $\delta < 1.75$.}
\label{fig:1-4_Eg1}
\end{marginfigure}

\begin{example} \label{Ex:1.4.Eg1}
Show that $\ds \lim_{x \to 4} \sqrt{x} = 2 $.

\solution Before we use the formal definition, let's try some numerical tolerances.  What if the $y$ tolerance is $0.5$, or $\epsilon =0.5$?  How close to $4$ does $x$ have to be so that $y$ is within $0.5$ units of $2$ (or $1.5 < y < 2.5$)?  In this case, we can just square these values since $y = \sqrt{x}$ to get
$1.5^2 < x < 2.5^2,$ or 
\[ 2.25 < x < 6.25.\]
So, what is the desired $x$ tolerance?  Remember, we want to find a symmetric interval of $x$ values, namely
$4 - \delta < x < 4 + \delta$.  The lower bound of $2.25$ is $1.75$ units from $4$; the upper bound of $6.25$ is $2.25$ units from $4$. We need the smaller of these two distances; we must have $\delta = 1.75$. See Figure~\ref{fig:1-4_Eg1}.

Now read it in the correct way:  For the $y$ tolerance $\epsilon =0.5$, we have found an $x$ tolerance, $\delta = 1.75$, so that whenever $x$ is within $\delta$ units of $4$, then $y$ is within $\epsilon$ units of $2$.  That's what we were trying to find.
  
Let's try another value of $\epsilon$.  What if the $y$ tolerance is $0.01$, or $\epsilon =0.01$?  How close to $4$ does $x$ have to be in order for $y$ to be within $0.01$ units of $2$ (or $1.99 < y < 2.01$)?  Again, we just square these values to get $1.99^2 < x < 2.01^2$, or 
\[ 3.9601 < x < 4.0401.\]  
So, what is the desired $x$ tolerance?  In this case we must have $\delta = 0.0399$.  Note that in some sense, it looks like there are two tolerances (below $4$ of $0.0399$ units and above $4$ of $0.0401$ units).  However, we couldn't use the larger value of $0.0401$ for $\delta$ since then the interval for $x$ would be  $3.9599 < x < 4.0401$ resulting in $y$ values of $1.98995 < y < 2.01$ (which contains values NOT within $0.01$ units of $2$).

What we have so far: if $\epsilon =0.5$, then $\delta = 1.75$ and if $\epsilon =0.01$, then $\delta = 0.0399$. A pattern is not easy to see, so we switch to general $\epsilon$ and $\delta$ and do the calculations symbolically.  We start by assuming $y=\sqrt{x}$ is within $\epsilon$ units of $2$:
\begin{eqnarray*}
|y - 2| < \epsilon &\\
-\epsilon < y - 2 < \epsilon& \qquad \textrm{(Absolute value)}\\
-\epsilon < \sqrt{x} - 2 < \epsilon  &\qquad (y=\sqrt{x})\\
2 - \epsilon < \sqrt{x} < 2+ \epsilon &\qquad \textrm{ (Add 2)}\\
(2 - \epsilon)^2 < x < (2+ \epsilon) ^2 &\qquad \textrm{ (Square all)}\\
4 - 4\epsilon + \epsilon^2 < x < 4 + 4\epsilon + \epsilon^2 &\qquad \textrm{ (Expand)}\\
4 - (4\epsilon - \epsilon^2) < x < 4 + (4\epsilon + \epsilon^2) &\qquad \textrm{ (Rewrite)}
\end{eqnarray*}

Since we want this last interval to describe an $x$ tolerance around $4$, we have that either $\delta = 4\epsilon + \epsilon^2$ or $\delta = 4\epsilon - \epsilon^2$. However, as we saw in the case when $\epsilon = 0.01$, we want the smaller of the two values for $\delta$.  So, to conclude this part, we set
$\delta$ equal to the minimum of these two values, or $\delta = \min\{4\epsilon + \epsilon^2, 4\epsilon - \epsilon^2\}$.  Since $\epsilon > 0$, the minimum will occur when $\delta = 4\epsilon - \epsilon^2$.  That's the formula!  

We can check this for our previous values.  If $\epsilon=0.5$, the formula gives
$\delta = 4(0.5) - (0.5)^2 = 1.75$ and when $\epsilon=0.01$, the formula gives $\delta = 4(0.01) - (0.01)^2 = 0.399$.

So given any $\epsilon >0$, we can set $\delta = 4\epsilon - \epsilon^2$ and the limit definition is satisfied.  We have shown formally (and finally!) that $\displaystyle \lim_{x\rightarrow 4} \sqrt{x} = 2 $.
\end{example}

%FOOTNOTE $**$: Actually, it is a pain, but this won't work if $\epsilon \ge 4$.  This shouldn't really occur since $\epsilon$ is supposed to be small, but it could happen.  In the cases where $\epsilon \ge 4$, just take $\delta = 1$ and you'll be fine. %EXAMPLE
 
If you are thinking this process is long, you would be right.  The previous example is also a bit unsatisfying in that $\sqrt{4}=2$; why work so hard to prove something so obvious? Many $\epsilon-\delta$ proofs are long and difficult to do. In this section, we will focus on examples where the answer is, frankly, obvious, because the non--obvious examples are even harder.  That is why theorems about limits are so useful! After doing a few more $\epsilon$--$\delta$ proofs, you will really appreciate the analytical "short cuts" we previously discussed. 

\begin{example} \label{Ex:1.4.Eg2} 
Show that $\ds \lim_{x \to 2} x^2 = 4$.

\solution Let's do this example symbolically from the start.  Let $\epsilon > 0$ be given; we want $|y-4| < \epsilon$, i.e.,  $|x^2-4| < \epsilon$.  How do we find $\delta$ such that when $|x-2| < \delta$, we are guaranteed that $|x^2-4|<\epsilon$?% for some $\delta$ (in terms of $\epsilon$)?

This is a bit trickier than the previous example, but let's start by noticing that 
$|x^2-4| = |x-2|\cdot|x+2|$.  Consider:
\begin{equation} |x^2-4| < \epsilon \longrightarrow |x-2|\cdot|x+2| < \epsilon \longrightarrow |x-2| < \frac{\epsilon}{|x+2|}.
\label{eq:limit1}
\end{equation} 
Could we not set $\ds \delta = \frac{\epsilon}{|x+2|}$?  

We are close to an answer, but the catch is that $\delta$ must be a \textit{constant} value (so it can't contain $x$).  There is a way to work around this, but we do have to make an assumption.  Remember that $\epsilon$ is supposed to be a small number, which implies that $\delta$ will also be a small value.  In particular, we can (probably) assume that $\delta < 1$.  If this is true, then $|x-2| < \delta$ would imply that $|x-2| < 1$, giving $1 < x < 3$.  

Now, back to the fraction $\ds \frac{\epsilon}{|x+2|}$.  If $1<x<3$, then $3<x+2<5$.  Taking reciprocals, we have $\ds \frac{1}{5}<\frac{1}{|x+2|}<\frac{1}{3}$ so that, in particular, 
\begin{equation} \frac{\epsilon}{5}<\frac{\epsilon}{|x+2|}.
\label{eq:limit2}
\end{equation}  
This suggests that we set 
$\ds \delta = \frac{\epsilon}{5}$. To see why, let's go back to the equations:

\begin{eqnarray*}
|x - 2| &<& \delta \\
|x - 2| &<& \frac{\epsilon}{5} \\%\qquad \text{\small(Our choice of $\delta$)}\\
|x - 2|\cdot|x + 2| &<& |x + 2|\cdot\frac{\epsilon}{5} \\%\qquad \text{\small(Multiply by $|x+2|$)}\\
|x^2 - 4|&<& |x + 2|\cdot\frac{\epsilon}{5} \\%\qquad \text{\small(Combine left side)}\\
|x^2 - 4|&<& |x + 2|\cdot\frac{\epsilon}{5} <|x + 2|\cdot\frac{\epsilon}{|x+2|}=\epsilon  %\qquad \text{\small(Using (\ref{eq:limit2}) as long as $\delta <1$)}
\end{eqnarray*}

We have arrived at $|x^2 - 4|<\epsilon$ as desired.  Note again, in order to make this happen we needed $\delta$ to first be less than 1.  That is a safe assumption; we want $\epsilon$ to be arbitrarily small, forcing $\delta$ to also be small. 

We have also picked $\delta$ to be smaller than ``necessary.'' We could get by with a slightly larger $\delta$, as shown in Figure~\ref{fig:1-4_Eg2}. The dashed, red lines show the boundaries defined by our choice of $\epsilon$. The gray, dashed lines show the boundaries defined by setting $\delta = \epsilon/5$. Note how these gray lines are within the red lines. That is perfectly fine; by choosing $x$ within the gray lines we are guaranteed that $f(x)$ will be within $\epsilon$ of $4$.%If the value we eventually used for $\delta$, namely $\epsilon/5$, is not less than 1, this proof won't work.  For the final fix, we instead set $\delta$ to be the minimum of 1 and $\epsilon/5$. This way all calculations above work.  

In summary, given $\epsilon > 0$, set $\delta=\epsilon/5$.  Then $|x - 2| < \delta$ implies 
$|x^2 - 4|< \epsilon$ (i.e. $|y - 4|< \epsilon$) as desired.  We have shown that $\ds \lim_{x\rightarrow 2} x^2 = 4 $. Figure~\ref{fig:1-4_Eg2} gives a visualization of this; by restricting $x$ to values within $\delta = \epsilon/5$ of $2$, we see that $f(x)$ is within $\epsilon$ of $4$.
\end{example}

\begin{marginfigure}[-8cm]
%\margingraphics{figures/figLimitProof2a}
\margingraphics{figs/1/1-4_Eg2.pdf}
\caption{Choosing $\delta = \epsilon/5$ in Example \ref{Ex:1.4.Eg2}.}\label{fig:1-4_Eg2}
\end{marginfigure} %EXAMPLE

\begin{example} \label{Ex:1.4.Eg3}
Show that $\ds \lim_{x \to 0} e^x = 1 $.

\solution Symbolically, we want to take the equation $|e^x - 1| < \epsilon$ and unravel it to the form $|x-0| < \delta$.  Let's look at some calculations:
\begin{eqnarray*}
|e^x - 1| < \epsilon&\\
-\epsilon < e^x - 1 < \epsilon& \qquad \textrm{(Definition of absolute value)}\\
1-\epsilon < e^x < 1+\epsilon & \qquad \textrm{(Add 1)}\\
\ln(1-\epsilon) < x < \ln(1+\epsilon) & \qquad \textrm{(Take natural logs)}
\end{eqnarray*}
Making the safe assumption that $\epsilon<1$ ensures the last inequality is valid (i.e., so that $\ln (1-\epsilon)$ is defined). Recall $\ln(1)= 0$ and $\ln(x)<0$ when $0<x<1$. So $\ln (1-\epsilon) <0$, hence we consider its absolute value. We can then set $\delta$ to be the minimum of $|\ln(1-\epsilon)|$ and $\ln(1+\epsilon)$; i.e.,  
%  Well, there is a catch.  The value of $\epsilon$ is supposed to be small, but if it happens that $\epsilon \ge 1$, then $\ln(1-\epsilon)$ would be undefined!  The way to work around this is to simply define a new epsilon that is guaranteed to be smaller than the original epsilon \textit{and} less than 1 (let's say less than 1/2 just to be on the safe side).  Let's call this new value $\epsilon_1$ and define it to be $\epsilon_1 = \min\{\epsilon, 1/2\}$.  Then we can use the calculations above to define 
\[ \delta = \min\{|\ln(1-\epsilon)|, \ln(1+\epsilon)\}. \]


Now, we work through the actual the proof:

\begin{eqnarray*}
|x - 0|<\delta\\
-\delta < x < \delta& \qquad \textrm{(Definition of absolute value)}\\
\ln(1-\epsilon) < x < \ln(1+\epsilon) & \qquad \textrm{(By our choice of}\; \delta)\\
1-\epsilon < e^x < 1+\epsilon & \qquad \textrm{(Exponentiate)}\\
-\epsilon < e^x - 1 < \epsilon & \qquad \textrm{(Subtract 1)}\\
%-\epsilon < e^x - 1 < \epsilon & \qquad \textrm{(Since}\; \epsilon_1 \le \epsilon)\\
\end{eqnarray*}

In summary, given $\epsilon > 0$, let $\delta = \min\{|\ln(1-\epsilon |, \ln(1+\epsilon)\}$. Then $|x - 0| < \delta$ implies $|e^x - 1|< \epsilon$ as desired.  We have shown that $\displaystyle \lim_{x\rightarrow 0} e^x = 1 $.
\end{example} %EXAMPLE

We note that we could actually show that $\ds \lim_{x \to a} e^x = e^a$ for any constant $a$.  We do this by factoring out $e^a$ from both sides, leaving us to show $\ds \lim_{x \to a} e^{x-a} = 1$ instead.  By using the substitution $y=x-a$, this reduces to showing $\lim_{y \to 0} e^y = 1 $ which we just did in the last example.  As an added benefit, this shows that in fact the function $f(x)=e^x$ is \textit{continuous} at all values of $x$.

%-----------------------------------------------------
% SUBSECTION LIMITS INVOLVING INFINITY
%-----------------------------------------------------
\subsection*{Limits involving infinity}

In the graph of $\ds f(x)= \frac{1}{x^2}$ shown in Figure~\ref{fig:oneoverxsquaredb}, we see that near $0$, the function explodes, getting larger and larger, heading off to positive infinity.

\begin{marginfigure}[1cm]
\margingraphics{figures/figoneoverxsquared} %APEX limits with infinity
\caption{Graph of $f(x) = 1/x^2$. }\label{fig:oneoverxsquaredb}
\end{marginfigure}

Recall from Section~\ref{S:1.2.Infinity} that in a case like this, we write
\[ \lim_{x \to 0} \frac{1}{x^2}=\infty. \]
We can make this notion precise as follows:

\definition{Limit of Infinity, $\infty$} %DEFINITON
{We say $\ds \lim_{x\rightarrow c} f(x)=\infty$ if for every $M>0$ there exists $\delta>0$ such that if $0<|x-c|<\delta$ then $f(x)\geq M$. %\index{limit!of infinity}  
}

This is just like the $\epsilon$--$\delta$ definition of a limit above.  In that definition, given any (small) value $\epsilon$, if we let $x$ get close enough to $c$ (within $\delta$ units of $c$) then $f(x)$ is guaranteed to be within $\epsilon$ of $f(c)$.  Here, given any (large) value $M$, if we let $x$ get close enough to $c$ (within $\delta$ units of $c$), then $f(x)$ will be at least as large as $M$.  In other words, if we get close enough to $c$, then we can make $f(x)$ as large as we want.  We can define limits equal to $-\infty$ in a similar way.

Once again note that by saying $\ds \lim_{x\to c}f(x) = \infty$ we are implicitly stating that \textit{the} limit of $f(x)$, as $x$ approaches $c$, \textit{does not exist.} A limit only exists when $f(x)$ approaches an actual numeric value. We use the concept of limits that approach infinity because they are helpful and descriptive. 

\begin{marginfigure}[8cm]
\margingraphics{figures/fignolimit2} %APEX limit infinity example
\caption{Observing infinite limit as $x\to 1$ in Example \ref{Ex:1.4.Eg4}.}\label{fig:1.4.Eg4}
\end{marginfigure}

\begin{example} \label{Ex:1.4.Eg4}
Find $\ds \lim_{x \to 1}\frac1{(x-1)^2}$ as shown in Figure~\ref{fig:1.4.Eg4}

\solution In Example~\ref{Ex:1.2.Eg1} of Section~\ref{S:1.2.Infinity}, by inspecting values of $x$ close to $1$ we concluded that this limit does not exist.  That is, it cannot equal any real number.  But the limit could be infinite.  And in fact, we see that the function does appear to be growing larger and larger, as $f(.99)=10^4$, $f(.999)=10^6$, $f(.9999)=10^8$.  A similar thing happens on the other side of $1$.  In general, let a ``large'' value $M$ be given. Let $\delta=1/\sqrt{M}$. If $x$ is within $\delta$ of $1$, i.e., if $|x-1|<1/\sqrt{M}$, then:
\begin{align*}
|x-1| &< \frac{1}{\sqrt{M}} \\
(x-1)^2 &< \frac{1}{M}\\
\frac{1}{(x-1)^2} &> M,
\end{align*}
which is what we wanted to show.  So we may say $\ds \lim_{x \to 1}1/{(x-1)^2}=\infty$.
\end{example} %EXAMPLE

%-----------------------------------------------------
% SUBSECTION LIMITS INVOLVING INFINITY
%-----------------------------------------------------
\subsection*{Limits at infinity}

Let's again consider $f(x) =\frac{1}{x^2}$, as shown in Figure \ref{fig:oneoverxsquaredb}. Note that as $x$ gets very large, $f(x)$ gets very, very close to zero. We represent this concept with notation such as
\[ \lim_{x \to \infty} \frac1{x^2}=0, \]
and give the following precise definition.

%\begin{marginfigure}[1cm]
%\margingraphics{figures/figoneoverxsquared} %APEX limits with infinity
%\caption{Graph of $f(x) = 1/x^2$. }\label{fig:oneoverxsquared}
%\end{marginfigure}

\definition{Limits at Infinity} %\label{precise.limit.infinity}
{\begin{enumerate}
\item We say $\ds\lim_{x\rightarrow\infty} f(x)=L$ if for every $\epsilon>0$ there exists $N>0$ such that if $x\geq N$, then $|f(x)-L|<\epsilon$.\index{limit!at infinity} 

\item We say $\ds\lim_{x\rightarrow-\infty} f(x)=L$ if for every $\epsilon>0$ there exists $N<0$ such that if $x\leq N$, then $|f(x)-L|<\epsilon$.
\end{enumerate}
}

This says that $f$ is sufficiently close to $L$ whenever $x$ is sufficiently large. In other words, if $x$ is greater than some number $N$ then $f(x)$ is between $L-\epsilon$ and $L+\epsilon$. If a smaller $\epsilon$ is chosen, a larger value of $N$ may be required. 

\begin{marginfigure}[-3cm]
\margingraphics{figures/2_8_Infty.eps} %graph of 1/x 
\caption{Observing limit as $x \to \infty$ in Example \ref{Ex:1.4.Eg5}.}\label{fig:1.4.Eg5}
\end{marginfigure}

\begin{margintable} %not sure if scalebox is needed
\begin{center}
\scalebox{1.25}{
	\begin{tabular}[b]{r|l} 
	$\epsilon$ & $N$ \\ 
	\hline $1$ & $1$ \\ 
	$0.2$ & $5$ \\ 
	$0.1$ & $10$ \\ 
	$0.05$ & $20$ \\ 
	$0.01$ & $100$ \\ 
	\end{tabular}
} % end scalebox 
\end{center}
\caption{Values of $\epsilon$ and corresponding values of $N$.} \label{T:1-4_Eg5}
\end{margintable}

\begin{example} \label{Ex:1.4.Eg5}
Show $\ds \lim_{x \to \infty}\frac{1}{x} = 0$ as shown in Figure~\ref{fig:1.4.Eg5}.

\solution By the precise definition of limits at infinity, given $\epsilon>0$, we need to find $N$ such that if $x>N$ then $\lvert \frac{1}{x} - 0 \rvert < \epsilon$.  Since $x$ is approcahing $\infty$, we may assume $x>0$. Then $\frac{1}{x}<\epsilon$ and thus $x>\frac{1}{\epsilon}$. Let $N=\frac{1}{\epsilon}$. If $x>N$ then $\lvert \frac{1}{x} - 0 \rvert < \epsilon$, which is what we wanted to show.  So we may say $\ds \lim_{x \to \infty} \frac{1}{x}=0$.
If we choose smaller values for $\epsilon$, we will need bigger values for $N$ but the inequality will still hold. Table~\ref{T:1-4_Eg5} shows different values of $\epsilon$ and the corresponding values of $N$. 


\end{example} %EXAMPLE

\vspace{2cm}

%-------------
% SUMMARY
%-------------
\begin{summary}
\item We now have definitions of limit that do not include arbitrary measures such as ``approaches'' or ``sufficiently close to''.

 \item The precise definitions given in this section are used to prove the Limit Laws we gave in Section~\ref{S:1.1.Limits}.
\end{summary}

\clearpage

%--------------
% EXERCISES
%--------------
\begin{adjustwidth*}{}{-2.25in}
\textbf{{\large Exercises}}
\setlength{\columnsep}{25pt}
\begin{multicols*}{2}
\noindent Terms and Concepts \small

\begin{enumerate}[1)]
\item What is wrong with the following ``definition'' of a limit?
	\begin{quote}
``The limit of $f(x)$, as $x$ approaches $a$, is $K$'' means that given any $\delta>0$ there exists $\epsilon>0$ such that whenever $|f(x)-K|< \epsilon$, we have $|x-a|<\delta$.
	\end{quote}
\item Which is given first in establishing a limit, the $x$--tolerance or the $y$--tolerance?
\item T/F: $\delta$ must always be positive.
\item T/F: $\epsilon$ must always be positive.
\end{enumerate} 

\noindent {\normalsize Problems\\} \small

\noindent In exercises 5--11, prove the given limit using an $\epsilon - \delta$ proof.

\begin{enumerate}[1),resume]
\item {$\displaystyle \lim_{x\to 2} 5 = 5$}
\item {$\displaystyle \lim_{x\to 5} 3-x = -2$}
\item {$\displaystyle \lim_{x\to 3} x^2-3 = 6$}
\item {$\displaystyle \lim_{x\to 4} x^2+x-5 = 15$}
\item {$\displaystyle \lim_{x\to 2} x^3-1 = 7$}
\item {$\displaystyle \lim_{x\to 0} e^{2x}-1 = 0$}
\item {$\displaystyle \lim_{x\to 0} \sin x= 0$ (Hint: use the fact that $|\sin x| \leq |x|$, with equality only when $x=0$.)}
\end{enumerate}

\vspace{.5cm}

%------------------------------------------
% END OF EXERCISES ON FIRST PAGE
%------------------------------------------
\end{multicols*}
\end{adjustwidth*}
\afterexercises 

\cleardoublepage
