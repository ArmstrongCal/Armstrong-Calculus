\section{Trigonometric Integrals} \label{S:5.2.TrigInt}

\begin{goals}
\item How do we integrate functions that are products of powers of the trigonometric functions?
\item How do we integrate functions that are products of trigonometric functions whose angles are constant multiples of a single variable?
\end{goals}

%-----------------------------------
%SUBSECTION: INTRODUCTION
%-----------------------------------

\subsection*{Introduction}

Functions involving trigonometric functions are useful for describing periodic behavior.  Learning techniques to integrate such functions helps solve problems involving periodic behavior and build general problem solving skills.
 This section describes several techniques for finding antiderivatives of certain combinations of trigonometric functions.

In Preview Activity~\ref{PA:5.2} we refresh our use of substitution and integration by parts to investigate some indefinite integrals that involve products of trigonometric functions.

\begin{pa} \label{PA:5.2}
In Section~\ref{S:4.6.Substitution}, we used subsitution to integrate composition functions.  In particular, recall that if $f$ and $g$ are differentiable functions of $x$, then
If $u = g(x)$, then $du = g'(x)dx$ and 
\[ \int F'(g(x))g'(x)\ dx = \int F'(u)\ du = F(u)+C = F(g(x))+C. \]
For each of the following integrals, use substitution to evaluate each one. 
	\be
		\item $\int \sin(x)\cos(x) \ dx$
		\item $\int \sin^2(x)\cos(x) \ dx$
		\item $\int \tan(x)\sec^2(x) \ dx$
		\item $\int \tan^3(x)\sec(x) \ dx$
		\item $\int \cos^2(x) \ dx$ Hint: Use the trigonometric identitiy $ \cos^2(x) = \frac{1}{2} + \frac{\cos(2x)}{2}$
	\ee
\end{pa} 
\afterpa %PREVIEW 

%-----------------------------------------------------------------
% SUBSECTION INTEGRALS OF THE FORM $\ds \int \sin^m x\cos^n x\ dx$
%-----------------------------------------------------------------
\subsection*{Integrals of the form $\ds \int \sin^m (x) \cos^n (x)\ dx$} 

In Preview Activity~\ref{PA:5.2}, we integrate $\int \sin(x) \cos(x) \ dx$; the integration is not difficult, and one could easily evaluate the indefinite integral by letting $u=\sin(x) $ or by letting $u = \cos(x)$. This integral is easy since the power of both sine and cosine is $1$. 

What if the power of both sine and cosine are not $1$? Preview Activity~\ref{PA:5.2} gives an idea on how to use substitution to evaluate such integrals. We generalize the integral $\int \sin(x) \cos(x) \ dx$ and consider integrals of the form $\int \sin^m(x) \cos^n(x)\ dx$, where $m$ and $n$ are nonnegative integers. Many times we use the Pythagorean identity $\cos^2 (x) + \sin^2(x) = 1$ to convert high powers of one trigonometric function into the other while leaving a single sine or cosine term in the integrand. This technique will allow us to use substitution to evaluate such integrals, and we summarize it in the following.

\concept{Integrals Involving Powers of Sine and Cosine} %CONCEPT
{Consider $\ds \int \sin^m(x) \cos^n(x) \ dx$, where $m$ and $n$ are nonnegative integers.\index{integration!of trig. powers}
\begin{enumerate}[1)]
\item If $m$ is odd, then $m=2k+1$ for some integer $k$. Rewrite \\
$$ \sin^m(x) = \sin^{2k+1}(x) = \sin^{2k}(x)\sin(x) $$
$$= (\sin^2(x))^k\sin(x) = (1-\cos^2(x))^k\sin(x).$$ 
Then 
$$\int \sin^m(x)\cos^n(x)\ dx $$
$$= \int (1-\cos^2(x))^k\sin(x)\cos^n(x)\ dx = -\int (1-u^2)^ku^n\ du,$$ 
where $u = \cos(x)$ and $du = -\sin(x)\ dx$. 

\item If $n$ is odd, then using substitutions similar to that outlined above we have
$$ \int \sin^m(x)\cos^n(x)\ dx = \int u^m(1-u^2)^k\ du,$$ %
where $u = \sin(x)$ and $du = \cos(x)\ dx$.

\item If both $m$ and $n$ are even, use the power--reducing identities %\small
$$\cos^2(x) = \frac{1+\cos (2x)}{2} \quad \text{and}\quad \sin^2(x) = \frac{1-\cos(2x)}2$$ %\normalsize
to reduce the degree of the integrand. Expand the result and apply the principles again.
\end{enumerate}
} %END CONCEPT

To demonstrate, we consider the following examples.

\begin{example} \label{eg:5.2.1} % EXAMPLE
Evaluate $\ds\int\sin^5(x)\cos^8(x)\ dx$.

\solution The power of the sine term is odd, so we rewrite $\sin^5(x)$ as 
$$\sin^5(x) = \sin^4(x)\sin(x) = (\sin^2(x))^2\sin(x) = (1-\cos^2(x))^2\sin(x).$$
Our integral is now $\ds \int (1-\cos^2(x))^2\cos^8(x)\sin(x)\ dx$. Let $u = \cos(x)$, hence $du = -\sin(x)\ dx$. Making the substitution and expanding the integrand gives
\begin{align*}
\int (1-\cos^2(x))^2\cos^8(x)\sin(x)\ dx &= -\int (1-u^2)^2u^8\ du \\
& = -\int \big(1-2u^2+u^4\big)u^8\ du \\
&= -\int \big(u^8-2u^{10}+u^{12}\big)\ du.
\end{align*}
This final integral is not difficult to evaluate, giving 
\begin{align*} -\int \big(u^8-2u^{10}+u^{12}\big)\ du &= -\frac19u^9 + \frac2{11}u^{11} - \frac1{13}u^{13} + C \\
										&=-\frac19\cos^9(x) + \frac2{11}\cos^{11}(x) - \frac1{13}\cos^{13}(x) + C.
\end{align*}
\end{example} % EXAMPLE

\marginnote[-4cm]{\textbf{Technology Note:} The work we are doing here can be a bit tedious, but the skills it develops (problem solving, algebraic manipulation, etc.) are important. Nowadays problems of this sort are often solved using a computer algebra system. The powerful program \textit{WolframAlpha}\textsuperscript{\textregistered} integrates $\int \sin^5(x)\cos^9(x)\ dx$ as 
\begin{align*}
f(x)&=-\frac{45 \cos (2 x)}{16384}-\frac{5 \cos (4 x)}{8192}+\frac{19 \cos (6x)}{49152}\\
&+\frac{\cos (8 x)}{4096}-\frac{\cos (10 x)}{81920}-\frac{\cos (12x)}{24576}-\frac{\cos (14 x)}{114688},
\end{align*}
which clearly has a different form than our answer in Example \ref{eg:5.2.1}, which is
\begin{align*}
g(x)&=\frac16\sin^6 (x)-\frac12\sin^8 (x)+\frac35\sin^{10} (x)\\
&-\frac13\sin^{12} (x)+\frac{1}{14}\sin^{14} (x).
\end{align*}
Figure \ref{F:5.2_Eg2} shows a graph of $f$ and $g$; they are clearly not equal. We leave it to the reader to recognize why both answers are correct.}

\begin{marginfigure}[1cm]
\margingraphics{figures/figtrigint2.pdf} 
\caption{A plot of $f(x)$ and $g(x)$ from Example \ref{eg:5.2.2} and the Technology Note.}
\label{F:5.2_Eg2}
\end{marginfigure}

\begin{example} \label{eg:5.2.2} % EXAMPLE
Evaluate $\ds \int\sin^5(x)\cos^9(x)\ dx$.

\solution The powers of both the sine and cosine terms are odd, therefore we can apply the techniques of substitution to either power. We choose to work with the power of the cosine term since the previous example used the sine term's power.

We rewrite $\cos^9(x)$ as
\begin{align*} \cos^9(x) &= \cos^8(x)\cos(x) \\
				&= (\cos^2(x))^4\cos(x) \\
				&= (1-\sin^2(x))^4\cos(x) \\
				&= (1-4\sin^2(x)+6\sin^4(x)-4\sin^6(x)+\sin^8(x))\cos(x).
\end{align*}

We rewrite the integral $\ds \int\sin^5(x)\cos^9(x)\ dx$ as 
$$ \int\sin^5(x)\big(1-4\sin^2(x)+6\sin^4(x)-4\sin^6(x)+\sin^8(x)\big)\cos(x)\ dx.$$

Now substitute and integrate, using $u = \sin(x) $ and $du = \cos(x)\ dx$.

\noindent
$\ds \int\sin^5(x)\big(1-4\sin^2(x)+6\sin^4(x)-4\sin^6(x)+\sin^8(x)\big)\cos(x)\ dx =$

\noindent
$\ds \int u^5(1-4u^2+6u^4-4u^6+u^8)\ du = \int\big(u^5-4u^7+6u^9-4u^{11}+u^{13}\big)\ du$

\noindent
$ \ds = \frac16u^6-\frac12u^8+\frac35u^{10}-\frac13u^{12}+\frac{1}{14}u^{14}+C $

\noindent
$ \ds = \frac16\sin^6(x)-\frac12\sin^8(x)+\frac35\sin^{10}(x)-\frac13\sin^{12}(x)+\frac{1}{14}\sin^{14}(x)+C.$
\end{example}

 % EXAMPLE

\begin{example} \label{eg:5.2.3} % EXAMPLE
Evaluate $\ds\int\cos^4(x)\sin^2(x)\ dx$.

\solution The powers of sine and cosine are both even, so we employ the power--reducing formulas and algebra as follows.
\begin{align*}
\int \cos^4(x)\sin^2(x)\ dx &= \int\left(\frac{1+\cos(2x)}{2}\right)^2\left(\frac{1-\cos(2x)}2\right)\ dx \\
				&= \int\frac{1+2\cos(2x)+\cos^2(2x)}4\cdot\frac{1-\cos(2x)}2\ dx\\
				&=	\int \frac18\big(1+\cos(2x)-\cos^2(2x)-\cos^3(2x)\big)\ dx
\end{align*}
The $\cos(2x)$ term is easy to integrate. The $\cos^2(2x)$ term is another trigonometric integral with an even power, requiring the power--reducing formula again. The $\cos^3(2x)$ term is a cosine function with an odd power, requiring a substitution as done before. We integrate each in turn below.

$$\int\cos(2x)\ dx = \frac12\sin(2x)+C.$$

$$\int\cos^2(2x)\ dx = \int \frac{1+\cos(4x)}2\ dx = \frac12\left(x+\frac14\sin(4x)\right)+C.$$

Finally, we rewrite $\cos^3(2x)$ as $$\cos^3(2x) = \cos^2(2x)\cos(2x) = \big(1-\sin^2(2x)\big)\cos(2x).$$
Letting $u=\sin(2x)$, we have $du = 2\cos(2x)\ dx$, hence
\begin{align*}
\int \cos^3(2x)\ dx &= \int\big(1-\sin^2(2x)\big)\cos(2x)\ dx\\
							&= \int \frac12(1-u^2)\ du\\
							&= \frac12\Big(u-\frac13u^3\Big)+C\\
							&=	\frac12\Big(\sin(2x)-\frac13\sin^3(2x)\Big)+C
\end{align*}

Putting all the pieces together, we have

\noindent
$\ds \int \cos^4x\sin^2x\ dx =\int \frac18\big(1+\cos(2x)-\cos^2(2x)-\cos^3(2x)\big)\ dx$

\noindent
$\ds = \frac18\Big[x+\frac12\sin(2x)-\frac12\big(x+\frac14\sin(4x)\big)-\frac12\Big(\sin(2x)-\frac13\sin^3(2x)\Big)\Big]+C $

\noindent 
$\ds =\frac18\Big[\frac12x-\frac18\sin(4x)+\frac16\sin^3(2x)\Big]+C$
\end{example} % EXAMPLE

The process above was a bit long and tedious, but being able to work a problem such as this from start to finish is important.

\begin{activity} \label{A:5.2.1}  Evaluate each of the following indefinite integrals.  

\bmtwo
\ba
	\item $\int \cos^3(x)\sin^2(x) \, dx$
	\item $\int \sin^5(x)\cos^2(x) \, dx$
	\item $\int \sin^3(x)\cos^3(x) \, dx $
	\item $\int \sin^2(x) \, dx $
\ea
\emtwo
\end{activity}
\begin{smallhint}
\ba
	\item Small hints for each of the prompts above.
\ea
\end{smallhint}
\begin{bighint}
\ba
	\item Big hints for each of the prompts above.
\ea
\end{bighint}
\begin{activitySolution}
\ba
	\item Solutions for each of the prompts above.
\ea
\end{activitySolution}
\aftera %ACTIVITY  

%------------------------------------------------
% SUBSECTION Integrals of the form $\ds \int\sin(mx)\sin(nx)\ dx,$ $\ds\int \cos(mx)\cos(nx)\ dx$, and $\ds\int \sin(mx)\cos(nx)\ dx$
%------------------------------------------------
\subsection*{Integrals of the form $\ds \int\sin(mx)\sin(nx)\ dx,$ $\ds\int \cos(mx)\cos(nx)\ dx$, and $\ds\int \sin(mx)\cos(nx)\ dx$}

Functions that contain products of sines and cosines of differing periods are important in many applications including the analysis of sound waves. Integrals of the form \small
$$\int\sin(mx)\sin(nx)\ dx,\quad \int \cos(mx)\cos(nx)\ dx \quad \text{and}\quad\int \sin(mx)\cos(nx)\ dx$$ \normalsize
are best approached by first applying the Product to Sum Formulas, namely
\begin{align*}
\sin(mx)\sin(nx) &= \frac12\Big[\cos\big((m-n)x\big)-\cos\big((m+n)x\big)\Big] \\
\cos(mx)\cos(nx) &= \frac12\Big[\cos\big((m-n)x\big)+\cos\big((m+n)x\big)\Big] \\
\sin(mx)\cos(nx) &=	\frac12\Big[\sin\big((m-n)x\big)+\sin\big((m+n)x\big)\Big]
\end{align*}

\begin{example} \label{eg:5.2.6} % EXAMPLE
Evaluate $\ds\int\sin(5x)\cos(2x)\ dx$.

\solution The application of the formula and subsequent integration are straightforward:
\begin{align*}
\int\sin(5x)\cos(2x)\ dx &= \int \frac12\Big[\sin(3x)+\sin(7x)\Big]\ dx \\
&= -\frac16\cos(3x) - \frac1{14}\cos(7x) + C
\end{align*}
\end{example} % EXAMPLE

%------------------------------------------------------
% SUBSECTION Integrals of the form $\ds\int\tan^mx\sec^nx\ dx$
%------------------------------------------------------
\subsection*{Integrals of the form $\ds\int\tan^m(x)\sec^n(x)\ dx$} 

When evaluating integrals of the form $\int \sin^m(x)\cos^n(x)\ dx$, the Pythagorean Theorem allowed us to convert even powers of sine into even powers of cosine, and vice--versa. If, for instance, the power of sine was odd, we pulled out one $\sin(x)$ and converted the remaining even power of $\sin(x)$ into a function using powers of $\cos(x)$, leading to an easy substitution.

The same basic strategy applies to integrals of the form \newline $\int \tan^m(x)\sec^n(x)\ dx$, albeit a bit more nuanced. The following three facts will prove useful:
\begin{itemize}
\item $\frac{d}{dx}(\tan(x)) = \sec^2(x)$, 
\item $\frac{d}{dx}(\sec(x)) = \sec(x)\tan(x)$ , and 
\item	$1+\tan^2(x) = \sec^2(x)$ (the Pythagorean Theorem).
\end{itemize}

If the integrand can be manipulated to separate a $\sec^2(x)$ term with the remaining secant power even, or if a $\sec(x)\tan(x)$ term can be separated with the remaining $\tan(x)$ power even, the Pythagorean Theorem can be employed, leading to a simple substitution. This strategy is outlined in the following:

\concept{Integrals Involving Powers of Tangent and Secant} % CONCEPT
{Consider $\ds\int\tan^m(x)\sec^n(x)\ dx$, where $m$ and $n$ are nonnegative integers.\index{integration!of trig. powers}
\begin{enumerate}[1)]
\item If $n$ is even, then $n=2k$ for some integer $k$. Rewrite $\sec^n(x)$ as 
$$\sec^n(x) = \sec^{2k}(x) = \sec^{2k-2}(x)\sec^2(x)$$
$$ = (1+\tan^2(x))^{k-1}\sec^2(x).$$
Then
$$\int\tan^m(x)\sec^n(x)\ dx=$$
$$\int\tan^m(x)(1+\tan^2(x))^{k-1}\sec^2(x)\ dx =$$
$$ \int u^m(1+u^2)^{k-1}\ du,$$
where $u = \tan(x)$ and $du = \sec^2(x)\ dx$.

\item If $m$ is odd, then $m=2k+1$ for some integer $k$. Rewrite $\tan^m(x)\sec^n(x)$ as
$$\tan^m(x)\sec^n(x) = \tan^{2k+1}(x)\sec^n(x) = $$
$$\tan^{2k}(x)\sec^{n-1}(x)\sec(x)\tan(x) = $$
$$(\sec^2(x)-1)^k\sec^{n-1}(x)\sec(x)\tan(x).$$
Then
$$\int\tan^m(x)\sec^n(x)\ dx=$$
$$\int(\sec^2(x)-1)^k\sec^{n-1}(x)\sec(x)\tan(x)\ dx = $$
$$\int(u^2-1)^ku^{n-1}\ du,$$
where $u = \sec(x)$ and $du = \sec(x)\tan(x)\ dx$.

\item If $n$ is odd and $m$ is even, then $m=2k$ for some integer $k$. Convert $\tan^m(x) $ to $(\sec^2(x)-1)^k$. Expand the new integrand and use Integration By Parts, with $dv = \sec^2(x)\ dx$.

\item If $m$ is even and $n=0$, rewrite $\tan^m(x)$ as
$$\tan^m(x) = \tan^{m-2}(x)\tan^2(x) = \tan^{m-2}(x)(\sec^2(x)-1) = $$
$$\tan^{m-2}(x)\sec^2(x)-\tan^{m-2}(x).$$
So
$$\int\tan^m(x)\ dx = $$
$$\underbrace{\int\tan^{m-2}(x)\sec^2(x)\ dx}_{\text{\small apply rule \#1}}\quad - \underbrace{\int\tan^{m-2}(x)\ dx}_{\text{\small apply rule \#4 again}}.$$

\end{enumerate}
} %end concept

The techniques described in items $1$) and $2$) above are relatively straightforward, but the techniques in items $3$) and $4$) can be rather tedious. A few examples will help with these methods.

\begin{example} \label{eg:5.2.4} % EXAMPLE
Evaluate $\ds\int \sec^3(x)\ dx$.

\solution We apply rule $3$) as the power of secant is odd and the power of tangent is even ($0$ is an even number). We use Integration by Parts; the rule suggests letting $dv = \sec^2(x)\ dx$, meaning that $u = \sec(x)$. \\ \footnotesize

\begin{tabular}{llcll}  
$u= \sec(x)$ & $v=\text{?}$ &  & $u= \sec(x)$ & $v=\tan(x)$ \\
 && $\Rightarrow$ && \\
$du= \text{?}$ & $dv=\sec^2(x)\ dx$ & & $du= \sec(x)\tan(x)\ dx$ & $dv=\sec^2(x)\ dx$ \\
\end{tabular}\small

\vspace{.25cm}

Employing Integration by Parts, we have
\begin{align*}
\int \sec^3(x)\ dx  	&=	\int \underbrace{\sec(x)}_u\cdot\underbrace{\sec^2(x)\ dx}_{dv}\\
						&=	\sec(x)\tan(x) - \int \sec(x) \tan^2(x)\ dx. \\
\intertext{This new integral also requires applying rule $3$):}
						&= \sec(x)\tan(x) - \int \sec(x) \big(\sec^2(x)-1\big)\ dx\\
						&=	\sec(x)\tan(x) - \int \sec^3(x)\ dx + \int \sec(x)\ dx \\
						&= \sec(x)\tan(x) -\int \sec^3(x)\ dx + \ln|\sec(x)+\tan(x)| \\
\intertext{In previous applications of Integration by Parts, we have seen where the original integral has reappeared in our work. We resolve this by adding $\int \sec^3(x)\ dx$ to both sides, giving:}
2\int \sec^3(x)\ dx &= \sec(x)\tan(x) + \ln|\sec(x)+\tan(x)| \\
\int \sec^3(x)\ dx &= \frac12\Big(\sec(x)\tan(x) + \ln|\sec(x)+\tan(x)|\Big)+C
\end{align*}	
\end{example} % EXAMPLE

\begin{example} \label{eg:5.2.5} % EXAMPLE
Evaluate $\ds\int\tan^6(x)\ dx$.

\solution
We employ rule $4$). 
\begin{align*}
\int \tan^6(x)\ dx &= \int \tan^4(x)\tan^2(x)\ dx \\
			&= \int\tan^4(x)\big(\sec^2(x)-1\big)\ dx\\
			&= \int\tan^4(x)\sec^2(x)\ dx - \int\tan^4(x)\ dx \\
\intertext{Integrate the first integral with substitution, $u=\tan(x)$; integrate the second by employing rule $4$) again.}
			&=	\frac15\tan^5(x)-\int\tan^2(x)\tan^2(x)\ dx\\
			&=	\frac15\tan^5(x)-\int\tan^2(x)\big(\sec^2(x)-1\big)\ dx \\
			&= \frac15\tan^5(x) -\int\tan^2(x)\sec^2(x)\ dx + \int\tan^2(x)\ dx\\
\intertext{Again, use substitution for the first integral and rule $4$) for the second.}
			&= \frac15\tan^5(x)-\frac13\tan^3(x)+\int\big(\sec^2(x)-1\big)\ dx \\
			&= \frac15\tan^5(x)-\frac13\tan^3(x)+\tan(x) - x+C
\end{align*}

\end{example} % EXAMPLE  

\input{activities/5.2.Act2} % ACTIVITY 

%------------------------------------------------------
% SUBSECTION Integrals of the form $\ds\int\csc^mx\cot^nx\ dx$
%------------------------------------------------------
\subsection*{Integrals of the form $\ds\int\csc^m(x)\cot^n(x)\ dx$} 

We can use similar methods to antidifferentiate integrals involving powers of cosecant and cotangent.  We leave the development of the process and practice to the reader.

%----------------------------------------------------------
% SUMMARY
%----------------------------------------------------------
\begin{summary}
\item The method of Trigonometric Integration enables functions of the form $\ds \sin^m (x) \cos^n (x)$, $\ds \sin(mx)\sin(nx)$, $\ds \cos(mx) \cos(nx)$, and $\ds (mx)\cos(nx)$ to be antidifferentiated.
\item Also using this method, we can antidifferentiate functions of the form $\ds \tan^m(x)\sec^n(x)$ and $\csc^m(x)\cot^n(x)$.
\end{summary}

\clearpage

%--------------
% EXERCISES
%--------------
\begin{adjustwidth*}{}{-2.25in}
\textbf{{\large Exercises}}
\setlength{\columnsep}{25pt}
\begin{multicols*}{2}
\noindent Terms and Concepts \small
\begin{enumerate}[1)]
\item T/F: $\ds \int \sin^2x\cos^2x \ dx$ cannot be evaluated using the techniques described in this section since both powers of $\sin x$ and $\cos x$ are even.
\item T/F: $\ds \int \sin^3x\cos^3x \ dx$ cannot be evaluated using the techniques described in this section since both powers of $\sin x$ and $\cos x$ are odd.
\item T/F: This section addresses how to evaluate indefinite integrals such as $\ds \int \sin^5x\tan^3x\ dx.$
\end{enumerate} 

\noindent {\normalsize Problems} \small

\noindent{\bf In exercises 4--26, evaluate the given indefinte integral.}

\begin{enumerate}[1),resume]
\item $\ds \int \sin x\cos^4x\ dx$
\item $\ds \int \sin^3 x\cos x\ dx$
\item $\ds \int \sin^3 x\cos^2 x\ dx$
\item $\ds \int \sin^3 x\cos^3 x\ dx$
\item $\ds \int \sin^6 x\cos^5 x\ dx$
\item $\ds \int \sin^2 x\cos^7 x\ dx$
\item $\ds \int \sin^2 x\cos^2 x\ dx$
\item $\ds \int \sin(5x)\cos(3x)\ dx$
\item $\ds \int \sin(x)\cos(2x)\ dx$
\item $\ds \int \sin(3x)\sin(7x)\ dx$
\item $\ds \int \sin(\pi x)\sin(2\pi x)\ dx$
\item $\ds \int \cos(x)\cos(2x)\ dx$
\item $\ds \int \cos\left(\frac{\pi}{2} x\right)\cos(\pi x)\ dx$
\item $\ds \int \tan^4x\sec^2x\ dx$
\item $\ds \int \tan^2x\sec^4x\ dx$
\item $\ds \int \tan^3x\sec^4x\ dx$
\item $\ds \int \tan^3x\sec^2x\ dx$
\item $\ds \int \tan^3x\sec^3x\ dx$
\item $\ds \int \tan^5x\sec^5x\ dx$
\item $\ds \int \tan^4x\ dx$
\item $\ds \int \sec^5x\ dx$
\item $\ds \int \tan^2x\sec x\ dx$
\item $\ds \int \tan^2x\sec^3x\ dx$
\end{enumerate}

\noindent{\bf In exercises 27--33, evaluate the definite integral. Note that the corresponding indefinite integrals appear in the previous set.}

\begin{enumerate}[1),resume]
\item $\ds \int_0^{\pi} \sin x\cos^4x\ dx$
\item $\ds \int_{-\pi}^{\pi} \sin^3 x\cos x\ dx$
\item $\ds \int_{-\pi/2}^{\pi/2} \sin^2 x\cos^7 x\ dx$
\item $\ds \int_{0}^{\pi/2} \sin(5 x)\cos(3x)\ dx$
\item $\ds \int_{-\pi/2}^{\pi/2} \cos(x)\cos(2x)\ dx$
\item $\ds \int_{0}^{\pi/4} \tan^4x\sec^2x\ dx$
\item $\ds \int_{-\pi/4}^{\pi/4} \tan^2x\sec^4x\ dx$
\end{enumerate}

%------------------------------------------
% END OF EXERCISES ON FIRST PAGE
%------------------------------------------
\end{multicols*}
\end{adjustwidth*}

%\clearpage
%
%\begin{adjustwidth*}{}{-2.25in}
%\setlength{\columnsep}{25pt}
%\begin{multicols*}{2}\small
%
%\end{enumerate}
%
%%---------------------------------------------
%% END OF EXERCISES ON SECOND PAGE
%%---------------------------------------------
%\end{multicols*}
%\end{adjustwidth*}

\afterexercises 

\cleardoublepage