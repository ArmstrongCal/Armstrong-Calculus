\section{Partial Fractions} \label{S:5.4.PartFrac}

\begin{goals}
\item How does the method of partial fractions enable any rational function to be antidifferentiated?
%\item What role have integral tables historically played in the study of calculus and how can a table be used to evaluate integrals such as $\int \sqrt{a^2 + u^2} \, du$?
%\item What role can a computer algebra system play in the process of finding antiderivatives?
\end{goals}

%--------------------------------------
% SUBSECTION INTRODUCTION
%--------------------------------------
\subsection*{Introduction}

In the preceding sections, we have learned very specific antidifferentiation techniques:  $u$-substitution, integration by parts, trigonometric integration, and trigonometric substitution.  But we have seen that each only works in very specialized circumstances.  For example, while $\int xe^{x^2} \, dx$ may be evaluated by $u$-substitution and $\int x e^x \, dx$ by integration by parts, neither method provides a route to evaluate $\int e^{x^2} \, dx$.  That fact is not a particular shortcoming of these two antidifferentiation techniques, as it turns out there does not exist an elementary algebraic antiderivative for $e^{x^2}$.  Said differently, no matter what antidifferentiation methods we could develop and learn to execute, none of them will be able to provide us with a simple formula that does not involve integrals for a function $F(x)$ that satisfies $F'(x) = e^{x^2}$.

In this section of the text, our main goals are to better understand some classes of functions that can always be antidifferentiated, as well as to learn some options for so doing.  

\input{previews/5.4.PA1} % PREVIEW ACTIVITY

%------------------------------------------------------------------
% SUBSECTION THE METHOD OF PARTIAL FRACTIONS
%------------------------------------------------------------------
\subsection*{The Method of Partial Fractions} \index{partial fractions}

The method of partial fractions is used to integrate rational functions, and essentially involves reversing the process of finding a common denominator.  For example, suppose we have the function $R(x) = \frac{5x}{x^2 - x - 2}$ and want to evaluate
$$\int \frac{5x}{x^2-x-2} \, dx.$$
Thinking algebraically, if we factor the denominator, we can see how $R$ might come from the sum of two fractions of the form $\frac{A}{x-2} + \frac{B}{x+1}.$  In particular, suppose that
$$\frac{5x}{(x-2)(x+1)} = \frac{A}{x-2} + \frac{B}{x+1}.$$
Multiplying both sides of this last equation by $(x-2)(x+1)$, we find that
$$5x = A(x+1) + B(x-2).$$
Since we want this equation to hold for every value of $x$, we can use insightful choices of specific $x$-values to help us find $A$ and $B$.  Taking $x = -1$, we have
$$5(-1) = A(0) + B(-3),$$
and thus $B = \frac{5}{3}$.  Choosing $x = 2$, it follows
$$5(2) = A(3) + B(0),$$
so $A = \frac{10}{3}.$
Therefore, we now know that 
$$\int \frac{5x}{x^2-x-2} \, dx = \int \frac{10/3}{x-2} + \frac{5/3}{x+1} \, dx.$$
This equivalent integral expression is straightforward to evaluate, and hence we find that
$$\int \frac{5x}{x^2-x-2} \, dx = \frac{10}{3} \ln|x-2| + \frac{5}{3}\ln|x+1| + C.$$

It turns out that for any rational function $R(x) = \frac{P(x)}{Q(x)}$ where the degree of the polynomial $P$ is less than the degree of the polynomial $Q$, the method of partial fractions can be used to rewrite the rational function as a sum of simpler rational functions of one of the following forms:
$$\frac{A}{x-c}, \enskip \frac{A}{(x-c)^n}, \enskip \mbox{or} \enskip \frac{Ax+B}{x^2 + k}$$
where $A$, $B$, and $c$ are real numbers, and $k$ is a positive real number.  Because each of these basic forms is one we can antidifferentiate, partial fractions enables us to antidifferentiate any rational function.

\marginnote[-4cm]{If the degree of $P$ is greater than or equal to the degree of $Q$, long division may be used to write $R$ as the sum of a polynomial plus a rational function where the numerator's degree is less than the denominator's.}

\concept{Partial Fraction Decomposition} % CONCEPT
{Let $\ds \frac{p(x)}{q(x)}$ be a rational function, where the degree of $p$ is less than the degree of $q$.\index{integration!partial fraction decomp.}
\begin{enumerate}[1)]
	\item	\textbf{Linear Terms:} Let $(x-a)$ divide $q(x)$, where $(x-a)^n$ is the highest power of $(x-a)$ that divides $q(x)$. Then the decomposition of $\frac{p(x)}{q(x)}$ will contain the sum
	$$\frac{A_1}{(x-a)} + \frac{A_2}{(x-a)^2} + \cdots +\frac{A_n}{(x-a)^n}.$$
	\item		\textbf{Quadratic Terms:} Let $x^2+bx+c$ divide $q(x)$, where $(x^2+bx+c)^n$ is the highest power of $x^2+bx+c$ that divides $q(x)$. Then the decomposition of $\frac{p(x)}{q(x)}$ will contain the sum 
	$$\frac{B_1x+C_1}{x^2+bx+c}+\frac{B_2x+C_2}{(x^2+bx+c)^2}+\cdots+\frac{B_nx+C_n}{(x^2+bx+c)^n}.$$
	\end{enumerate}
	To find the coefficients $A_i$, $B_i$ and $C_i$:
	\begin{enumerate}[1)]
	\item	Multiply all fractions by $q(x)$, clearing the denominators. Collect like terms.
	\item		Equate the resulting coefficients of the powers of $x$ and solve the resulting system of linear equations.
	\end{enumerate}
	To integrate rational functions of the form  $\ds \frac{1}{a^2 + x^2}$,  use
	$$\int \frac{1}{a^2 + x^2} = \frac{1}{a} \arctan\left( \frac{x}{a} \right) + c $$
} % end concept

\begin{example} \label{eg:5.4.1} % EXAMPLE
Perform the partial fraction decomposition of $\ds \frac{1}{x^2-1}$.

\solution The denominator factors into two linear terms: $x^2-1 = (x-1)(x+1)$. Thus 
$$\frac{1}{x^2-1} = \frac{A}{x-1} + \frac{B}{x+1}.$$
To solve for $A$ and $B$, first multiply through by $x^2-1 = (x-1)(x+1)$:
\begin{align*}
1 &= \frac{A(x-1)(x+1)}{x-1}+\frac{B(x-1)(x+)}{x+1} \\
	&= A(x+1) + B(x-1)\\
	&= Ax+A + Bx-B \\
	\intertext{Now collect like terms.}
	&= (A+B)x + (A-B).
\end{align*}
The next step is key. Note the equality we have:
$$1 = (A+B)x+(A-B).$$
For clarity's sake, rewrite the left hand side as
$$0x+1 = (A+B)x+(A-B).$$
On the left, the coefficient of the $x$ term is $0$; on the right, it is $(A+B)$. Since both sides are equal, we must have that $0=A+B$. 

Likewise, on the left, we have a constant term of $1$; on the right, the constant term is $(A-B)$. Therefore we have $1=A-B$.

We have two linear equations with two unknowns. This one is easy to solve by hand, leading to 
$$\begin{array}{c} A+B = 0 \\ A-B = 1 \end{array} \Rightarrow \begin{array}{c} A=1/2 \\ B = -1/2\end{array}.$$
Thus $$\frac{1}{x^2-1} = \frac{1/2}{x-1}-\frac{1/2}{x+1}.$$
\end{example} % EXAMPLE

\begin{example} \label{eg:5.4.2} % EXAMPLE
Use partial fraction decomposition to integrate $\ds\int\frac{1}{(x-1)(x+2)^2}\ dx.$

\solution We decompose the integrand as follows:
$$\frac{1}{(x-1)(x+2)^2} = \frac{A}{x-1} + \frac{B}{x+2} + \frac{C}{(x+2)^2}.$$
To solve for $A$, $B$ and $C$, we multiply both sides by $(x-1)(x+2)^2$ and collect like terms:
\begin{align}
1 &= A(x+2)^2 + B(x-1)(x+2) + C(x-1)\label{eq:pf3}\\
	&= Ax^2+4Ax+4A + Bx^2 + Bx-2B + Cx-C \notag \\
	&= (A+B)x^2 + (4A+B+C)x + (4A-2B-C)\notag
\end{align}
Equation~\ref{eq:pf3} offers a direct route to finding the values of $A$, $B$ and $C$. Since the equation holds for all values of $x$, it holds in particular when $x=1$. However, when $x=1$, the right hand side simplifies to $A(1+2)^2 = 9A$. Since the left hand side is still $1$, we have $1 = 9A$. Hence $A = 1/9$.

Likewise, the equality holds when $x=-2$; this leads to the equation $1=-3C$. Thus $C = -1/3$.

We can find the value of $B$ by expanding the terms as shown in the example.

We have $$0x^2+0x+ 1 = (A+B)x^2 + (4A+B+C)x + (4A-2B-C)$$
leading to the equations 
$$A+B = 0, \quad 4A+B+C = 0 \quad \text{and} \quad 4A-2B-C = 1.$$
These three equations of three unknowns lead to a unique solution:
$$A = 1/9,\quad B = -1/9 \quad \text{and} \quad C = -1/3.$$
Thus 
$$\int\frac{1}{(x-1)(x+2)^2}\ dx = \int \frac{1/9}{x-1}\ dx + \int \frac{-1/9}{x+2}\ dx + \int \frac{-1/3}{(x+2)^2}\ dx.$$

Each can be integrated with a simple substitution with $u=x-1$ or $u=x+2$. The end result is
$$\int\frac{1}{(x-1)(x+2)^2}\ dx = \frac19\ln|x-1| -\frac19\ln|x+2| +\frac1{3(x+2)}+C.$$
\end{example} % EXAMPLE

\begin{activity} \label{A:5.4.1} For each of the following problems, evaluate the integral by using the method of partial fractions.

\bmthree
\ba
\item $\ds \int \frac{1}{x^2 - 2x - 3} \, dx$
	
\item $\ds \int \frac{x^2+1}{x^3 - x^2} \, dx$
	
\item $\ds \int \frac{x-2}{x^4 + x^2}\, dx$
\ea
\emthree

\end{activity}
\begin{smallhint}
\ba
	\item Small hints for each of the prompts above.
\ea
\end{smallhint}
\begin{bighint}
\ba
	\item Big hints for each of the prompts above.
\ea
\end{bighint}
\begin{activitySolution}
\ba
	\item Solutions for each of the prompts above.
\ea
\end{activitySolution}
\aftera % ACTIVITY

\begin{example} \label{eg:5.4.3} % EXAMPLE
Use partial fraction decomposition to integrate $\ds \int \frac{x^3}{(x-5)(x+3)}\ dx$.

\solution Partial fraction decomposition presumes that the degree of the numerator is less than the degree of the denominator. Since this is not the case here, we begin by using polynomial division to reduce the degree of the numerator. We omit the steps, but encourage the reader to verify that $$\frac{x^3}{(x-5)(x+3)} = x+2+\frac{19x+30}{(x-5)(x+3)}.$$
We can rewrite the new rational function as:
$$\frac{19x+30}{(x-5)(x+3)} = \frac{A}{x-5} + \frac{B}{x+3}$$ for appropriate values of $A$ and $B$. Clearing denominators, we have 

\begin{align*}
19x+30 &= A(x+3) + B(x-5)\\
			&= (A+B)x + (3A-5B).
\intertext{This implies that:}
19&= A+B \\
30&= 3A-5B.\\
\intertext{Solving this system of linear equations gives}
125/8 &=A\\
27/8 &=B.
\end{align*}

We can now integrate.
\begin{align*}
\int \frac{x^3}{(x-5)(x+3)}\ dx &= \int\left(x+2+\frac{125/8}{x-5}+\frac{27/8}{x+3}\right)\ dx \\
&= \frac{x^2}2 + 2x + \frac{125}{8}\ln|x-5| + \frac{27}8\ln|x+3| + C.
\end{align*}
\end{example} % EXAMPLE

\begin{example} \label{eg:5.4.4} % EXAMPLE
Use partial fraction decomposition to evaluate $\ds \int\frac{7x^2+31x+54}{(x+1)(x^2+6x+11)}\ dx.$

\solution The degree of the numerator is less than the degree of the denominator so we have:
\begin{align*}
\frac{7x^2+31x+54}{(x+1)(x^2+6x+11)} &= \frac{A}{x+1} + \frac{Bx+C}{x^2+6x+11}. \\
\intertext{Now clear the denominators.}
7x^2+31x+54 &= A(x^2+6x+11) + (Bx+C)(x+1)\\
					&= (A+B)x^2 + (6A+B+C)x + (11A+C).\\
\intertext{This implies that:}
				7&=A+B\\
				31 &= 6A+B+C\\
				54 &= 11A+C.
\end{align*}
Solving this system of linear equations gives the nice result of $A=5$, $B = 2$ and $C=-1$. Thus
$$\int\frac{7x^2+31x+54}{(x+1)(x^2+6x+11)}\ dx = \int\left(\frac{5}{x+1} + \frac{2x-1}{x^2+6x+11}\right)\ dx.$$

The first term of this new integrand is easy to evaluate; it leads to a $5\ln|x+1|$ term. The second term is not hard, but takes several steps and uses substitution techniques.

The integrand $\ds \frac{2x-1}{x^2+6x+11}$ has a quadratic in the denominator and a linear term in the numerator. This leads us to try substitution. Let $u = x^2+6x+11$, so $du = (2x+6)\ dx$. The numerator is $2x-1$, not $2x+6$, but we can get a $2x+6$ term in the numerator by adding $0$ in the form of ``$7-7$.''
\begin{align*}
\frac{2x-1}{x^2+6x+11} &= \frac{2x-1+7-7}{x^2+6x+11} \\
					&= \frac{2x+6}{x^2+6x+11} - \frac{7}{x^2+6x+11}.
\end{align*}
We can now integrate the first term with substitution, leading to a $\ln|x^2+6x+11|$ term. The final term can be integrated using arctangent. First, complete the square in the denominator:
$$\frac{7}{x^2+6x+11} = \frac{7}{(x+3)^2+2}.$$
An antiderivative of the latter term is
$$\int \frac{7}{x^2+6x+11}\ dx = \frac{7}{\sqrt{2}}\arctan\left(\frac{x+3}{\sqrt{2}}\right)+C.$$

Let's start at the beginning and put all of the steps together.
\begin{align*}
\int\frac{7x^2+31x+54}{(x+1)(x^2+6x+11)}\ dx &= \int\left(\frac{5}{x+1} + \frac{2x-1}{x^2+6x+11}\right)\ dx \\
\end{align*}
\vspace*{-.75cm}
$$= \int\frac{5}{x+1}\ dx  + \int\frac{2x+6}{x^2+6x+11}\ dx -\int\frac{7}{x^2+6x+11}\ dx$$
$$= 5\ln|x+1|+ \ln|x^2+6x+11| -\frac{7}{\sqrt{2}}\arctan\left(\frac{x+3}{\sqrt{2}}\right)+C.$$
As with many other problems in calculus, it is important to remember that one is not expected to ``see'' the final answer immediately after seeing the problem. Rather, given the initial problem, we break it down into smaller problems that are easier to solve. The final answer is a combination of the answers of the smaller problems.
\end{example} % EXAMPLE

\begin{activity} \label{A:5.4.2} For each of the following problems, evaluate the integral by using the method of partial fractions.

\bmtwo
\ba
\item $\ds \int \frac{x^5 + x^2 + 2}{x^3 - x} \, dx$
	
\item $\ds \int \frac{x^5 - 4x^3 + 1}{x^3 - 4x} \, dx$
	
\item $\ds \int \frac{x^3 - 2x^2 + 2x - 2}{x^2 + 1}\, dx$

\item $\ds \int \frac{x^4 + 6x^3 + 10x^2 + x}{x^2 + 6x + 10}\, dx$
\ea
\emtwo

\end{activity}
\begin{smallhint}
\ba
	\item Small hints for each of the prompts above.
\ea
\end{smallhint}
\begin{bighint}
\ba
	\item Big hints for each of the prompts above.
\ea
\end{bighint}
\begin{activitySolution}
\ba
	\item Solutions for each of the prompts above.
\ea
\end{activitySolution}
\aftera % ACTIVITY

Our final example demonstrates how to set up the partial fraction decomposition when the denominator contains each type of factor(a linear factor, a repeated linear factor, a quadratic factor as well as a repeated quadratic factor.

\begin{example} \label{eg:5.4.5} % EXAMPLE
Decompose $\ds f(x)=\frac{1}{(x+5)(x-2)^3(x^2+2x+1)(x^2+x+7)^2}$ without solving for the resulting coefficients.

\solution
The denominator is already factored; we need to decompose $f(x)$ properly. Since $(x+5)$ is a linear term that divides the denominator, there will be a $\ds\frac{A}{x+5}$ term in the decomposition.

As $(x-2)^3$ divides the denominator, we will have the following terms in the decomposition:
$$\frac{B}{x-2},\quad \frac{C}{(x-2)^2}\quad \text{and}\quad \frac{D}{(x-2)^3}.$$

The $x^2+2x+1$ term in the denominator results in a $\ds\frac{Ex+F}{x^2+2x+1}$ term.

Finally, the $(x^2+x+7)^2$ term results in the terms $$\frac{Gx+H}{x^2+x+7}\quad \text{and}\quad \frac{Ix+J}{(x^2+x+7)^2}.$$
All together, we have $\ds \frac{1}{(x+5)(x-2)^3(x^2+2x+1)(x^2+x+7)^2} =$\footnotesize

\[ \frac{A}{x+5} + \frac{B}{x-2}+ \frac{C}{(x-2)^2}+\frac{D}{(x-2)^3}+ \frac{Ex+F}{x^2+2x+1}+ \frac{Gx+H}{x^2+x+7}+\frac{Ix+J}{(x^2+x+7)^2}\]\small

Solving for the coefficients $A$, $B \ldots J$ would be a bit tedious but not ``hard.''

\end{example} % EXAMPLE

Performing the partial fraction decomposition can be very tedious but is an important technique that will be used to solve differential equations. Problems such as the one in Example \ref{eg:5.4.5} can be decomposed using a computer algebra system.
A computer algebra system such as \emph{Maple}, \emph{Mathematica}, or \emph{WolframAlpha} can be used to find the partial fraction decomposition of any rational function.  In \emph{WolframAlpha}, entering 
\begin{quote}
\texttt{partial fraction 5x/(x\^{}2-x-2)}
\end{quote}
results in the output $\ds \frac{5x}{x^2-x-2} = \frac{10}{3(x-2)} + \frac{5}{3(x+1)}.$

%\begin{activity} \label{A:5.5.1}  In this activity we explore the improper integrals $\ds \int_1^{\infty} \frac{1}{x} \, dx$ and $\ds \int_1^{\infty} \frac{1}{x^{3/2}} \, dx$.
\ba
	\item First we investigate $\ds \int_1^{\infty} \frac{1}{x} \, dx$.
	\be
		\item[i.] Use the First FTC to determine the exact values of $\ds \int_1^{10} \frac{1}{x} \, dx$, $\ds \int_1^{1000} \frac{1}{x} \, dx$, and $\ds \int_1^{100000} \frac{1}{x} \, dx$.  Then, use your calculator to compute a decimal approximation of each result.
		\item[ii.]  Use the First FTC to evaluate the definite integral $\ds \int_1^{b} \frac{1}{x} \, dx$ (which results in an expression that depends on $b$).
		\item[iii.]  Now, use your work from (ii.) to evaluate the limit given by
	$$\lim_{b \to \infty}  \int_1^{b} \frac{1}{x} \, dx.$$
	\ee
	\item Next, we investigate $\ds \int_1^{\infty} \frac{1}{x^{3/2}} \, dx$.
	\be
		\item[i.] Use the First FTC to determine the exact values of $\ds \int_1^{10} \frac{1}{x^{3/2}} \, dx$, $\ds \int_1^{1000} \frac{1}{x^{3/2}} \, dx$, and $\ds \int_1^{100000} \frac{1}{x^{3/2}} \, dx$.  Then, use your calculator to compute a decimal approximation of each result.
		\item[ii.]  Use the First FTC to evaluate the definite integral $\ds \int_1^{b} \frac{1}{x^{3/2}} \, dx$ (which results in an expression that depends on $b$).
		\item[iii.]  Now, use your work from (ii.) to evaluate the limit given by
	$$\lim_{b \to \infty}  \int_1^{b} \frac{1}{x^{3/2}} \, dx.$$
	\ee
	\item Plot the functions $y = \frac{1}{x}$ and $y = \frac{1}{x^{3/2}}$ on the same coordinate axes for the values $x = 0 \ldots 10$.  How would you compare their behavior as $x$ increases without bound?  What is similar?  What is different?
	\item How would you characterize the value of $\ds \int_1^{\infty} \frac{1}{x} \, dx$? of $\ds \int_1^{\infty} \frac{1}{x^{3/2}} \, dx$?  What does this tell us about the respective areas bounded by these two curves for $x \ge 1$?
\ea

\end{activity}
\begin{smallhint}
\ba
	\item Small hints for each of the prompts above.
\ea
\end{smallhint}
\begin{bighint}
\ba
	\item Big hints for each of the prompts above.
\ea
\end{bighint}
\begin{activitySolution}
\ba
	\item Solutions for each of the prompts above.
\ea
\end{activitySolution}
\aftera % ACTIVITY

%-------------
% SUMMARY
%-------------
\begin{summary}
\item The method of partial fractions enables any rational function to be antidifferentiated, because any polynomial function can be factored into a product of linear and irreducible quadratic terms.  This allows any rational function to be written as the sum of a polynomial plus rational terms of the form $\frac{A}{(x-c)^n}$ (where $n$ is a natural number) and $\frac{Bx+C}{x^2 + k}$ (where $k$ is a positive real number).
%\item Until the development of computing algebra systems, integral tables enabled students of calculus to more easily evaluate integrals such as $\int \sqrt{a^2 + u^2} \, du$, where $a$ is a positive real number.  A short table of integrals may be found in Appendix~\ref{C:9.IntegralTable}.
%\item Computer algebra systems can play an important role in finding antiderivatives, though we must be cautious to use correct input, to watch for unusual or unfamiliar advanced functions that the CAS may cite in its result, and to consider the possibility that a CAS may need further assistance or insight from us in order to answer a particular question.
\end{summary}

\clearpage

%--------------
% EXERCISES
%--------------
\begin{adjustwidth*}{}{-2.25in}
\textbf{{\large Exercises}}
\setlength{\columnsep}{25pt}
\begin{multicols*}{2}
\noindent Terms and Concepts \small
\begin{enumerate}[1)]
\item Fill in the blank: Partial Fraction Decomposition is a method of rewriting \underline{\hskip .5in} functions.
\item T/F: It is sometimes necessary to use polynomial division before using Partial Fraction Decomposition.
\item Decompose $\ds \frac{1}{x^2-3x}$ without solving for the coefficients.
\item Decompose $\ds \frac{7-x}{x^2-9}$ without solving for the coefficients.
\item Decompose $\ds \frac{x-3}{x^2-7}$ without solving for the coefficients.
\item Decompose $\ds \frac{2x+5}{x^3+7x}$ without solving for the coefficients.
\end{enumerate} 

\noindent {\normalsize Problems} \small

\noindent{\bf In exercises 7--25, evaluate the indefinite integral.}

\begin{enumerate}[1),resume]
\item $\ds \int \frac{7x+7}{x^2+3x-10}\ dx$
\item $\ds \int \frac{7x-2}{x^2+x}\ dx$
\item $\ds \int \frac{-4}{3x^2-12}\ dx$
\item $\ds \int \frac{x+7}{(x+5)^2}\ dx$
\item $\ds \int \frac{-3x-20}{(x+8)^2}\ dx$
\item $\ds \int \frac{9x^2+11x+7}{x(x+1)^2}\ dx$
\item $\ds \int \frac{-12x^2-x+33}{(x-1)(x+3)(3-2x)}\ dx$
\item $\ds \int \frac{94x^2-10x}{(7x+3)(5x-1)(3x-1)}\ dx$
\item $\ds \int \frac{x^2+x+1}{x^2+x-2}\ dx$
\item $\ds \int \frac{x^3}{x^2-x-20}\ dx$
\item $\ds \int \frac{2x^2-4x+6}{x^2-2x+3}\ dx$
\item $\ds \int \frac{1}{x^3+2x^2+3x}\ dx$
\item $\ds \int \frac{x^2+x+5}{x^2+4x+10}\ dx$
\item $\ds \int \frac{12x^2+21x+3}{(x+1)(3x^2+5x-1)}\ dx$
\item $\ds \int \frac{6x^2+8x-4}{(x-3)(x^2+6x+10)}\ dx$
\item $\ds \int \frac{2x^2+x+1}{(x+1)(x^2+9)}\ dx$
\item $\ds \int \frac{x^2-20x-69}{(x-7)(x^2+2x+17)}\ dx$
\item $\ds \int \frac{9x^2-60x+33}{(x-9)(x^2-2x+11)}\ dx$
\item $\ds \int \frac{6x^2+45x+121}{(x+2)(x^2+10x+27)}\ dx$
\end{enumerate}

\noindent{\bf In exercises 26--29, evaluate the definite integral.}

\begin{enumerate}[1),resume]
\item $\ds \int_1^2 \frac{8x+21}{(x+2)(x+3)}\ dx$
\item $\ds \int_0^5 \frac{14x+6}{(3x+2)(x+4)}\ dx$
\item $\ds \int_{-1}^1 \frac{x^2+5x-5}{(x-10)(x^2+4x+5)}\ dx$
\item $\ds \int_{0}^1 \frac{x}{(x+1)(x^2+2x+1)}\ dx$
\end{enumerate}

%------------------------------------------
% END OF EXERCISES ON FIRST PAGE
%------------------------------------------
\end{multicols*}
\end{adjustwidth*}

%\clearpage
%
%\begin{adjustwidth*}{}{-2.25in}
%\setlength{\columnsep}{25pt}
%\begin{multicols*}{2}\small
%
%\end{enumerate}
%
%%---------------------------------------------
%% END OF EXERCISES ON SECOND PAGE
%%---------------------------------------------
%\end{multicols*}
%\end{adjustwidth*}

\afterexercises 

\cleardoublepage