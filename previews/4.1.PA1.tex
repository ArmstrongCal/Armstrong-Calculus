\begin{marginfigure}
\margingraphics{figs/4/preview_4-1a.pdf}
\caption{Axes for plotting $y = v(t)$.} \label{fig:4.1.PA1a}
\end{marginfigure}

\begin{marginfigure}[.25in]
\margingraphics{figs/4/preview_4-1b.pdf}
\caption{Axes for plotting $y = s(t)$.} \label{fig:4.1.PA1b}
\end{marginfigure}

\begin{pa} \label{PA:4.1}
Suppose that a person is taking a walk along a long straight path and walks at a constant rate of 3 miles per hour.
\ba
\item On the axes provided in Figure~\ref{fig:4.1.PA1a}, sketch a labeled graph of the velocity function $v(t) = 3$.  

Note that while the scale on the two sets of axes is the same, the units on the axes provided in Figure~\ref{fig:4.1.PA1a} differ from those in Figure~\ref{fig:4.1.PA1b}.

\item How far did the person travel during the two hours?  How is this distance related to the area of a certain region under the graph of $y = v(t)$?

\item Find an algebraic formula, $s(t)$, for the position of the person at time $t$, assuming that $s(0) = 0$.  Explain your thinking.

\item On the axes provided in Figure~\ref{fig:4.1.PA1b}, sketch a labeled graph of the position function $y = s(t)$.

\item For what values of $t$ is the position function $s$ increasing?  Explain why this is the case using relevant information about the velocity function $v$.
\ea
\end{pa} 
\afterpa