\begin{pa} \label{PA:5.5}  A company with a large customer base has a call center that receives thousands of calls a day.  After studying the data that represents how long callers wait for assistance, they find that the function $p(t) = 0.25e^{-0.25t}$ models the time customers wait in the following way:  the fraction of customers who wait between $t = a$ and $t = b$ minutes is given by
$$\int_a^b p(t) \, dt.$$  
Use this information to answer the following questions.
\ba
	\item Determine the fraction of callers who wait between $5$ and $10$ minutes.
	\item Determine the fraction of callers who wait between $10$ and $20$ minutes.
	\item Next, let's study how the fraction who wait up to a certain number of minutes:
	\begin{itemize}
		\item[i.] What is the fraction of callers who wait between $0$ and $5$ minutes?
		\item[ii.] What is the fraction of callers who wait between $0$ and $10$ minutes?
		\item[iii.] Between $0$ and $15$ minutes?  Between $0$ and $20$? 
	\end{itemize}
	\item Let $F(b)$ represent the fraction of callers who wait between $0$ and $b$ minutes.  Find a formula for $F(b)$ that involves a definite integral, and then use the First FTC to find a formula for $F(b)$ that does not involve a definite integral.
	\item What is the value of $\ds \lim_{b \to \infty} F(b)$?  Why?
\ea
\end{pa} 
\afterpa