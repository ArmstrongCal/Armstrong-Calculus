\begin{pa} \label{PA:8.4}
Preview Activity \ref{PA:8.3} showed how we can approximate the number $e$ with linear, quadratic, and other polynomial approximations. We use a similar approach in this activity to obtain linear and quadratic approximations to $\ln(2)$. Along the way, we encounter a type of series that is different than most of the ones we have seen so far. Throughout this activity, let $f(x) = \ln(1+x)$.

\ba
\item Find the tangent line to $f$ at $x=0$ and use this linearization to approximate $\ln(2)$.  That is, find $L(x)$, the tangent line approximation to $f(x)$, and use the fact that $L(1) \approx f(1)$ to estimate $\ln(2)$.

\begin{activitySolution}

The linearization of $f$ at $x=a$ is
\[f(a) + f'(a)(x-a),\]
so the linearization $P_1(x)$ of $f(x) = \ln(1+x)$ at $x=0$ is
\[P_1(x) = 0 + \frac{1}{1+0}(x-0) = x.\]
Now
\[f(x) \approx P_1(x)\]
for $x$ close to $0$ and so
\[\ln(2) = \ln(1+1) \approx P_1(1) = 1.\]

\end{activitySolution}


\item The linearization of $\ln(1+x)$ does not provide a very good approximation to $\ln(2)$ since 1 is not that close to 0. To obtain a better approximation, we alter our approach; instead of using a straight line to approximate $\ln(2)$, we use a quadratic function to account for the concavity of $\ln(1+x)$ for $x$ close to 0. With the linearization, both the function's value and slope agree with the linearization's value and slope at $x=0$. We will now make a quadratic approximation $P_2(x)$ to $f(x) = \ln(1+x)$ centered at $x=0$ with the property that $P_2(0) = f(0)$, $P'_2(0) = f'(0)$, \text{and} $P''_2(0) = f''(0)$. 
    \begin{itemize}
    \item[(i)] Let $P_2(x) =  x - \frac{x^2}{2}$. Show that $P_2(0) = f(0)$, $P'_2(0) = f'(0)$, \text{and} $P''_2(0) = f''(0)$. Use $P_2(x)$ to approximate $\ln(2)$ by using the fact that $P_2(1) \approx f(1)$.

\begin{activitySolution}

The derivatives of $P_2$ and $f$ are
\begin{align*}
P_2(x) &= x - \frac{x^2}{2} & f(x) &= \ln(1+x) \\
P'_2(x) &= 1-x & f'(x) &= \frac{1}{1+x} \\
P''_2(x) &= -1 & f''(x) &= -\frac{1}{(1+x)^2},
\end{align*}
and so the derivatives of $P_2$ and $f$ evaluated at 0 are
\begin{align*}
P_2(0) &= 0 & f(0) &= \ln(1) = 0 \\
P'_2(0) &= 1 & f'(0) &= \frac{1}{1+0} = 1 \\
P''_2(0) &= -1 &= -\frac{1}{(1+0)^2} = -1.
\end{align*}
Then
\[\ln(2) = \ln(1+1) \approx P_2(1) = 1 - \frac{1}{2} = \frac{1}{2}.\]

\end{activitySolution}

    \item[(ii)] We can continue approximating $\ln(2)$ with polynomials of larger degree whose derivatives agree with those of $f$ at 0. This makes the polynomials fit the graph of $f$ better for more values of $x$ around 0. For example, let $P_3(x) = x - \frac{x^2}{2}+\frac{x^3}{3}$. Show that $P_3(0) = f(0)$, $P'_3(0) = f'(0)$, $P''_3(0) = f''(0)$, and $P'''_3(0) = f'''(0)$. Taking a similar approach to preceding questions, use $P_3(x)$ to approximate $\ln(2)$.

\begin{activitySolution}

The derivatives of $P_3$ and $f$ are
\begin{align*}
P_3(x) &= x-\frac{x^2}{2}+\frac{x^3}{3} & f(x) &= \ln(1+x) \\
P'_3(x) &= 1 - x + x^2  & f'(x) &= \frac{1}{1+x} \\
P''_3(x) &= -1+2x  & f''(x) &= -\frac{1}{(1+x)^2} \\
P'''_3(x) &= 2  & f'''(x) &= \frac{2}{(1+x)^3},
\end{align*}
and so the derivatives of $P_3$ and $f$ evaluated at 0 are
\begin{align*}
P_3(0) &= 0 & f(0) &= \ln(1+0) = 0 \\
P'_3(0) &= 1 & f'(0) &= \frac{1}{1+0} = 1 \\
P''_3(0) &= -1 & f''(0) &= -\frac{1}{(1+0)^2} = -1 \\
P'''_3(0) &= 2 & f'''(0) &= \frac{2}{(1+0)^3} = 2.
\end{align*}
Then
\[\ln(2) = \ln(1+1) \approx P_3(0) = 1 - \frac{1}{2} + \frac{1}{3} = \frac{5}{6} \approx 0.83.\]

\end{activitySolution}

    \item[(iii)]  If we used a degree 4 or degree 5 polynomial to approximate $\ln(1+x)$, what approximations of $\ln(2)$ do you think would result?  Use the preceding questions to conjecture a pattern that holds, and state the degree 4 and degree 5 approximation.

    \end{itemize}

\ea

\end{pa}
\afterpa 