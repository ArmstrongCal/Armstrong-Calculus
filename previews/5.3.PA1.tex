\begin{pa} \label{PA:5.3}
In Section~\ref{S:2.8.Inverse}, we discussed derivatives of inverse functions including the inverse trigonometric functions $\arcsin (x)$, $\arccos (x)$, and $\arctan (x)$. We will need the techniques to find the derivatives to evaluate integrals of a certain form. Let's review the technique.
Suppose $f(x) = \arcsin (x)$ for $-1<x<1$ and $g(x)=\arctan(x)$ for $-1<x<1$
	\be
		\item Rewrite $f(x)$ as $\theta$. Label the right triangle that corresponds to $\theta=\arcsin(x)$. 
		\item Use the completed triangle to show $$ \tan (\arcsin (x)) = \frac{x}{1-x^2} $$
		\item Rewrite $g(x)$ as $\phi$. Label the right triangle that corresponds to $\phi=\arctan(x)$.
		\item Use the completed triangle to show $$ \cos (\arctan (x)) = \frac{1}{\sqrt{1+x^2}} $$
	\ee
\end{pa} 
\afterpa

%add two triangles like figure 2.41