\begin{marginfigure}[1cm] % MARGIN FIGURE
\margingraphics{figs/4/4-3_EckApplet.pdf}
\caption{A left Riemann sum with 5 subintervals for the function $f(x) = \frac{3}{1+x^2}$ on the interval $[-5,5]$.  The value of the sum is $L_5 = 7.43076923$.} \label{fig:4-3_EckApplet}
\end{marginfigure}

\begin{marginfigure}[1cm] % MARGIN FIGURE
\margingraphics{figs/4/4-3_EckApplet2.pdf}
\caption{A left Riemann sum with 5 subintervals for the function $f(x) = 2x+1$ on the interval $[1,4]$.  The value of the sum is $L_5 = 16.2$.} \label{fig:4-3_EckApplet2}
\end{marginfigure}

\begin{pa} \label{PA:4.3}
Consider the applet found at \href{http://gvsu.edu/s/aw}{\texttt{http://gvsu.edu/s/aw}}.  There, you will initially see the situation shown in Figure~\ref{fig:4-3_EckApplet}.

Observe that we can change the window in which the function is viewed, as well as the function itself.  Set the minimum and maximum values of $x$ and $y$ so that we view the function on the window where $1 \le x \le 4$ and $-1 \le y \le 12$, where the function is $f(x) = 2x + 1$ (note that you need to enter ``\texttt{2*x+1}'' as the function's formula).  You should see the updated figure shown in Figure~\ref{fig:4-3_EckApplet2}.

Note that the value of the Riemann sum of our choice is displayed in the upper left corner of the window.  Further, by updating the value in the ``Intervals'' window and/or the ``Method'', we can see the different value of the Riemann sum that arises by clicking the ``Compute!'' button.
\ba
	\item Update the applet so that the function being considered is $f(x) = 2x+1$ on $[1,4]$, as directed above.  For this function on this interval, compute $L_{n}$, $M_{n}$,  $R_{n}$ for $n = 10$, $n = 100$, and $n = 1000$.  What do you conjecture is the exact area bounded by $f(x) = 2x+1$ and the $x$-axis on $[1,4]$?
	\item Use basic geometry to determine the exact area bounded by $f(x) = 2x+1$ and the $x$-axis on $[1,4]$.
	\item Based on your work in (a) and (b), what do you observe occurs when we increase the number of subintervals used in the Riemann sum?
	\item Update the applet to consider the function $f(x) = x^2 + 1$ on the interval $[1,4]$ (note that you will want to increase the maximum value of $y$ to at least $17$, and you need to enter ``\texttt{x\^{}2 + 1}'' for the function formula).  Use the applet to compute $L_{n}$, $M_{n}$,  $R_{n}$ for $n = 10$, $n = 100$, and $n = 1000$.  What do you conjecture is the exact area bounded by $f(x) = x^2+1$ and the $x$-axis on $[1,4]$?
	\item Why can we not compute the exact value of the area bounded by $f(x) = x^2+1$ and the $x$-axis on $[1,4]$ using a formula like we did in (b)?
\ea
\end{pa} 
\afterpa