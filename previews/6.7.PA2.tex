\begin{pa} \label{PA:7.5}
Any time that the rate of change of a quantity is related to the amount of a quantity, a differential equation naturally arises.  In the following two problems, we see two such scenarios; for each, we want to develop a differential equation whose solution is the quantity of interest.
  \ba
\item Suppose you have a bank account in which money grows at an
  annual rate of 3\%.
  \begin{itemize}
    \item[(i)] If you have \$10,000 in the account, at what rate is your
      money growing?
    \item[(ii)] Suppose that you are also withdrawing money from the account
      at \$1,000 per year.  What is the rate of change in the amount
      of money in the account?  What are the units on this rate of change? 
  \end{itemize}
\item Suppose that a water tank holds 100 gallons and that a salty
  solution, which contains 20 grams of salt in every gallon, enters the
  tank at 2 gallons per minute.   
  \begin{itemize}
    \item[(i)] How much salt enters the tank each minute?
    \item[(ii)] Suppose that initially there are 300 grams of salt in the tank.  How
      much salt is in each gallon at this point in time?
    \item[(iii)] Finally, suppose that evenly mixed solution is pumped out of the tank at the
      rate of 2 gallons per minute.  How much salt leaves the tank
      each minute?
    \item[(iv)] What is the total rate of change in the amount of salt in
      the tank?
  \end{itemize}
\ea
\end{pa} 
\afterpa
