\begin{pa} \label{PA:7.3}
Consider the initial value problem 
$$
\frac{dy}{dt} = \frac12 (y + 1), \ y(0) = 0.
$$
\ba
\item Use the differential equation to find the slope of the tangent
  line to the solution $y(t)$ at $t=0$.  Then use the given initial value to
  find the equation of the tangent line at $t=0$.  

\item Sketch the tangent line on the axes below on the interval $0\leq
  t\leq 2$ and use it to approximate $y(2)$,
  the value of the solution at $t=2$.   

  \begin{center}
    \includegraphics{figures/7_3_PA.eps}
  \end{center}

\item Assuming that your approximation for $y(2)$ is the actual value
  of $y(2)$, use the differential equation to find the slope of the
  tangent line to $y(t)$ at $t=2$.  Then, write the equation of the
  tangent line at $t=2$.  

\item Add a sketch of this tangent line to your plot on the axes above on the interval $2\leq
  t\leq 4$; use this new tangent line to approximate $y(4)$,
  the value of the solution at $t=4$.   

\item Repeat the same step to find an approximation for $y(6)$.

\ea
\end{pa} 
\afterpa
