\begin{pa} \label{PA:4.4}
Consider the function $A$ defined by the rule
\[ A(x) = \int_1^x f(t) \ dt, \]
where $f(t) = 4-2t$ as seen in Figure~\ref{fig:preview_4-4}.

\ba
\item Compute $A(1)$ and $A(2)$ exactly using geometry.

\item Suppose that we can produce a formula for $A(x)$ that does not use integrals, and suppose that formula is $\ds A(x) = -x^2 + 4x - 3$.  Observe that $f$ is a linear function; what kind of function is this additional formula for $A$?
%Use the First Fundamental Theorem of Calculus to find an equivalent formula for $A(x)$ that does not involve integrals.  That is, use the first FTC to evaluate $\int_1^x (4-2t) \, dt$.

%\item Observe that $f$ is a linear function; what kind of function is $A$?

\item Using the additional formula for $A(x)$ found in (b) that does not involve integrals, compute $A(1)$, $A(2)$, and $A'(x)$.

\item While we have defined $f$ by the rule $f(t) = 4-2t$, it is equivalent to say that $f$ is given by the rule $f(x) = 4 - 2x$.  What do you observe about the relationship between $A$ and $f$?
\ea
\end{pa}
\begin{marginfigure}[-3cm] % MARGIN FIGURE
\margingraphics{figs/4/preview_4-4.pdf}
\caption{$f(t) = 4-2t$} \label{fig:preview_4-4}
\end{marginfigure} 
\afterpa