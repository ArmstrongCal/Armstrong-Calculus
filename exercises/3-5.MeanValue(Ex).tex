\begin{adjustwidth*}{}{-2.25in}
\textbf{{\large Exercises}}
\setlength{\columnsep}{25pt}
\begin{multicols*}{2}
\noindent Terms and Concepts \small
\begin{enumerate}[1)]
\item Explain in your own words what the Mean Value Theorem states.
\item Explain in your own words what Rolle's Theorem states.
\end{enumerate} 

\noindent {\normalsize Problems} \small

\noindent{\bf In exercises 3--10, a function $f(x)$ and interval $[a,b]$ are given.  Check if Rolle's Theorem can be applied, and if so, find $c$ in $[a,b]$ such that $f'(c) = 0$.}

\begin{enumerate}[1),resume]
\item $f(x) = 6$ on $[-1,1]$.
\item $f(x) = 6x$ on $[-1,1]$.
\item $f(x) = x^2+x-6$ on $[-3,2]$.
\item $f(x) = x^2+x-2$ on $[-3,2]$.
\item $f(x) = x^2+x$ on $[-2,2]$.
\item $f(x) = \sin x$ on $[\pi/6,5\pi/6]$.
\item $f(x) = \cos x$ on $[0,\pi]$.
\item $\ds f(x) = \frac{1}{x^2-2x+1}$ on $[0,2]$.
\end{enumerate}

\noindent{\bf In exercises 11-20, a function $f(x)$ and interval $[a,b]$ are given Check if the Mean Value Theorem can be applied, and if so, find a value $c$ guaranteed by the Mean Value Theorem.}

\begin{enumerate}[1),resume]
\item $\ds f(x) = x^2+3x-1$ on $[-2,2]$.
\item $\ds f(x) = 5x^2-6x+8$ on $[0,5]$.
\item $\ds f(x) = \sqrt{9-x^2}$ on $[0,3]$.
\item $\ds f(x) = \sqrt{25-x}$ on $[0,9]$.
\item $\ds f(x) = \ln x$ on $[1,5]$.
\item $\ds f(x) = \tan x$ on $[-\pi/4,\pi/4]$.
\item $\ds f(x) = x^3-2x^2+x+1$ on $[-2,2]$.
\item $\ds f(x) = 2x^3-5x^2+6x+1$ on $[-5,2]$.
\item $\ds f(x) = \sin^{-1} x$ on $[-1,1]$.
\item $\ds f(x) =\frac{x^2-9}{x^2-1}$ on $[0,2]$.
\end{enumerate}

\begin{enumerate}[1),resume]
\item Show that the equation $6x^4 -7x+1 =0$ does not have more
than two distinct real roots.
%\begin{answer}
%Seeking a contradiction, suppose that we have 3 real roots, call them
%$a$, $b$, and $c$. By Rolle's Theorem, $24x^3-7$ must have a root on
%both $(a,b)$ and $(b,c)$, but this is impossible as $24x^3-7$ has only
%one real root.
%\end{answer}

\item Let $f(x)$ be differentiable on $\mathbb{R}$. Suppose that $f'(x) \ne
0$ for every $x$. Prove that $f$ has at most one real root.
%\begin{answer}
%Seeking a contradiction, suppose that we have 2 real roots, call them
%$a$, $b$. By Rolle's Theorem, $f'(x)$ must have a root on $(a,b)$, but
%this is impossible.
%\end{answer}


\item Two runners start a race at the same time and finish in a tie. Prove that at some time during the race they have the same speed.
\end{enumerate}

%------------------------------------------
% END OF EXERCISES ON FIRST PAGE
%------------------------------------------
\end{multicols*}
\end{adjustwidth*}

\afterexercises