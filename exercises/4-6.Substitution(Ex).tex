\begin{adjustwidth*}{}{-2.25in}
\textbf{{\large Exercises}}
\setlength{\columnsep}{25pt}
\begin{multicols*}{2}
\noindent Terms and Concepts \small
\begin{enumerate}[1)]
\item Substitution ``undoes'' what derivative rule?
\item T/F: One can use algebra to rewrite the integrand of an integral to make it easier to evaluate.
\end{enumerate} 

\noindent {\normalsize Problems} \small

\noindent{\bf In exercises 3--14, evaluate the indefinite integral to develop an understanding of Substitution.}

\begin{enumerate}[1),resume]
\item $\ds \int 3 x^2 \left(x^3-5\right)^7 dx $
\item $\ds \int (2 x-5) \left(x^2-5 x+7\right)^3 dx $
\item $\ds \int x \left(x^2+1\right)^8 dx $
\item $\ds \int (12 x+14) \left(3 x^2+7 x-1\right)^5 dx $
\item $\ds \int \frac{1}{2 x+7} dx $
\item $\ds \int \frac{1}{\sqrt{2 x+3}} dx $
\item $\ds \int \frac{x}{\sqrt{x+3}} dx $
\item $\ds \int \frac{x^3-x}{\sqrt{x}} dx $
\item $\ds \int \frac{e^{\sqrt{x}}}{\sqrt{x}} dx $
\item $\ds \int \frac{x^4}{\sqrt{x^5+1}} dx $
\item $\ds \int \frac{\frac{1}{x}+1}{x^2} dx $
\item $\ds \int \frac{\ln(x)}{x} dx $
\end{enumerate}

\noindent{\bf In Exercises 15--21, use Substitution to evaluate the indefinite integral involving trigonometric functions.}

\begin{enumerate}[1),resume]
\item $\ds \int \sin ^2(x) \cos (x) dx $
\item $\ds \int \cos (3-6 x) dx $
\item $\ds \int \sec ^2(4-x) dx $
\item $\ds \int \sec (2 x) dx $
\item $\ds \int \tan ^2(x) \sec ^2(x) dx $
\item $\ds \int x \cos \left(x^2\right) dx $
\item $\ds \int \tan ^2(x) dx $
\end{enumerate}

\columnbreak

\noindent{\bf In Exercises 22--28, use Substitution to evaluate the indefinite integral involving exponential functions.}

\begin{enumerate}[1),resume]
\item $\ds \int e^{3 x-1} dx $
\item $\ds \int e^{x^3} x^2 dx $
\item $\ds \int e^{x^2-2 x+1} (x-1) dx $
\item $\ds \int \frac{e^x+1}{e^x} dx $
\item $\ds \int \frac{e^x-e^{-x}}{e^{2x}} dx $
\item $\ds \int 3^{3 x} dx $
\item $\ds \int 4^{2 x} dx $
\end{enumerate}

\noindent{\bf In Exercises 29--31, use Substitution to evaluate the indefinite integral involving logarithmic functions.}

\begin{enumerate}[1),resume]
\item $\ds \int \frac{\ln ^2(x)}{x} dx $
\item $\ds \int \frac{\ln \left(x^3\right)}{x} dx $
\item $\ds \int \frac{1}{x \ln \left(x^2\right)} dx $
\end{enumerate}

\noindent{\bf In Exercises 32--37, use Substitution to evaluate the indefinite integral involving rational functions.}

\begin{enumerate}[1),resume]
\item $\ds \int \frac{x^2 + 3x + 1}{x} dx $
\item $\ds \int \frac{x^3+x^2+x+1}{x} dx $
\item $\ds \int \frac{x^3-1}{x+1} dx $
\item $\ds \int \frac{x^2+2 x-5}{x-3} dx $
\item $\ds \int \frac{3 x^2-5 x+7}{x+1} dx $
\item $\ds \int \frac{x^2+2 x+1}{x^3+3 x^2+3 x} dx $
\end{enumerate}

\noindent{\bf In Exercises 38--47, use Substitution to evaluate the indefinite integral involving inverse trigonometric functions.}

\begin{enumerate}[1),resume]
\item $\ds \int \frac{7}{x^2+7} dx $
\item $\ds \int \frac{3}{\sqrt{9-x^2}} dx $
\item $\ds \int \frac{14}{\sqrt{5-x^2}} dx $
\item $\ds \int \frac{2}{x \sqrt{x^2-9}} dx $
\end{enumerate}

%------------------------------------------
% END OF EXERCISES ON FIRST PAGE
%------------------------------------------
\end{multicols*}
\end{adjustwidth*}

\clearpage

\begin{adjustwidth*}{}{-2.25in}
\setlength{\columnsep}{25pt}
\begin{multicols*}{2}\small

\begin{enumerate}[1),start=42]
\item $\ds \int \frac{5}{\sqrt{x^4-16 x^2}} dx $
\item $\ds \int \frac{x}{\sqrt{1-x^4}} dx $
\item $\ds \int \frac{1}{x^2-2 x+8} dx $
\item $\ds \int \frac{2}{\sqrt{-x^2+6 x+7}} dx $
\item $\ds \int \frac{3}{\sqrt{-x^2+8 x+9}} dx $
\item $\ds \int \frac{5}{x^2+6 x+34} dx $
\end{enumerate}

\vspace{.25cm}

\noindent{\bf In Exercises 48--72, use Substitution to evaluate the indefinite integral. }

\begin{enumerate}[1),resume]
\item $\ds \int \frac{x^2}{\left(x^3+3\right)^2} dx $
\item $\ds \int \left(3 x^2+2 x\right) \left(5 x^3+5 x^2+2\right)^8 dx $
\item $\ds \int \frac{x}{\sqrt{1-x^2}} dx $
\item $\ds \int x^2 \csc ^2\left(x^3+1\right) dx $
\item $\ds \int \sin (x) \sqrt{\cos (x)} dx $
\item $\ds \int \frac{1}{x-5} dx $
\item $\ds \int \frac{7}{3 x+2} dx $
\item $\ds \int \frac{3 x^3+4 x^2+2 x-22}{x^2+3 x+5} dx $
\item $\ds \int \frac{2 x+7}{x^2+7 x+3} dx $
\item $\ds \int \frac{9 (2 x+3)}{3 x^2+9 x+7} dx $
\item $\ds \int \frac{-x^3+14 x^2-46 x-7}{x^2-7 x+1} dx $
\item $\ds \int \frac{x}{x^4+81} dx $
\item $\ds \int \frac{2}{4 x^2+1} dx $
\item $\ds \int \frac{1}{x \sqrt{4 x^2-1}} dx $
\item $\ds \int \frac{1}{\sqrt{16-9 x^2}} dx $

\item $\ds \int \frac{3 x-2}{x^2-2 x+10} dx $
\item $\ds \int \frac{7-2 x}{x^2+12 x+61} dx $
\item $\ds \int \frac{x^2+5 x-2}{x^2-10 x+32} dx $
\item $\ds \int \frac{x^3}{x^2+9} dx $
\item $\ds \int \frac{x^3-x}{x^2+4 x+9} dx $
\item $\ds \int \frac{\sin (x)}{\cos ^2(x)+1} dx $
\item $\ds \int \frac{\cos (x)}{\sin ^2(x)+1} dx $
\item $\ds \int \frac{\cos (x)}{1-\sin ^2(x)} dx $
\item $\ds \int \frac{3 x-3}{\sqrt{x^2-2 x-6}} dx $
\item $\ds \int \frac{x-3}{\sqrt{x^2-6 x+8}} dx $
\end{enumerate}

\noindent{\bf In Exercises 73--80, use Substitution to evaluate the definite integral. }

\begin{enumerate}[1),resume]
\item $\ds \int_1^3 \frac{1}{x-5} dx $
\item $\ds \int_2^6 x\sqrt{x-2} dx $
\item $\ds \int_{-\pi/2}^{\pi/2} \sin^2(x) \cos (x)\ dx $
\item $\ds \int_{0}^{1} 2x(1-x^2)^4\ dx $
\item $\ds \int_{-2}^{-1} (x+1)e^{x^2+2x+1}\ dx $
\item $\ds \int_{-1}^{1} \frac{1}{1+x^2}\ dx $
\item $\ds \int_{2}^{4} \frac{1}{x^2-6x+10}\ dx $
\item $\ds \int_{1}^{\sqrt{3}} \frac{1}{\sqrt{4-x^2}}\ dx $

  \item For the town of Mathland, MI, residential power consumption has shown certain trends over recent years.  Based on data reflecting average usage, engineers at the power company have modeled the town's rate of energy consumption by the function
 $$r(t) = 4 + \sin(0.263t + 4.7) + \cos(0.526t+9.4).$$
Here, $t$ measures time in hours after midnight on a typical weekday, and $r$ is the rate of consumption in megawatts at time $t$. %{\em Note: The unit megawatt is itself a rate, which measures energy consumption per unit time.  A megawatt-hour is the total amount of energy that is equivalent to a constant stream of 1 megawatt of power being sustained for 1 hour.}
Units are critical throughout this problem.
	\ba
		\item Sketch a carefully labeled graph of $r(t)$ on the interval $[0,24]$ and explain its meaning.  Why is this a reasonable model of power consumption?
		\item Without calculating its value, explain the meaning of $\int_0^{24} r(t) \, dt$.   Include appropriate units on your answer.
		
  		\item Determine the exact amount of power Mathland consumes in a typical day.  
		\item What is Mathland's average rate of energy consumption in a given $24$-hour period?  What are the units on this quantity?
	\ea
\end{enumerate}

%---------------------------------------------
% END OF EXERCISES ON SECOND PAGE
%---------------------------------------------
\end{multicols*}
\end{adjustwidth*}

\afterexercises