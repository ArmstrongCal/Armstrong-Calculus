\begin{adjustwidth*}{}{-2.25in}
\textbf{{\large Exercises}}
\setlength{\columnsep}{25pt}
\begin{multicols*}{2}
\noindent Terms and Concepts \small
\begin{enumerate}[1)]
\item {T/F: Let $f$ be a position function. The average rate of change on $[a,b]$ is the slope of the line through the points $(a, f(a))$ and $(b,f(b))$.}
\item {T/F: The definition of the derivative of a function at a point involves taking a limit.}
\item {In your own words, explain the difference between the average rate of change and instantaneous rate of change.}
\item What is the instantaneous rate of change of position called?
\item Given a function $y=f(x)$, in your own words describe how to find the units of $f'(x)$.
\item Let $V(x)$ measure the volume, in decibels, measured inside a restaurant with $x$ customers. What are the units of $V'(x)$?
\item Let $v(t)$ measure the velocity, in ft/s, of a car moving in a straight line $t$ seconds after starting. What are the units of $v'(t)$?
\end{enumerate} 

\noindent {\normalsize Problems} \small

\noindent {\bf In exercises 8--14, a function and an $x$-value $a$ are given.
\begin{enumerate}[a),leftmargin=12pt]
\item Find the derivative of the function at the given point.
\item Find the tangent line to the graph of the function at $a$.
\end{enumerate}}

\begin{enumerate}[1),resume]
\item $\ds f(x) = 6; \quad a = -2$
\item $\ds f(x) = 2x; \quad a = 3$
\item $\ds h(x) = 4 - 3x; \quad a = 7$
\item $\ds g(x) = x^2; \quad a = 2$
\item $\ds f(x) = 3x^2 - x + 4; \quad a = -1$
\item $\ds h(x) = \frac{1}{x}; \quad a = -2$
\item $\ds r(x) = \frac{1}{x-2}; \quad a = 3$
\end{enumerate}

\noindent {\bf In exercises 15--17, use a central difference to estimate the instantaneous rates of change at the indicated values.}
\begin{enumerate}[1),resume]
\item The yearly profits $P(t)$, in millions of dollars, of a certain company from $1990$ to $1996$ are given in the following table.

\scalebox{0.85}{\begin{tabular}{|c|c|c|c|c|c|c|c|}
\hline
Year & $1990$ & $1991$ & $1992$ & $1993$ & $1994$ & $1995$ & $1996$ \\ \hline
Profit & $0.5$ & $1.0$ & $1.2$ & $1.6$ & $2.5$ & $1.6$ & $2.0$ \\ \hline
\end{tabular}}% end scalebox

Estimate $P'(1991)$, $P'(1993)$, and $P'(1995)$.  Include units in your answers.

\item The position, $s(t)$, of an object moving in a straight line at time $t$ is given by the following table.

\begin{center}
\scalebox{0.85}{\begin{tabular}{|c|c|c|c|c|c|}
\hline
$t$ & $0$ & $0.5$ & $1$ & $1.5$ & $2$  \\ \hline
$s(t)$ & $0$ & $30$ & $52$ & $66$ & $72$  \\ \hline
\end{tabular}}% end scalebox
\end{center}

Estimate $s'(0.5)$, $s'(1)$, and $s'(1.5)$.  Include units in your answers.

\item The average price, $p(t)$, for a ticket to a movie theater in North America for selected years is shown in the following table.

 \scalebox{0.85}{\begin{tabular}{|c|c|c|c|c|c|c|c|}
\hline
Year & $1987$ & $1991$ & $1995$ & $1999$ & $2003$ & $2007$ & $2009$ \\ \hline
Price (\$) & $3.91$ & $4.21$ & $4.35$ & $5.06$ & $6.03$ & $6.88$ & $7.50$ \\ \hline
\end{tabular}}% end scalebox

{\footnotesize (Source: National Association of Theater Owners, www.natoonline.org)}

\ba
\item Estimate $p'(1991)$ and $p'(2003)$.  Include units in your answers.
\item Are we able to estimate $p'(2007)$ using a central difference?  Why or why not?
\ea

\item A cup of coffee has its temperature $F$ (in degrees Fahrenheit) at time $t$ given by the function $F(t) = 75 + 110 e^{-0.05t}$, where time is measured in minutes.
\ba
\item Use a central difference with $h = 0.01$ to estimate the value of $F'(10)$.
\item What are the units on the value of $F'(10)$ that you computed in (a)?  What is the practical meaning of the value of $F'(10)$?
\item Which do you expect to be greater: $F'(10)$ or $F'(20)$?  Why?  
\item Write a sentence that describes the behavior of the function $y = F'(t)$ on the time interval $0 \le t \le 30$.  How do you think its graph will look?  Why?
\ea

\item The temperature change $T$ (in Fahrenheit degrees), in a patient, that is generated by a dose $q$ (in milliliters), of a drug, is given by the function $T = f(q)$.
\ba
\item What does it mean to say $f(50) = 0.75$?  Write a complete sentence to explain, using correct units.
\item A person's sensitivity, $s$, to the drug is defined by the function $s(q) = f'(q)$.  What are the units of sensitivity?
\item Suppose that $f'(50) = -0.02$.  Write a complete sentence to explain the meaning of this value.  Include in your response the information given in (a).
\ea

\end{enumerate}

%------------------------------------------
% END OF EXERCISES ON FIRST PAGE
%------------------------------------------
\end{multicols*}
\end{adjustwidth*}

\clearpage

\begin{adjustwidth*}{}{-2.25in}
\setlength{\columnsep}{25pt}
\begin{multicols*}{2}\small

\begin{enumerate}[1),start=20]
\item The velocity of a ball that has been tossed vertically in the air is given by $v(t) = 16 - 32t$, where $v$ is measured in feet per second, and $t$ is measured in seconds.  The ball is in the air from $t = 0$ until $t = 2$.
\ba
\item When is the ball's velocity greatest?
\item Determine the value of $v'(1)$.  Justify your thinking.
\item What are the units on the value of $v'(1)$?  What does this value and the corresponding units tell you about the behavior of the ball at time $t = 1$?  
\item What is the physical meaning of the function $v'(t)$?
\ea

\item The value, $V$, of a particular automobile (in dollars) depends on the number of miles, $m$, the car has been driven, according to the function $V = h(m)$.  
\ba
\item Suppose that $h(40000) = 15500$ and $h(55000) = 13200$.  What is the average rate of change of $h$ on the interval $[40000,55000]$, and what are the units on this value?
\item In addition to the information given in (a), say that $h(70000) = 11100$.  Determine the best possible estimate of $h'(55000)$ and write one sentence to explain the meaning of your result, including units on your answer.
\item Which value do you expect to be greater: $h'(30000)$ or $h'(80000)$?  Why?
\item Write a sentence to describe the long-term behavior of the function $V = h(m)$, plus another sentence to describe the long-term behavior of $h'(m)$.  Provide your discussion in practical terms regarding the value of the car and the rate at which that value is changing.
\ea

\end{enumerate}

%------------------------------------------------
% END OF EXERCISES ON SECOND PAGE
%------------------------------------------------
\end{multicols*}
\end{adjustwidth*}
\afterexercises