\begin{exercises} 

  \item Consider the curve $f(x) = 3 \cos(\frac{x^3}{4})$ and the portion of its graph that lies in the first quadrant between the $y$-axis and the first positive value of $x$ for which $f(x) = 0$.  Let  $R$ denote the region bounded by this portion of $f$, the $x$-axis, and the $y$-axis.  
   \ba
   		\item Set up a definite integral whose value is the exact arc length of $f$ that lies along the upper boundary of $R$.  Use technology appropriately to evaluate the integral you find. 	
		\item Set up a definite integral whose value is the exact area of $R$.  Use technology appropriately to evaluate the integral you find. 	
		\item Suppose that the region $R$ is revolved around the $x$-axis.  Set up a definite integral whose value is the exact volume of the solid of revolution that is generated.   Use technology appropriately to evaluate the integral you find. 	
		\item Suppose instead that $R$ is revolved around the $y$-axis.  If possible, set up an integral expression whose value is the exact volume of the solid of revolution and evaluate the integral using appropriate technology.  If not possible, explain why. 
   \ea

\item Consider the curves given by $y = \sin(x)$ and $y = \cos(x)$.  For each of the following problems, you should include a sketch of the region/solid being considered, as well as a labeled representative slice.
	\ba
		\item Sketch the region $R$ bounded by the $y$-axis and the curves $y = \sin(x)$ and $y = \cos(x)$ up to the first positive value of $x$ at which they intersect. What is the exact intersection point of the curves?
		\item Set up a definite integral whose value is the exact area of $R$.
		\item Set up a definite integral whose value is the exact volume of the solid  of revolution generated by revolving $R$ about the $x$-axis.
		\item Set up a definite integral whose value is the exact volume of the solid  of revolution generated by revolving $R$ about the $y$-axis.
		\item Set up a definite integral whose value is the exact volume of the solid  of revolution generated by revolving $R$ about the line $y = 2$.
		\item Set up a definite integral whose value is the exact volume of the solid  of revolution generated by revolving $R$ about the $x = -1$.
	\ea


    
   \item  Consider the finite region $R$ that is bounded by the curves $y = 1+\frac{1}{2}(x-2)^2$, $y=\frac{1}{2}x^2$, and $x = 0$.
         \ba
		\item Determine a definite integral whose value is the of the region enclosed by the two curves.  
    		\item Find an expression involving one or more definite integrals whose value is the volume of the solid of revolution generated by revolving the region $R$ about the line $y = -1$. 
		\item Determine an expression involving one or more definite integrals whose value is the volume of the solid of revolution generated by revolving the region $R$ about the $y$-axis.  
		\item Find an expression involving one or more definite integrals whose value is the perimeter of the region $R$.
         \ea
   
\end{exercises}
\afterexercises
