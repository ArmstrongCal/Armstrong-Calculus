\begin{exercises} 
  \item  Newton's Law of Cooling says that the rate at which an
    object, such as a cup of coffee, cools is proportional to the
    difference in the object's temperature and room temperature.  If
    $T(t)$ is the object's temperature and $T_r$ is room temperature,
    this law is expressed at
    $$
    \frac{dT}{dt} = -k(T-T_r),
    $$
    where $k$ is a constant of proportionality.  In this problem,
    temperature is 
    measured in degrees Fahrenheit and time in minutes.

    \ba
    \item  Two calculus students, Alice and Bob, enter a 70$^\circ$
    classroom at the same time.  Each has a cup of coffee that is
    100$^\circ$.  The differential equation for Alice has a constant
    of proportionality $k=0.5$, while the constant of proportionality
    for Bob is $k=0.1$.

    What is the initial rate of change for Alice's coffee? 
    What is the initial rate of change for Bob's coffee?

    \item  What feature of Alice's and Bob's cups of coffee could explain
    this difference?

 \item  As the heating unit turns on and off in the room, the
    temperature in the room is $$T_r=70+10\sin t.$$  Implement Euler's
    method with a step size of $\Delta t = 0.1$ to approximate the
    temperature of Alice's coffee over the time interval $0\leq t\leq
    50$.  This will most easily be performed using a spreadsheet such
    as \emph{Excel}.  Graph the temperature of her coffee and room
    temperature over this interval.

   \item  In the same way, implement Euler's method to approximate the
    temperature of Bob's
    coffee over the same time interval.  Graph the temperature of his
    coffee and room temperature over the interval.

    \item  Explain the similarities and differences that you see in the
    behavior of Alice's and Bob's cups of coffee.
\ea

  \item We have seen that the error in approximating the solution to
    an initial value problem is proportional to $\Delta t$.  That is,
    if $E_{\Delta t}$ is the Euler's method approximation to the
    solution to an initial value problem at $\overline{t}$, then
    $$y(\overline{t})-E_{\Delta t} \approx K\Delta t
    $$
    for some constant of proportionality $K$.

    In this problem, we will see how to use this fact to improve our
    estimates, using an idea called {\em accelerated convergence}.

    \ba
    \item  We will create a new approximation by assuming the error is
    {\em exactly} proportional to $\Delta t$, according to the formula
    $$y(\overline{t})-E_{\Delta t} =K\Delta t.
    $$
    
    Using our earlier results from the initial value problem $dy/dt = y$ and
    $y(0)=1$ with $\Delta t = 0.2$ and $\Delta t = 0.1$, we have
    \begin{eqnarray*}
      y(1) - 2.4883 & = & 0.2K \\
      y(1) - 2.5937 & = & 0.1K. \\
    \end{eqnarray*}

    This is a system of two linear equations in the unknowns $y(1)$
    and $K$.  Solve this system to find a new approximation for
    $y(1)$.  (You may remember that the exact value is $y(1) = e =
    2.71828\ldots.$)

    \item Use the other data, $E_{0.05} = 2.6533$ and $E_{0.025} = 2.6851$
    to do similar work as in (a) to obtain another approximation.  Which gives the better
    approximation?  Why do you think this is?

    \item  Let's now study the initial value problem
    $$
      \frac{dy}{dt} = t-y, \ y(0) = 0.
    $$
    Approximate $y(0.3)$ by applying Euler's method to find
    approximations $E_{0.1}$ and $E_{0.05}$.  Now use the idea of
    accelerated convergence to obtain a better approximation.  (For the sake of comparison, you want to  note that the actual value is $y(0.3) =
    0.0408$.) 

\ea

  \item  In this problem, we'll modify Euler's method to obtain better
    approximations to solutions of initial value problems.  This
    method is called the {\em Improved Euler's method.}

    In Euler's method, we walk across an interval of width $\Delta t$
    using the slope obtained from the differential equation at the
    left endpoint of the interval.  Of course, the slope of the
    solution will most likely change over this interval.  We can
    improve our approximation by trying to incorporate the change in
    the slope over the interval.

    Let's again consider the initial value problem $dy/dt = y$ and
    $y(0) = 1$, which we will approximate using steps of width $\Delta
    t = 0.2$.  Our first interval is therefore $0\leq t \leq 0.2$.  At
    $t=0$, the differential equation tells us that the slope is 1, and
    the approximation we obtain from Euler's method is that
    $y(0.2)\approx y_1= 1+ 1(0.2)= 1.2$.  

    This gives us some idea for how the slope has changed over the
    interval $0\leq t\leq 0.2$.  We know the slope at $t=0$ is 1,
    while the slope at $t=0.2$ is 1.2, trusting in the Euler's method
    approximation.  We will therefore refine our estimate of the
    initial slope to be the average of these two slopes;  that is, we
    will estimate the slope to be $(1+1.2)/2 = 1.1$.  This gives the
    new approximation $y(1) = y_1 = 1 + 1.1(0.2) = 1.22$.  

    The first few steps look like this:
    \begin{center}
      \begin{tabular}{|c|c|c|c|}
        \hline
        $t_i$ & $y_i$ & Slope at $(t_{i+1},y_{i+1})$ & Average slope \\
        \hline
        0.0&1.0000&1.2000&1.1000\\
        0.2&1.2200&1.4640&1.3420\\
        0.4&1.4884&1.7861&1.6372\\
        \vdots &  \vdots &  \vdots &  \vdots \\
        \hline
      \end{tabular}
    \end{center}

\ba
	\item Continue with this method to obtain an approximation for $y(1) =
    e$. 

    \item  Repeat this method with $\Delta t = 0.1$ to obtain a better
    approximation for $y(1)$.

\item   We saw that the error in Euler's method is proportional to
    $\Delta t$.  Using your results from parts (a) and (b), what power
    of $\Delta t$ appears to be proportional to the error in the
    Improved Euler's Method?
        \ea
        
\end{exercises}
\afterexercises
