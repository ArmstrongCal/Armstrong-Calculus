\begin{adjustwidth*}{}{-2.25in}
\textbf{{\large Exercises}}
\setlength{\columnsep}{25pt}
\begin{multicols*}{2}
\noindent {\normalsize Problems} \small

\begin{enumerate}[1)]
  \item For each of the following integrals involving rational functions, (1) use a CAS to find the partial fraction decomposition of the integrand; (2) evaluate the integral of the resulting function without the assistance of technology; (3) use a CAS to evaluate the original integral to test and compare your result in (2).
  \ba
  	\item $\ds \int \frac{x^3 + x + 1}{x^4 - 1} \, dx$
	\item $\ds \int \frac{x^5 + x^2 + 3}{x^3 - 6x^2 + 11x - 6} \, dx$
	\item $\ds \int \frac{x^2 - x - 1}{(x-3)^3} \, dx$
  \ea

  \item For each of the following integrals involving radical functions, (1) use an appropriate $u$-substitution along with Appendix~\ref{C:9.IntegralTable} to evaluate the integral without the assistance of technology, and (2) use a CAS to evaluate the original integral to test and compare your result in (1).
  \ba
  	\item $\ds \int \frac{1}{x \sqrt{9x^2 + 25}} \, dx$
	\item $\ds \int x \sqrt{1 + x^4} \, dx$
	\item $\ds \int  e^x \sqrt{4 + e^{2x}} \, dx$
	\item $\ds \int \frac{\tan(x)}{\sqrt{9 - \cos^2(x)}}  \, dx$
  \ea

  
  \item Consider the indefinite integral given by
   $$\int \frac{\sqrt{x+\sqrt{1+x^2}}}{x} \, dx.$$
  	\ba
		\item Explain why $u$-substitution does not offer a way to simplify this integral by discussing at least two different options you might try for $u$.
		\item Explain why integration by parts does not seem to be a reasonable way to proceed, either, by considering one option for $u$ and $dv$.
		\item Is there any line in the integral table in Appendix~\ref{C:9.IntegralTable} that is helpful for this integral?
		\item Evaluate the given integral using \emph{WolframAlpha}.  What do you observe?
	\ea
\end{enumerate}

%------------------------------------------
% END OF EXERCISES ON FIRST PAGE
%------------------------------------------
\end{multicols*}
\end{adjustwidth*}

%\clearpage
%
%\begin{adjustwidth*}{}{-2.25in}
%\setlength{\columnsep}{25pt}
%\begin{multicols*}{2}\small
%
%\begin{enumerate}[1),start=27]
%  \item For an unknown function $f(x)$, the following information is known.  
%  \begin{itemize}
%  	\item $f$ is continuous on $[3,6]$;
%	\item $f$ is either always increasing or always decreasing on $[3,6]$;
%	\item $f$ has the same concavity throughout the interval $[3,6]$;
%	\item As approximations to $\int_3^6 f(x) \, dx$, $L_4 = 7.23$, $R_4 = 6.75$, and $M_4 = 7.05$.
%  \end{itemize}
%  \ba
%  	\item Is $f$ increasing or decreasing on $[3,6]$?  What data tells you?
%	\item Is $f$ concave up or concave down on $[3,6]$?  Why?
%	\item Determine the best possible estimate you can for $\int_3^6 f(x) \, dx$, based on the given information.
%  \ea
%  
%    \item The rate at which water flows through Table Rock Dam on the White River in Branson, MO, is measured in thousands of cubic feet per second (TCFS).  As engineers open the floodgates, flow rates are recorded according to the following chart.
%  \begin{center}
%\begin{tabular}{|l|c|c|c|c|c|c|c|}
%\hline
%seconds, $t$ & 0 & 10 & 20 & 30 & 40 & 50 & 60 \\
%\hline
%flow in TCFS, $r(t)$ & 2000 & 2100 & 2400 & 3000 & 3900 & 5100 & 6500 \\
%\hline
%\end{tabular}
%\end{center}
%	\ba
%		\item What definite integral measures the total volume of water to flow through the dam in the 60 second time period provided by the table above?
%		\item Use the given data to calculate $M_n$ for the largest possible value of $n$ to approximate the integral you stated in (a).  Do you think $M_n$ over- or under-estimates the exact value of the integral?  Why?
%		
%		\item Approximate the integral stated in (a) by calculating $S_n$ for the largest possible value of $n$, based on the given data.
%		
%		\item Compute $\frac{1}{60} S_n$ and $\frac{2000+2100+2400+3000+3900+5100+6500}{7}$.  What quantity do both of these values estimate?  Which is a more accurate approximation?
%	\ea
%\end{enumerate}
%
%%---------------------------------------------
%% END OF EXERCISES ON SECOND PAGE
%%---------------------------------------------
%\end{multicols*}
%\end{adjustwidth*}

\afterexercises