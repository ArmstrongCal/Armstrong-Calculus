\begin{adjustwidth*}{}{-2.25in}
\textbf{{\large Exercises}}
\setlength{\columnsep}{25pt}
\begin{multicols*}{2}
\noindent Terms and Concepts \small
\begin{enumerate}[1)]
\item T/F: The integral formula for computing Arc Length was found by first approximating arc length with straight line segments.
\item T/F: The integral formula for computing Arc Length includes a square--root, meaning the integration is probably easy.
\end{enumerate} 

\noindent {\normalsize Problems} \small

\noindent{\bf In Exercises 3--11, find the arc length of the function on the given integral.}

\begin{enumerate}[1),resume]
\item $\ds f(x) = x$ on $[0, 1]$
\item $\ds f(x) = \sqrt{8x}$ on $[-1, 1]$
\item $\ds f(x) = \frac13x^{3/2}-x^{1/2}$ on $[0,1]$
\item $\ds f(x) = \frac1{12}x^{3}+\frac1x$ on $[1,4]$
\item $\ds f(x) = 2x^{3/2}-\frac16\sqrt{x}$ on $[0,9]$
\item $\ds f(x) = \frac12\big(e^x+e^{-x}\big)$ on $[0, \ln 5]$
\item $\ds f(x) = \frac1{12}x^5+\frac1{5x^3}$ on $[.1, 1]$
\item $\ds f(x) = \ln \big(\sin x\big)$ on $[\pi/6, \pi/2]$
\item $\ds f(x) = \ln \big(\cos x\big)$ on $[0, \pi/4]$
\end{enumerate}

\noindent{\bf In Exercises 12--19, set up the integral to compute the arc length of the function on the given interval. Try to compute the integral by hand, and use a CAS to compute the integral.  Also, use Simpson's Rule with $n = 4$ to approximate the arc length.}

\begin{enumerate}[1),resume]
\item $\ds f(x) = x^2$ on $[0, 1]$.\label{ex_07_04_ex_13}
\item $\ds f(x) = x^{10}$ on $[0, 1]$
\item $\ds f(x) = \sqrt{x}$ on $[0, 1]$
\item $\ds f(x) = \ln x$ on $[1, e]$
\item $\ds f(x) = \sqrt{1-x^2}$ on $[-1, 1]$. (Note: this describes the top half of a circle with radius 1.)
\item $\ds f(x) = \sqrt{1-x^2/9}$ on $[-3, 3]$. (Note: this describes the top half of an ellipse with a major axis of length 6 and a minor axis of length 2.)
\item $\ds f(x) = \frac1x$ on $[1,2]$
\item $\ds f(x) = \sec x$ on $[-\pi/4,\pi/4]$.\label{ex_07_04_ex_20}
\end{enumerate}

\columnbreak

\noindent{\bf In Exercises 20--24, find the surface area of the described solid of revolution.}

\begin{enumerate}[1),resume]
\item The solid formed by revolving $y=2x$ on $[0,1]$ about the $x$-axis.
\item The solid formed by revolving $y=x^2$ on $[0,1]$ about the $y$-axis.
\item The solid formed by revolving $y=x^3$ on $[0,1]$ about the $x$-axis.
\item The solid formed by revolving $y=\sqrt{x}$ on $[0,1]$ about the $x$-axis.
\item The sphere formed by revolving $y=\sqrt{1-x^2}$ on $[-1,1]$ about the $x$-axis.
\end{enumerate}

%------------------------------------------
% END OF EXERCISES ON FIRST PAGE
%------------------------------------------
\end{multicols*}
\end{adjustwidth*}

%\clearpage
%
%\begin{adjustwidth*}{}{-2.25in}
%\setlength{\columnsep}{25pt}
%\begin{multicols*}{2}\small
%
%\begin{enumerate}[1),start=12]
%
%\end{enumerate}
%
%%---------------------------------------------
%% END OF EXERCISES ON SECOND PAGE
%%---------------------------------------------
%\end{multicols*}
%\end{adjustwidth*}

\afterexercises