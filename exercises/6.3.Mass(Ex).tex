\begin{exercises} 
  \item Let a thin rod of length $a$ have density distribution function $\rho(x) = 10e^{-0.1x}$, where $x$ is measured in cm and $\rho$ in grams per centimeter.
  \ba
  	\item If the mass of the rod is 30 g, what is the value of $a$?
	\item For the 30g rod, will the center of mass lie at its midpoint, to the left of the midpoint, or to the right of the midpoint?  Why?
	\item For the 30g rod, find the center of mass, and compare your prediction in (b).
	\item At what value of $x$ should the 30g rod be cut in order to form two pieces of equal mass? 
  \ea
  
  \item Consider two thin bars of constant cross-sectional area, each of length 10 cm, with respective mass density functions $\rho(x) = \frac{1}{1+x^2}$ and $p(x) = e^{-0.1x}$.
  \ba
  	\item Find the mass of each bar.
	\item Find the center of mass of each bar.
	\item Now consider a new 10 cm bar whose mass density function is $f(x) = \rho(x) + p(x)$.  
	\be
		\item[i.]  Explain how you can easily find the mass of this new bar with little to no additional work.
		\item[ii.]  Similarly, compute $\int_0^{10} xf(x) \, dx$ as simply as possible, in light of earlier computations.
		\item[iii.]  True or false:  the center of mass of this new bar is the average of the centers of mass of the two earlier bars.  Write at least one sentence to say why your conclusion makes sense.
	\ee
  \ea
  
  \item Consider the curve given by $y = f(x) = 2xe^{-1.25x} + (30-x) e^{-0.25(30-x)}$.
  \ba
  	\item Plot this curve in the window $x = 0 \ldots 30$, $y = 0 \ldots 3$ (with constrained scaling so the units on the $x$ and $y$ axis are equal), and use it to generate a solid of revolution about the $x$-axis.  Explain why this curve could generate a reasonable model of a baseball bat.
	\item Let $x$ and $y$ be measured in inches.  Find the total volume of the baseball bat generated by revolving the given curve about the $x$-axis.  Include units on your answer
	\item Suppose that the baseball bat has constant weight density, and that the weight density is 0.6 ounces per cubic inch.  Find the total weight of the bat whose volume you found in (b).
	\item Because the baseball bat does not have constant cross-sectional area, we see that the amount of weight concentrated at a location $x$ along the bat is determined by the volume of a slice at location $x$.  Explain why we can think about the function $\rho(x) = 0.6 \pi f(x)^2$ (where $f$ is the function given at the start of the problem) as being the weight density function for how the weight of the baseball bat is distributed from $x = 0$ to $x = 30$.
	\item Compute the center of mass of the baseball bat.  
  \ea
  
\end{exercises}
\afterexercises
