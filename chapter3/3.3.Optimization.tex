\section{Global Optimization} \label{S:3.3.Optimization}

\begin{goals}
\item What are the differences between finding relative extreme values and global extreme values of a function?
\item How is the process of finding the global maximum or minimum of a function over the function's entire domain different from determining the global maximum or minimum on a restricted domain?
\item For a function that is guaranteed to have both a global maximum and global minimum on a closed, bounded interval, what are the possible points at which these extreme values occur?
\end{goals}

%--------------------------------------
% SUBSECTION INTRODUCTION
%--------------------------------------
\subsection*{Introduction}

We have seen that we can use the first derivative of a function to determine where the function is increasing or decreasing, and the second derivative to know where the function is concave up or concave down.  Each of these approaches provides us with key information that helps us determine the overall shape and behavior of the graph, as well as whether the function has a relative minimum or relative maximum at a given critical value.  Remember that the difference between a relative maximum and a global maximum is that there is a relative maximum of $f$ at $x = p$ if $f(p) \ge f(x)$ for all $x$ near $p$, while there is a global maximum at $p$ if $f(p) \ge f(x)$ for all $x$ in the domain of $f$.  

\begin{marginfigure}[3cm] % MARGIN FIGURE
\margingraphics{figures/3_3_Intro.eps}
\caption{A function $f$ with a global maximum, but no global minimum.} \label{F:3.3.Intro}
\end{marginfigure}

For instance, in Figure~\ref{F:3.3.Intro}, we see a function $f$ that has a global maximum at $x = c$ and a relative maximum at $x = a$, since $f(c)$ is greater than $f(x)$ for every value of $x$, while $f(a)$ is only greater than the value of $f(x)$ for $x$ near $a$.  Since the function appears to decrease without bound, $f$ has no global minimum, though clearly $f$ has a relative minimum at $x = b$.

Our emphasis in this section is on finding the global extreme values of a function (if they exist).  In so doing, we will either be interested in the behavior of the function over its entire domain or on some restricted portion.  The former situation is familiar and similar to work that we did in the two preceding sections of the text.  We explore this through a particular example in the following preview activity.

\begin{pa} \label{PA:3.3}
Let $\ds f(x) = 2 + \frac{3}{1+(x+1)^2}$.
\ba
	\item Determine all of the critical values of $f$.
	\item Construct a first derivative sign chart for $f$ and thus determine all intervals on which $f$ is increasing or decreasing.
	\item Does $f$ have a global maximum?  If so, why, and what is its value and where is the maximum attained?  If not, explain why.
	\item Determine $\ds \lim_{x \to \infty} f(x)$ and $\ds \lim_{x \to -\infty} f(x)$.
	\item Explain why $f(x) > 2$ for every value of $x$.
	\item Does $f$ have a global minimum?  If so, why, and what is its value and where is the minimum attained?  If not, explain why.
\ea
\end{pa} \afterpa %PREVIEW

%------------------------------------------------
% SUBSECTION GLOBAL OPTIMIZATION
%-----------------------------------------------
\subsection*{Global Optimization}

For the functions in Figure~\ref{F:3.3.Intro} and Preview Activity~\ref{PA:3.3}, we were interested in finding the global minimum and global maximum on the entire domain, which turned out to be $(-\infty, \infty)$ for each.  At other times, our perspective on a function might be more focused due to some restriction on its domain.  For example, rather than considering $f(x) = 2 + \frac{3}{1+(x+1)^2}$ for every value of $x$,  perhaps instead we are only interested in those $x$ for which $0 \le x \le 4$, and we would like to know which values of $x$ in the interval $[0,4]$ produce the largest possible and smallest possible values of $f$.  We are accustomed to critical values playing a key role in determining the location of extreme values of a function; now, by restricting the domain to an interval, it makes sense that the endpoints of the interval will also be important to consider, as we see in the following activity.  When limiting ourselves to a particular interval, we will often refer to the \emph{absolute} maximum or minimum value, rather than the \emph{global} maximum or minimum.

\begin{activity} \label{A:3.3.1}  Let $g(x) = \frac{1}{3}x^3 - 2x + 2.$
	\ba
	\item Find all critical values of $g$ that lie in the interval $-2 \le x \le 3$.
	\item Use a graphing utility to construct the graph of $g$ on the interval $-2 \le x \le 3$.
	\item From the graph, determine the $x$-values at which the absolute minimum and absolute maximum of $g$ occur on the interval $[-2,3]$.
	\item How do your answers change if we instead consider the interval $-2 \le x \le 2$?
	\item What if we instead consider the interval $-2 \le x \le 1$?
	\ea
\end{activity}
\begin{smallhint}
	\ba
	\item Check that each critical value you find satisfies $-2 \le x \le 3$.
	\item \href{http://wolframalpha.com}{\texttt{http://wolframalfpha.com}} is a great choice.
	\item On the graph, look for the lowest and highest possible values of the function.
	\item Ask yourself the same questions as (a)-(c), simply using the new interval.
	\ea
\end{smallhint}
\begin{bighint}
	\ba
	\item Check that each critical value you find satisfies $-2 \le x \le 3$.
	\item \href{http://wolframalpha.com}{\texttt{http://wolframalfpha.com}} is a great choice.
	\item On the graph, look for the lowest and highest possible values of the function.
	\item Ask yourself the same questions as (a)-(c), simply using the new interval.
	\ea
\end{bighint}
\begin{activitySolution}
	\ba
	\item Since $g'(x) = x^2 - 2$, the critical values of $g$ are $x = \pm \sqrt{2} \approx \pm 1.414$, both of which lie in the interval $-2 \le x \le 3$.
	\item The figure shown below shows three related plots, each with the emphases on the interval provided in (c), (d), and (e).
	\begin{center}
	\includegraphics{figures/3_3_Act1Soln.eps}
	\end{center}
	\item On $[-2,3]$, $g$ has a global maximum at $x = 3$ and a global minimum at $x = \sqrt{2}$.
	\item On $[-2,2]$, $g$ has a global maximum at $x = -\sqrt{2}$ and a global minimum at $x = \sqrt{2}$.
	\item On $[-2,3]$, $g$ has a global maximum at $x = -\sqrt{2}$ and a global minimum at $x = 1$.
	\ea
\end{activitySolution}
\aftera %ACTIVITY

In Activity~\ref{A:3.3.1}, we saw how the absolute maximum and absolute minimum of a function on a closed, bounded interval $[a,b]$, depend not only on the critical values of the function, but also on the selected values of $a$ and $b$.  These observations demonstrate several important facts that hold much more generally.  First, we state an important result called the Extreme Value Theorem.

\concept{The Extreme Value Theorem\index{extreme value theorem}} % CONCEPT
{If $f$ is a continuous function on a closed interval $[a,b]$, then $f$ attains both an absolute minimum and absolute maximum on $[a,b]$.  That is, for some value $x_m$ such that $a \le x_m \le b$, it follows that $f(x_m) \le f(x)$ for all $x$ in $[a,b]$.  Similarly, there is a value $x_M$ in $[a,b]$ such that $f(x_M) \geq f(x)$ for all $x$ in $[a,b]$.  Letting $m = f(x_m)$ and $M = f(x_M)$, it follows that $m \le f(x) \le M$ for all $x$ in $[a,b]$.
} % end concept

The Extreme Value Theorem tells us that provided a function is continuous on any closed interval, $[a,b]$, the function has to achieve both an absolute minimum and an absolute maximum.  Note, however, that this result does not tell us where these extreme values occur, but rather only that they must exist.  As seen in the examples of Activity~\ref{A:3.3.1}, it is apparent that the only possible locations for absolute extremes are either the endpoints of the interval or at a critical value (the latter being where a relative miniumum or maximum could occur, which is a potential location for an absolute extreme).  Thus, we have the following approach to finding the absolute maximum and minimum of a continuous function $f$ on the interval $[a,b]$:
\begin{itemize}
	\item find all critical values of $f$ that lie in the interval;
	\item evaluate the function $f$ at each critical value in the interval and at each endpoint of the interval;
	\item from among the noted function values, the smallest is the absolute minimum of $f$ on the interval, while the largest is the absolute maximum.
\end{itemize}

\begin{marginfigure}
\margingraphics{figures/figextval4} %APEX ex78 
\caption{A graph of $f(x) = 2x^3+3x^2-12x$ on $[0,3]$ as in Example \ref{Ex:3.3.Eg2}. } \label{F:3.3.Ex2}
\end{marginfigure}

\begin{margintable}
\begin{center}
\scalebox{1.25}{
\begin{tabular}{cc} 
$x$ & $f(x)$ \\ \hline \rule{0pt}{10pt}
 $0$ & $0$ \\ 
 $1$ & $-7$\\
 $3$ & $45$ 
\end{tabular}
}% end scalebox
\end{center}
\caption{Finding the extreme values of $f$ in Example \ref{Ex:3.3.Eg2}.} \label{T:3.3.Ex2}
\end{margintable}

\begin{example} \label{Ex:3.3.Eg2}
Find the extreme values of $f(x) = 2x^3+3x^2-12x$ on $[0,3]$, graphed in Figure~\ref{F:3.3.Ex2}.

\solution
We follow the steps outlined. We first evaluate $f$ at the endpoints: $$f(0) = 0 \quad \text{and}\quad f(3) =45.$$

Next, we find the critical values of $f$ on $[0,3]$. $\fp(x) = 6x^2+6x-12 = 6(x+2)(x-1)$; therefore the critical values of $f$ are $x=-2$ and $x=1$. Since $x=-2$ does not lie in the interval $[0,3]$, we ignore it. Evaluating $f$ at the only critical number in our interval gives: $f(1) = -7$. 

The table in Table~\ref{T:3.3.Ex2} gives $f$ evaluated at the ``important'' $x$ values in $[0,3]$. We can easily see the maximum and minimum values of $f$: the maximum value is $45$ and the minimum value is $-7$.


\end{example} % EXAMPLE

\begin{example} \label{Ex:3.3.Eg3}
Find the maximum and minimum values of $f$ on $[-4,2]$, where $$f(x) = \left\{\begin{array}{cc} (x-1)^2 & x\leq 0 \\ x+1 & x>0 \end{array}\right. .$$

\solution
Here $f$ is piecewise--defined, but we can still apply our approach. Evaluating $f$ at the endpoints gives: 
$$ f(-4) = 25 \quad \text{and} \quad f(2) = 3.$$

We now find the critical numbers of $f$. We have to define $\fp$ in a piecewise manner; it is $$\fp(x) =\left\{\begin{array}{cc} 2(x-1) & x < 0 \\ 1 & x>0 \end{array}\right. .$$ Note that while $f$ is defined for all of $[-4,2]$, $\fp$ is not, as the derivative of $f$ does not exist when $x=0$. (From the left, the derivative approaches $-2$; from the right the derivative is $1$.) Thus one critical number of $f$ is $x=0$.

We now set $\fp(x) = 0$. When $x >0$, $\fp(x)$ is never $0$.  When $x<0$, $\fp(x)$ is also never $0$. (We may be tempted to say that $\fp(x) = 0 $ when $x=1$. However, this is nonsensical, for we only consider $\fp(x) = 2(x-1)$ when $x<0$, so we will ignore a solution that says $x=1$. 

So we have three important $x$ values to consider: $x= -4, 2$ and $0$. Evaluating $f$ at each gives, respectively, $25$, $3$ and $1$, shown in Table~\ref{T:3.3.Ex3}. Thus the absolute minimum of $f$ is $1$; the absolute maximum of $f$ is $25$. Our answer is confirmed by the graph of $f$ in Figure \ref{F:3.3.Ex3}.
\end{example}

\begin{marginfigure}[-18cm]
\margingraphics{figures/figextval5} %APEX ex79 
\caption{A graph of $f(x)$ on $[-4,2]$ as in Example \ref{Ex:3.3.Eg3}. } \label{F:3.3.Ex3}
\end{marginfigure}

\begin{margintable}[-6cm]
\begin{center}
\scalebox{1.25}{
\begin{tabular}{cc} 
$x$ & $f(x)$ \\ \hline \rule{0pt}{10pt} 
$-4$ & $25$ \\ 
$0$ & $1$ \\ 
$2$ & $3$
\end{tabular}
} % end scalebox
\end{center}
\caption{Finding the extreme values of $f$ in Example \ref{Ex:3.3.Eg3}.} \label{T:3.3.Ex3}
\end{margintable} % EXAMPLE

\begin{marginfigure}[-1cm]
\margingraphics{figures/figextval6} %APEX ex80 
\caption{A graph of $f(x)=\cos(x^2)$ on $[-2,2]$ as in Example \ref{Ex:3.3.Eg4}. } \label{F:3.3.Ex4}
\end{marginfigure}

\begin{margintable}
\begin{center}
\scalebox{1.25}{
\begin{tabular}{cc} 
$x$ & $f(x)$ \\ \hline \rule{0pt}{10pt} 
$-2$ & $-0.65$ \\ 
$-\sqrt{\pi}$ & $-1$ \\
$0$ & $1$\\
$\sqrt{\pi}$ & $-1$ \\
$2$ & $-0.65$
\end{tabular}
} % end scalebox
\end{center}
\caption{Finding the extreme values of $f(x)= \cos (x^2)$ in Example \ref{Ex:3.3.Eg4}.} \label{T:3.3.Ex4}
\end{margintable}

\begin{example} \label{Ex:3.3.Eg4}
Find the extrema of  $f(x) = \cos (x^2)$ on $[-2,2]$.

\solution
We again our approach to find extrema. Evaluating $f$ at the endpoints of the interval gives: $f(-2) = f(2) = \cos (4) \approx -0.6536.$ We now find the critical values of $f$.

Applying the Chain Rule, we find $\fp(x) = -2x\sin (x^2)$. Set $\fp(x) = 0$ and solve for $x$ to find the critical values of $f$. 

We have $\fp(x) = 0$ when $x = 0$ and when $\sin (x^2) = 0$. In general, $\sin(t) = 0$ when $t = \ldots -2\pi, -\pi, 0, \pi, \ldots$ Thus $\sin (x^2) = 0$ when $x^2 = 0, \pi, 2\pi, \ldots$ ($x^2$ is always positive so we ignore $-\pi$, etc.) So $\sin (x^2)=0$ when $x= 0, \pm \sqrt{\pi}, \pm\sqrt{2\pi}, \ldots$. The only values to fall in the given interval of $[-2,2]$ are $-\sqrt{\pi}$ and $\sqrt{\pi}$, approximately $\pm 1.77$.

We again construct a table of important values in Table~\ref{T:3.3.Ex4}. In this example we have $5$ values to consider: $x= 0, \pm 2, \pm\sqrt{\pi}$. 

From the table it is clear that the maximum value of $f$ on $[-2,2]$ is $1$; the minimum value is $-1$. The graph in Figure \ref{F:3.3.Ex4} confirms our results.
\end{example} % EXAMPLE

\begin{activity} \label{A:3.3.2}  Find the \emph{exact} absolute maximum and minimum of each function on the stated interval.
	\ba
	\item $h(x) = xe^{-x}$, $[0,3]$
	\item $p(t) = \sin(t) + \cos(t)$, $[-\frac{\pi}{2}, \frac{\pi}{2}]$
	\item $q(x) = \frac{x^2}{x-2}$, $[3,7]$
	\item $f(x) = 4 - e^{-(x-2)^2}$, $(-\infty, \infty)$
	\item $h(x) =  xe^{-ax}$, $[0, \frac{2}{a}]$ ($a > 0$)
	\item $f(x) = b - e^{-(x-a)^2}$, $(-\infty, \infty)$, $a, b > 0$
	\ea
\end{activity}
\begin{smallhint}
\ba
	\item After computing $h'(x)$, factor to write the derivative as a product.
	\item The sine and cosine functions have the same value at $\frac{\pi}{4} \pm k\pi$ for any integer $k$.
	\item Upon finding $q'(x)$, factor its numerator.
	\item Remember that $e^{-(x-2)^2}$ is never zero.
\ea
\end{smallhint}
\begin{bighint}
\ba
	\item After computing $h'(x)$, factor to write the derivative as a product.  Remember that $e^{-x} \ne 0$ for all $x$.
	\item The sine and cosine functions have the same value at $\frac{\pi}{4} \pm k\pi$ for any integer $k$.  Which of these occur within the given interval?
	\item Upon finding $q'(x)$, factor its numerator.  Observe that even though $q'(2)$ is not defined, $2$ is not a critical number because $q(2)$ is not defined; moreover, $2$ is not in the interval under consideration.
	\item Remember that $e^{-(x-2)^2}$ is never zero.  Note that the domain being considered is all real numbers.
\ea
\end{bighint}
\begin{activitySolution}
	\ba
	\item For $h(x) = xe^{-x}$, we know that $h'(x) = xe^{-x}(-1) + e^{-x} = e^{-x}(-x+1)$.  Therefore, the only critical value of $h$ is $x = 1$.  Next, we compute $h(1)$, $h(0)$, and $h(3)$.  Observe that 
	\begin{itemize}
	\item	$h(1) = e^{-1} \approx 0.36788$
	\item  $h(0) = 0$
	\item  $h(3) = 3e^{-3} \approx 0.14936$
	\end{itemize}
	Thus, on $[0,3]$, the absolute maximum of $h$ is $e^{-1}$ and the absolute minimum is $0$.
	\item Given $p(t) = \sin(t) + \cos(t)$, it follows $p'(t) = \cos(t) - \sin(t)$, so $p'(t) = 0$ implies that $\cos(t) =\sin(t)$.   The sine and cosine functions have the same value at $\frac{\pi}{4} \pm k\pi$ for any integer $k$.  The only time this occurs in $[-\frac{\pi}{2}, \frac{\pi}{2}]$ is for $x = \frac{\pi}{4}$, and thus this is the only critical value of $p$ in the given interval.  Now,
	\begin{itemize}
	\item	$p(\frac{\pi}{4}) = \sin(\frac{\pi}{4}) + \cos(\frac{\pi}{4}) = \frac{\sqrt{2}}{2} + \frac{\sqrt{2}}{2} = \sqrt{2} \approx 1.41421$
	\item $p(-\frac{\pi}{2}) = \sin(-\frac{\pi}{2}) + \cos(-\frac{\pi}{2}) = -1 + 0 = -1$
	\item  $p(\frac{\pi}{2}) = \sin(\frac{\pi}{2}) + \cos(\frac{\pi}{2}) = 1 + 0 = 1$
	\end{itemize}	
	Therefore, on $[-\frac{\pi}{2},\frac{\pi}{2}]$, the absolute maximum of $p$ is $\sqrt{2}$ and the absolute minimum is $-1$. 
	\item With $q(x) = \frac{x^2}{x-2}$, we have 
	$$q'(x) = \frac{(x-2)(2x) - x^2(1)}{(x-2)^2} = \frac{2x^2 - 4x - x^2}{(x-2)^2} = \frac{x^2-4x}{(x-2)^2} = \frac{x(x-4)}{(x-2)^2}.$$
	Hence, the critical values of $q$ are $x = 0$ and $x = 4$.  Only the latter critical value lies in the interval $[3,7]$, and thus we evaluate $q$ and find
	\begin{itemize}
	\item $q(4) = \frac{16}{2} = 8$
	\item $q(3) = \frac{9}{1} = 9$
	\item $q(7) = \frac{49}{5} = 9.8$
	\end{itemize} 
	We now see that on $[3,7]$ the absolute maximum of $q$ is 9.8 and the absolute minimum is 8.
	\item Here, we first observe that we are working on the domain of all real numbers, not a closed bounded interval.  Hence, we need to think about the overall behavior of the function.  First, since $f(x) = 4 - e^{-(x-2)^2}$, by the chain rule we see that $f'(x) = -e^{-(x-2)^2}(-2(x-2)) = 2(x-2)e^{-(x-2)^2}.$  Since $e^{-(x-2)^2}$ is always positive (in particular, never zero), it follows that the only critical value of $f$ is $x = 2$.  Furthermore, with $f'(x) = 2(x-2)e^{-(x-2)^2}$, we see that for $x < 2$, $f'(x) < 0$, while for $x > 2$, $f'(x) > 0$.  This tells us by the first derivative test that $f$ is decreasing for $x < 2$ and increasing for $x > 2$, which tells us that $f$ has an absolute minimum at $x = 2$, and $f$ does not have an absolute maximum.
	\ea
\end{activitySolution}
\aftera %ACTIVITY

One of the big lessons in finding absolute extreme values is the realization that the interval we choose has nearly the same impact on the problem as the function under consideration.  Consider, for instance, the function pictured in Figure~\ref{F:3.3.Interval}.

\begin{marginfigure}[1cm] % MARGIN FIGURE
\margingraphics{figures/3_3_Interval.eps}
\caption{A function $g$ considered on three different intervals.} \label{F:3.3.Interval}
\end{marginfigure}

In sequence, from left to right, as we see the interval under consideration change from $[-2,3]$ to $[-2,2]$ to $[-2,1]$, we move from having two critical values in the interval with the absolute minimum at one critical value and the absolute maximum at the right endpoint, to still having both critical numbers in the interval but then with the absolute minimum and maximum at the two critical values, to finally having just one critical value in the interval with the absolute maximum at one critical value and the absolute minimum at one endpoint.  It is particularly essential to always remember to only consider the critical values that lie within the interval.

%--------------------------------------------------------------
% SUBSECTION MOVING TOWARDS APPLICATIONS
%--------------------------------------------------------------
\subsection*{Moving towards applications}

In Section~\ref{S:3.4.AppliedOpt}, we will focus almost exclusively on applied optimization problems:  problems where we seek to find the absolute maximum or minimum value of a function that represents some physical situation.  We conclude this current section with an example of one such problem because it highlights the role that a closed, bounded domain can play in finding absolute extrema.  In addition, these problems often involve considerable preliminary work to develop the function which is to be optimized, and this example demonstrates that process.

\begin{example} \label{Ex:3.3.Eg1}
A $20$ cm piece of wire is cut into two pieces.  One piece is used to form a square and the other an equilateral triangle.  How should the wire be cut to maximize the total area enclosed by the square and triangle?  to minimize the area?

\solution
We begin by constructing a picture that exemplifies the given situation.  The primary variable in the problem is where we decide to cut the wire.  We thus label that point $x$, and note that the remaining portion of the wire then has length $20-x$

As shown in Figure~\ref{F:3.3.Ex1}, we see that the $x$ cm of the wire that are used to form the equilateral triangle result in a triangle with three sides of length $\frac{x}{3}$.  For the remaining $20-x$ cm of wire, the square that results will have each side of length $\frac{20-x}{4}$.

At this point, we note that there are obvious restrictions on $x$:  in particular, $0 \le x \le 20$.  In the extreme cases, all of the wire is being used to make just one figure.  For instance, if $x = 0$, then all $20$ cm of wire are used to make a square that is $5 \times 5$.

Now, our overall goal is to find the absolute minimum and absolute maximum areas that can be enclosed.  We note that the area of the triangle is $A_{\triangle} = \frac{1}{2} bh = \frac{1}{2} \cdot \frac{x}{3} \cdot \frac{x\sqrt{3}}{6}$, since the height of an equilateral triangle is $\sqrt{3}$ times half the length of the base.  Further, the area of the square is $A_{\Box} = s^2 = \left( \frac{20-x}{4} \right)^2$.  Therefore, the total area function is
$$A(x) = \frac{\sqrt{3}x^2}{36} + \left( \frac{20-x}{4} \right)^2.$$
Again, note that we are only considering this function on the restricted domain $[0,20]$ and we seek its absolute minimum and absolute maximum.

Differentiating $A(x)$, we have
$$A'(x) = \frac{\sqrt{3}x}{18} + 2\left( \frac{20-x}{4} \right)\left( -\frac{1}{4} \right) = \frac{\sqrt{3}}{18} x + \frac{1}{8}x - \frac{5}{2}.$$
Setting $A'(x) = 0$, it follows that $x = \frac{180}{4\sqrt{3}+9} \approx 11.3007$ is the only critical value of $A$, and we note that this lies within the interval $[0,20]$.  

Evaluating $A$ at the critical value and endpoints, we see that
\begin{itemize}
	\item $\ds A\left(\frac{180}{4\sqrt{3}+9}\right) = \frac{\sqrt{3}(\frac{180}{4\sqrt{3}+9})^2}{4} + \left( \frac{20-\frac{180}{4\sqrt{3}+9}}{4} \right)^2 \approx 10.8741$
	\item $\ds A(0) = 25$
	\item $\ds A(20) = \frac{\sqrt{3}}{36}(400) = \frac{100}{9} \sqrt{3} \approx 19.2450$
\end{itemize}
Thus, the absolute minimum occurs when $x \approx 11.3007$ and results in the minimum area of approximately $10.8741$ square centimeters, while the absolute maximum occurs when we invest all of the wire in the square (and none in the triangle), resulting in 25 square centimeters of area.  These results are confirmed by a plot of $y = A(x)$ on the interval $[0,20]$, as shown in Figure~\ref{F:3.3.Ex1Plot}.
\end{example}

\begin{marginfigure}[-32cm]
\margingraphics{figures/3_3_Ex1.eps} %Active ex3.4 
\caption{A 20 cm piece of wire cut into two pieces, one of which forms an equilateral triangle, the other which yields a square.} \label{F:3.3.Ex1}
\end{marginfigure}

\begin{marginfigure}[-10cm]
\margingraphics{figures/3_3_Ex1Plot.eps} 
\caption{A plot of the area function from Example~\ref{Ex:3.3.Eg1}.} \label{F:3.3.Ex1Plot}
\end{marginfigure} % EXAMPLE

\begin{activity} \label{A:3.3.3}  A piece of cardboard that is $10 \times 15$ (each measured in inches) is being made into a box without a top.  To do so, squares are cut from each corner of the box and the remaining sides are folded up.  If the box needs to be at least $1$ inch deep and no more than $3$ inches deep, what is the maximum possible volume of the box?  what is the minimum volume?  Justify your answers using calculus.
\ba
	\item Draw a labeled diagram that shows the given information.  What variable should we introduce to represent the choice we make in creating the box?  Label the diagram appropriately with the variable, and write a sentence to state what the variable represents.
	\item Determine a formula for the function $V$ (that depends on the variable in (a)) that tells us the volume of the box.
	\item What is the domain of the function $V$?  That is, what values of $x$ make sense for input?  Are there additional restrictions provided in the problem?
	\item Determine all critical values of the function $V$.
	\item Evaluate $V$ at each of the endpoints of the domain and at any critical values that lie in the domain.
	\item What is the maximum possible volume of the box?  the minimum?
\ea
\end{activity}
\begin{smallhint}
\ba
	\item Consider letting the length of one side of the removed squares be represented by $x$.
	\item Remember that the volume of a box is length $\times$ width $\times$ height.
	\item Read the given information carefully and think about the picture.
	\item Note that since $V$ is a cubic function, $V'$ is quadratic.
	\item Which critical values satisfy $1 \le x \le 3$?
	\item Evaluate the function at appropriate points.
\ea
\end{smallhint}
\begin{bighint}
\ba
	\item Consider letting the length of one side of the removed squares be represented by $x$.  Note, then, that one side of the box will have length $10 - 2x$.
	\item Remember that the volume of a box is length $\times$ width $\times$ height.  Write each of length, width, and height in terms of $x$.
	\item Read the given information carefully and think about the picture.
	\item Note that since $V$ is a cubic function, $V'$ is quadratic, so you can find the critical values exactly.
	\item Which critical values satisfy $1 \le x \le 3$?
	\item Evaluate the function at appropriate points and see which is greatest and which is least.
\ea
\end{bighint}
\begin{activitySolution}
\ba
	\item We let $x$ represent the length of a side of the square that is cut from each corner, so that we have the following picture:
	\begin{center}
	\includegraphics{figures/3_3_Act3Soln.eps}
	\end{center}
	\item Because the box has dimensions $(10-2x) \times (15-2x) \times x$, the volume of the box is given by 
	$$V(x) = x (10-2x) (15-2x) = 4x^3 - 50x^2 + 150x.$$
	\item Clearly the smallest $x$ can be is 0 and the largest $x$ can be is 5, since one side of the cardboard has length 10.  But we're told in the problem to restrict the value of $x$ to $1 \le x \le 3$, so this is the domain we use for $V$, even though $V$ is defined for every real number $x$.
	\item Since $V'(x) = 12x^2 - 100x + 150$, it follows that the critical values (where $V'(x) = 0$) are
	$$x = \frac{25 \pm 5\sqrt{7}}{6} \approx 6.371459426, 1.961873908.$$
	\item Only the latter critical number is in the relevant domain of $V$, and hence we consider
	\begin{itemize}
		\item $V(1.961873908) = 132.0382370$
		\item $V(1) = 104$
		\item $V(3) = 108$
	\end{itemize} 
	\item Hence the absolute maximum possible volume of the box is 132.0382370 and occurs when $x = 1.961873908$, while the absolute minimum is 104, which occurs when $x=1$.
\ea
\end{activitySolution}
\aftera %ACTIVITY

The approaches shown in Example~\ref{Ex:3.3.Eg1} and experienced in Activity~\ref{A:3.3.3} include standard steps that we undertake in almost every applied optimization problem:  we draw a picture to demonstrate the situation, introduce one or more variables to represent quantities that are changing, work to find a function that models the quantity to be optimized, and then decide an appropriate domain for that function.  Once that work is done, we are in the familiar situation of finding the absolute minimum and maximum of a function over a particular domain, at which time we apply the calculus ideas that we have been studying to this point in Chapter~\ref{CH:3}.

%--------------
% SUMMARY
%--------------
\begin{summary}
\item To find relative extreme values of a function, we normally use a first derivative sign chart and classify all of the function's critical values.  If instead we are interested in absolute extreme values, we first decide whether we are considering the entire domain of the function or a particular interval.  
\item In the case of finding global extremes over the function's entire domain, we again use a first or second derivative sign chart in an effort to make overall conclusions about whether or not the function can have a absolute maximum or minimum.
If we are working to find absolute extremes on a restricted interval, then we first identify all critical values of the function that lie in the interval.
\item For a continuous function on a closed, bounded interval, the only possible points at which absolute extreme values occur are the critical values and the endpoints.  Thus, to find said absolute extremes, we simply evaluate the function at each endpoint and each critical value in the interval, and then we compare the results to decide which is largest (the absolute maximum) and which is smallest (the absolute minimum). 
\end{summary}

\clearpage

%--------------
% EXERCISES
%--------------
\begin{adjustwidth*}{}{-2.25in}
\textbf{{\large Exercises}}
\setlength{\columnsep}{25pt}
\begin{multicols*}{2}
\noindent Terms and Concepts \small
\begin{enumerate}[1)]
\item Describe what an ``extreme value'' of a function is in your own words.
\item Sketch the graph of a function $f$ on $(-1,1)$ that has both a maximum and minimum value.
\item Describe the difference between and absolute and relative maximum in your own words.
\item Sketch the graph of a function $f$ where $f$ has a relative maximum at $x=1$ and $\fp(1)$ is undefined.
\item T/F: If $c$ is a critical value of a function $f$, then $f$ has either a relative maximum or relative minimum at $x=c$. 
\end{enumerate} 

\noindent {\normalsize Problems} \small

\noindent{\bf In exercises 6--7, identify each of the marked points as an absolute maximum or minimum, a local maximum or minimum, or none of the above.}

\begin{enumerate}[1),resume]
\item \begin{minipage}{\linewidth}
\includegraphics[scale=.8]{figures/fig03_01_ex_06}
\end{minipage}

\item \begin{minipage}{\linewidth}
\includegraphics[scale=.8]{figures/fig03_01_ex_07}
\end{minipage}
\end{enumerate}

\noindent{\bf In exercises 8--17, find the absolute extreme values of the given function on the specified interval.}

\begin{enumerate}[1),resume]
\item $\ds f(x) = x^2 + x + 4$\quad on \quad $[-1,2]$.
\item $\ds f(x) = x^3 - \frac{9}{2}x^2 - 30 x + 3$\quad  on \quad $[0,6]$.
\item $\ds f(x) = 3 \sin(x)$ \quad  on \quad $[\pi/4,2\pi/3]$.
\item $\ds f(x) = x^2 \sqrt{4-x^2}$ \quad  on \quad $[-2,2]$.
\item $\ds f(x) = x + \frac{3}{x}$ \quad  on \quad $[1,5]$.
\item $\ds f(x) = \frac{x^2}{x^2 + 5}$\quad  on \quad $[-3,5]$.
\item $\ds f(x) = e^x \cos(x)$\quad  on \quad $[0,\pi]$.
\item $\ds f(x) = e^x \sin(x)$\quad  on \quad $[0,\pi]$.
\item $\ds f(x) = \frac{\ln(x)}{x}$\quad  on \quad $[1,4]$.
\item $\ds f(x) = x^{2/3} - x$\quad  on \quad $[0,2]$.

\item Based on the given information about each function, decide whether the function has global maximum, a global minimum, neither, both, or that it is not possible to say without more information.  Assume that each function is twice differentiable and defined for all real numbers, unless noted otherwise.  In each case, write one sentence to explain your conclusion.
	\ba
		\item $f$ is a function such that $f''(x) < 0$ for every $x$.
		\item $g$ is a function with two critical values $a$ and $b$ (where $a < b$), and $g'(x) < 0$ for $x < a$, $g'(x) < 0$ for $a < x < b$, and $g'(x) > 0$ for $x > b$.
		\item $h$ is a function with two critical values $a$ and $b$ (where $a < b$), and $h'(x) < 0$ for $x < a$, $h'(x) > 0$ for $a < x < b$, and $h'(x) < 0$ for $x > b$.  In addition, $\lim_{x \to \infty} h(x) = 0$ and $\lim_{x \to -\infty} h(x) = 0$.
		\item $p$ is a function differentiable everywhere except at $x = a$ and $p''(x) > 0$ for $x < a$ and $p''(x) < 0$ for $x > a$.
	\ea

\item For each of the functions described below (each continuous on $[a,b]$), state the location of the function's absolute maximum and absolute minimum on the interval $[a,b]$, or say there is not enough information provided to make a conclusion.  Assume that any critical values mentioned in the problem statement represent all of the critical numbers the function has in $[a,b]$.  In each case, write one sentence to explain your answer.

	\ba
		\item $f'(x) \le 0$ for all $x$ in $[a,b]$
		\item $g$ has a critical value at $c$ such that $a < c< b$ and $g'(x) > 0$ for $x < c$ and $g'(x) < 0$ for $x > c$
		\item $h(a) = h(b)$ and $h''(x) < 0$ for all $x$ in $[a,b]$
		\item $p(a) > 0$, $p(b) < 0$, and for the critical value $c$ such that $a < c < b$, $p'(x) < 0$ for $x < c$ and $p'(x) > 0$ for $x > c$
	\ea

	\item Let $s(t) = 3\sin(2(t-\frac{\pi}{6})) + 5.$  Find the exact absolute maximum and minimum of $s$ on the provided intervals by testing the endpoints and finding and evaluating all relevant critical values of $s$.
	\ba
		\item $[\frac{\pi}{6}, \frac{7\pi}{6}]$
		\item $[0, \frac{\pi}{2}]$
		\item $[0, 2\pi]$
		\item $[\frac{\pi}{3}, \frac{5\pi}{6}]$
	\ea	
\end{enumerate}

%------------------------------------------
% END OF EXERCISES ON FIRST PAGE
%------------------------------------------
\end{multicols*}
\end{adjustwidth*}

%\clearpage
%
%\begin{adjustwidth*}{}{-2.25in}
%\setlength{\columnsep}{25pt}
%\begin{multicols*}{2}\small
%
%\begin{enumerate}[1),start=12]
%
%\item \label{exer:04_02_ex_12}A rope, attached to a weight, goes up through a pulley at the ceiling and back down to a worker. The man holds the rope at the same height as the connection point between rope and weight.
%
%\includegraphics[scale=1.25]{figures/fig04_02_ex_12}
%
%Suppose the man stands directly next to the weight (i.e., a total rope length of 60 ft) and begins to walk away at a rate of 2ft/s. How fast is the weight rising when the man has walked:
%\begin{enumerate}
%\item	10 feet?
%\item	40 feet?
%\end{enumerate}
%How far must the man walk to raise the weight all the way to the pulley?
%
%\item Consider the situation described in Exercise \ref{exer:04_02_ex_12}. Suppose the man starts 40ft from the weight and begins to walk away at a rate of 2ft/s. 
%\begin{enumerate}
%\item	How long is the rope?
%\item	How fast is the weight rising after the man has walked 10 feet?
%\item	How fast is the weight rising after the man has walked 40 feet?
%\item	How far must the man walk to raise the weight all the way to the pulley?
%\end{enumerate}
%
%\item A company that produces landscaping materials is dumping sand into a conical pile. The sand is being poured at a rate of 5ft$^3$/sec; the physical properties of the sand, in conjunction with gravity, ensure that the cone's height is roughly 2/3 the length of the diameter of the circular base. 
%
%How fast is the cone rising when it has a height of 30 feet?
%
%\item A baseball diamond is a square with sides 90 feet long.  Suppose a baseball player is advancing from second to third base at the rate of 24 feet per second, and an umpire is standing on home plate.  Let  $\theta$ be the angle between the third baseline and the line of sight from the umpire to the runner.  How fast is $\theta$ changing when the runner is 30 feet from third base?
%	
%\item Sand is being dumped off a conveyor belt onto a pile in such a way that the pile forms in the shape of a cone whose radius is always equal to its height.  Assuming that the sand is being dumped at a rate of 10 cubic feet per minute, how fast is the height of the pile changing when there are 1000 cubic feet on the pile?
%	
%\item A swimming pool is 60 feet long and 25 feet wide. Its depth varies uniformly from 3 feet at the shallow end to 15 feet at the deep end, as shown below.
%\includegraphics{figures/3_5_Ez3.eps}
%Suppose the pool has been emptied and is now being filled with water at a rate of 800 cubic feet per minute. At what rate is the depth of water (measured at the deepest point of the pool) increasing when it is 5 feet deep at that end?  Over time, describe how the depth of the water will increase:  at an increasing rate, at a decreasing rate, or at a constant rate.  Explain.
%
%\end{enumerate}
%
%%---------------------------------------------
%% END OF EXERCISES ON SECOND PAGE
%%---------------------------------------------
%\end{multicols*}
%\end{adjustwidth*}

\afterexercises 

\cleardoublepage