\begin{example} \label{Ex:1.4.Eg2} 
Show that $\ds \lim_{x \to 2} x^2 = 4$.

\solution Let's do this example symbolically from the start.  Let $\epsilon > 0$ be given; we want $|y-4| < \epsilon$, i.e.,  $|x^2-4| < \epsilon$.  How do we find $\delta$ such that when $|x-2| < \delta$, we are guaranteed that $|x^2-4|<\epsilon$?% for some $\delta$ (in terms of $\epsilon$)?

This is a bit trickier than the previous example, but let's start by noticing that 
$|x^2-4| = |x-2|\cdot|x+2|$.  Consider:
\begin{equation} |x^2-4| < \epsilon \longrightarrow |x-2|\cdot|x+2| < \epsilon \longrightarrow |x-2| < \frac{\epsilon}{|x+2|}.
\label{eq:limit1}
\end{equation} 
Could we not set $\ds \delta = \frac{\epsilon}{|x+2|}$?  

We are close to an answer, but the catch is that $\delta$ must be a \textit{constant} value (so it can't contain $x$).  There is a way to work around this, but we do have to make an assumption.  Remember that $\epsilon$ is supposed to be a small number, which implies that $\delta$ will also be a small value.  In particular, we can (probably) assume that $\delta < 1$.  If this is true, then $|x-2| < \delta$ would imply that $|x-2| < 1$, giving $1 < x < 3$.  

Now, back to the fraction $\ds \frac{\epsilon}{|x+2|}$.  If $1<x<3$, then $3<x+2<5$.  Taking reciprocals, we have $\ds \frac{1}{5}<\frac{1}{|x+2|}<\frac{1}{3}$ so that, in particular, 
\begin{equation} \frac{\epsilon}{5}<\frac{\epsilon}{|x+2|}.
\label{eq:limit2}
\end{equation}  
This suggests that we set 
$\ds \delta = \frac{\epsilon}{5}$. To see why, let's go back to the equations:

\begin{eqnarray*}
|x - 2| &<& \delta \\
|x - 2| &<& \frac{\epsilon}{5} \\%\qquad \text{\small(Our choice of $\delta$)}\\
|x - 2|\cdot|x + 2| &<& |x + 2|\cdot\frac{\epsilon}{5} \\%\qquad \text{\small(Multiply by $|x+2|$)}\\
|x^2 - 4|&<& |x + 2|\cdot\frac{\epsilon}{5} \\%\qquad \text{\small(Combine left side)}\\
|x^2 - 4|&<& |x + 2|\cdot\frac{\epsilon}{5} <|x + 2|\cdot\frac{\epsilon}{|x+2|}=\epsilon  %\qquad \text{\small(Using (\ref{eq:limit2}) as long as $\delta <1$)}
\end{eqnarray*}

We have arrived at $|x^2 - 4|<\epsilon$ as desired.  Note again, in order to make this happen we needed $\delta$ to first be less than 1.  That is a safe assumption; we want $\epsilon$ to be arbitrarily small, forcing $\delta$ to also be small. 

We have also picked $\delta$ to be smaller than ``necessary.'' We could get by with a slightly larger $\delta$, as shown in Figure~\ref{fig:1-4_Eg2}. The dashed, red lines show the boundaries defined by our choice of $\epsilon$. The gray, dashed lines show the boundaries defined by setting $\delta = \epsilon/5$. Note how these gray lines are within the red lines. That is perfectly fine; by choosing $x$ within the gray lines we are guaranteed that $f(x)$ will be within $\epsilon$ of $4$.%If the value we eventually used for $\delta$, namely $\epsilon/5$, is not less than 1, this proof won't work.  For the final fix, we instead set $\delta$ to be the minimum of 1 and $\epsilon/5$. This way all calculations above work.  

In summary, given $\epsilon > 0$, set $\delta=\epsilon/5$.  Then $|x - 2| < \delta$ implies 
$|x^2 - 4|< \epsilon$ (i.e. $|y - 4|< \epsilon$) as desired.  We have shown that $\ds \lim_{x\rightarrow 2} x^2 = 4 $. Figure~\ref{fig:1-4_Eg2} gives a visualization of this; by restricting $x$ to values within $\delta = \epsilon/5$ of $2$, we see that $f(x)$ is within $\epsilon$ of $4$.
\end{example}

\begin{marginfigure}[-8cm]
%\margingraphics{figures/figLimitProof2a}
\margingraphics{figs/1/1-4_Eg2.pdf}
\caption{Choosing $\delta = \epsilon/5$ in Example \ref{Ex:1.4.Eg2}.}\label{fig:1-4_Eg2}
\end{marginfigure}