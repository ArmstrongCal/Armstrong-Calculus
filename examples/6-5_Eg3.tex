%\begin{marginfigure} % MARGIN FIGURE
%\margingraphics{figures/6_4_DamEx.eps}
%\caption{A trapezoidal dam that is 25 feet tall, 60 feet wide at its base, 90 feet wide at its top, with the water line 5 feet down from the top of its face.} \label{F:6.4.DamEx}
%\end{marginfigure}

\begin{example} \label{eg:6.5.3} % EXAMPLE
A box of 100 lb of sand is being pulled up at a uniform rate a distance of 50 ft over 1 minute. The sand is leaking from the box at a rate of 1 lb/s. The box itself weighs 5 lb and is pulled by a rope weighing .2 lb/ft. 
	\begin{enumerate}
	\item		How much work is done lifting just the rope?
	\item		How much work is done lifting just the box and sand?
	\item		What is the total amount of work performed?
	\end{enumerate}

\solution
\begin{enumerate}
	\item		We start by forming the force function $F_r(x)$ for the rope (where the subscript denotes we are considering the rope). As in the previous example, let $x$ denote the amount of rope, in feet, pulled in. (This is the same as saying $x$ denotes the height of the box.) The weight of the rope with $x$ feet pulled in is $F_r(x) = 0.2(50-x) = 10-0.2x$. (Note that we do not have to include the acceleration of gravity here, for the \textit{weight} of the rope per foot is given, not its \textit{mass} per meter as before.) The work performed lifting the rope is 
	$$W_r = \int_0^{50} (10-0.2x)\ dx = 250\ \text{ft--lb}.$$
	
	\item	The sand is leaving the box at a rate of 1 lb/s. As the vertical trip is to take one minute, we know that 60 lb will have left when the box reaches its final height of 50 ft. Again letting $x$ represent the height of the box, we have two points on the line that describes the weight of the sand: when $x=0$, the sand weight is 100 lb, producing the point $(0,100)$; when $x=50$, the sand in the box weighs 40 lb, producing the point $(50,40)$. The slope of this line is $\frac{100-40}{0-50} = -1.2$, giving the equation of the weight of the sand at height $x$ as $w(x) = -1.2x+100$. The box itself weighs a constant 5 lb, so the total force function is $F_b(x) = -1.2x+105$. Integrating from $x=0$ to $x=50$ gives the work performed in lifting box and sand:
	$$W_b = \int_0^{50} (-1.2x+105)\ dx = 3750\ \text{ft--lb.}$$
	
	\item	The total work is the sum of $W_r$ and $W_b$: $250+3750=4000$ ft--lb. We can also arrive at this via integration:
	\begin{align*} W &= \int_0^{50} (F_r(x)+F_b(x))\ dx \\
									&= \int_0^{50} (10-0.2x-1.2x+105)\ dx \\
									&= \int_0^{50} (-1.4x+115) \ dx \\
									&= 4000 \ \text{ft--lb.}
	\end{align*}	
\end{enumerate}

\end{example}