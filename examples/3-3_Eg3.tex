\begin{example} \label{Ex:3.3.Eg3}
Find the maximum and minimum values of $f$ on $[-4,2]$, where $$f(x) = \left\{\begin{array}{cc} (x-1)^2 & x\leq 0 \\ x+1 & x>0 \end{array}\right. .$$

\solution
Here $f$ is piecewise--defined, but we can still apply our approach. Evaluating $f$ at the endpoints gives: 
$$ f(-4) = 25 \quad \text{and} \quad f(2) = 3.$$

We now find the critical numbers of $f$. We have to define $\fp$ in a piecewise manner; it is $$\fp(x) =\left\{\begin{array}{cc} 2(x-1) & x < 0 \\ 1 & x>0 \end{array}\right. .$$ Note that while $f$ is defined for all of $[-4,2]$, $\fp$ is not, as the derivative of $f$ does not exist when $x=0$. (From the left, the derivative approaches $-2$; from the right the derivative is $1$.) Thus one critical number of $f$ is $x=0$.

We now set $\fp(x) = 0$. When $x >0$, $\fp(x)$ is never $0$.  When $x<0$, $\fp(x)$ is also never $0$. (We may be tempted to say that $\fp(x) = 0 $ when $x=1$. However, this is nonsensical, for we only consider $\fp(x) = 2(x-1)$ when $x<0$, so we will ignore a solution that says $x=1$. 

So we have three important $x$ values to consider: $x= -4, 2$ and $0$. Evaluating $f$ at each gives, respectively, $25$, $3$ and $1$, shown in Table~\ref{T:3.3.Ex3}. Thus the absolute minimum of $f$ is $1$; the absolute maximum of $f$ is $25$. Our answer is confirmed by the graph of $f$ in Figure \ref{F:3.3.Ex3}.
\end{example}

\begin{marginfigure}[-18cm]
\margingraphics{figures/figextval5} %APEX ex79 
\caption{A graph of $f(x)$ on $[-4,2]$ as in Example \ref{Ex:3.3.Eg3}. } \label{F:3.3.Ex3}
\end{marginfigure}

\begin{margintable}[-6cm]
\begin{center}
\scalebox{1.25}{
\begin{tabular}{cc} 
$x$ & $f(x)$ \\ \hline \rule{0pt}{10pt} 
$-4$ & $25$ \\ 
$0$ & $1$ \\ 
$2$ & $3$
\end{tabular}
} % end scalebox
\end{center}
\caption{Finding the extreme values of $f$ in Example \ref{Ex:3.3.Eg3}.} \label{T:3.3.Ex3}
\end{margintable}