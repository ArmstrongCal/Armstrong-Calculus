\begin{example} \label{Ex:1.1.Eg5}
Explore why $\ds \lim_{x\to 1} f(x)$ does not exist, where 
\[ f(x) = \left\{ \begin{array}{cl} x^2-2x+3 & x\leq 1 \\ x & x>1 \end{array} \right..\]

\solution A graph of $f(x)$ around $x=1$ is given in Figure~\ref{fig:1-1_Eg5}. It is clear that as $x$ approaches 1, $f(x)$ does not approach a single number. Instead, $f(x)$ approaches two different numbers. When considering values of $x$ less than 1 (approaching 1 from the left), $f(x)$ is approaching 2; when considering values of $x$ greater than 1 (approaching 1 from the right), $f(x)$ is approaching 1. Recognizing this behavior is important; we'll study this in greater depth later. Right now, it suffices to say that the limit does not exist since $f(x)$ is not approaching a single value as $x$ approaches 1.
\end{example}

\begin{marginfigure}[-8cm]
\margingraphics{figs/1/fignolimit1.pdf}
\caption{Observing no limit as $x\to 1$.}\label{fig:1-1_Eg5}
\end{marginfigure}