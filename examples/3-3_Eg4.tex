\begin{marginfigure}[-1cm]
\margingraphics{figures/figextval6} %APEX ex80 
\caption{A graph of $f(x)=\cos(x^2)$ on $[-2,2]$ as in Example \ref{Ex:3.3.Eg4}. } \label{F:3.3.Ex4}
\end{marginfigure}

\begin{margintable}
\begin{center}
\scalebox{1.25}{
\begin{tabular}{cc} 
$x$ & $f(x)$ \\ \hline \rule{0pt}{10pt} 
$-2$ & $-0.65$ \\ 
$-\sqrt{\pi}$ & $-1$ \\
$0$ & $1$\\
$\sqrt{\pi}$ & $-1$ \\
$2$ & $-0.65$
\end{tabular}
} % end scalebox
\end{center}
\caption{Finding the extreme values of $f(x)= \cos (x^2)$ in Example \ref{Ex:3.3.Eg4}.} \label{T:3.3.Ex4}
\end{margintable}

\begin{example} \label{Ex:3.3.Eg4}
Find the extrema of  $f(x) = \cos (x^2)$ on $[-2,2]$.

\solution
We again our approach to find extrema. Evaluating $f$ at the endpoints of the interval gives: $f(-2) = f(2) = \cos (4) \approx -0.6536.$ We now find the critical values of $f$.

Applying the Chain Rule, we find $\fp(x) = -2x\sin (x^2)$. Set $\fp(x) = 0$ and solve for $x$ to find the critical values of $f$. 

We have $\fp(x) = 0$ when $x = 0$ and when $\sin (x^2) = 0$. In general, $\sin(t) = 0$ when $t = \ldots -2\pi, -\pi, 0, \pi, \ldots$ Thus $\sin (x^2) = 0$ when $x^2 = 0, \pi, 2\pi, \ldots$ ($x^2$ is always positive so we ignore $-\pi$, etc.) So $\sin (x^2)=0$ when $x= 0, \pm \sqrt{\pi}, \pm\sqrt{2\pi}, \ldots$. The only values to fall in the given interval of $[-2,2]$ are $-\sqrt{\pi}$ and $\sqrt{\pi}$, approximately $\pm 1.77$.

We again construct a table of important values in Table~\ref{T:3.3.Ex4}. In this example we have $5$ values to consider: $x= 0, \pm 2, \pm\sqrt{\pi}$. 

From the table it is clear that the maximum value of $f$ on $[-2,2]$ is $1$; the minimum value is $-1$. The graph in Figure \ref{F:3.3.Ex4} confirms our results.
\end{example}