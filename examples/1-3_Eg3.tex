\begin{marginfigure}[6cm]
\margingraphics{figs/1/figXMinusCosX.pdf} 
\caption{Graphing a root of $f(x) = x-\cos(x)$.}\label{fig:1-3_Eg3}
\end{marginfigure}

\begin{margintable}
\scalebox{1.1}{
\begin{tabular}{ccc}
Iteration \# & Interval & Midpoint Sign \\ \hline
			1 & $[0.7,0.9]$ & $f(0.8) >0$ \\
			2 & $[0.7,0.8] $ & $f(0.75) >0$ \\
			3 & $[0.7,0.75]$ & $f(0.725)<0$\\
			4 & $[0.725,0.75]$ & $f(0.7375)<0$\\
			5 & $[0.7375,0.75]$ & $f(0.7438)>0$\\
			6 & $[0.7375,0.7438]$ & $f(0.7407)>0$\\
			7 & $[0.7375,0.7407]$ & $f(0.7391)>0$\\
			8 & $[0.7375,0.7391]$ & $f(0.7383)<0$\\
			9 & $[0.7383,0.7391]$ & $f(0.7387)<0$\\
			10 & $[0.7387,0.7391]$ & $f(0.7389)<0$\\
			11 & $[0.7389,0.7391]$ & $f(0.7390)<0$\\
			12 & $[0.7390,0.7391]$ &   \\
\end{tabular}
} % end scalebox 
\caption{Iterations of the Bisection Method of Root Finding}\label{T:1-3_Eg3}
\end{margintable}

\begin{example}   %Using the Bisection Method 
Approximate the root of $f(x) = x-\cos x$, accurate to three places after the decimal.

\solution Consider the graph of $f(x) = x-\cos x$, shown in Figure~\ref{fig:1-3_Eg3}. It is clear that the graph crosses the $x$-axis somewhere near $x=0.8$. To start the Bisection Method, pick an interval that contains $0.8$. We choose $[0.7,0.9]$. Note that all we care about are signs of $f(x)$, not their actual value, so this is all we display.

\begin{description}
\item[Iteration 1:] $f(0.7) < 0$, $f(0.9) > 0$, and $f(0.8) >0$. So replace $0.9$ with $0.8$ and repeat.
\item[Iteration 2:] $f(0.7)<0$, $f(0.8) > 0$, and at the midpoint, $0.75$, we have $f(0.75) >0 $. So replace $0.8$ with $0.75$ and repeat. Note that we don't need to continue to check the endpoints, just the midpoint. Thus we put the rest of the iterations in Table~\ref{T:1-3_Eg3}.
\end{description}

Notice that in the $12^{\text{th}}$ iteration we have the endpoints of the interval each starting with $0.739$. Thus we have narrowed the zero down to an accuracy of the first three places after the decimal. Using a computer, we have 
\[ f(0.7390) = -0.00014, \quad f(0.7391) = 0.000024. \] 
Either endpoint of the interval gives a good approximation of where $f$ is $0$. The Intermediate Value Theorem states that the actual zero is still within this interval. While we do not know its exact value, we know it starts with $0.739$. 

This type of exercise is rarely done by hand. Rather, it is simple to program a computer to run such an algorithm and stop when the endpoints differ by a preset small amount. One of the authors did write such a program and found the zero of $f$, accurate to $10$ places after the decimal, to be $0.7390851332$. While it took a few minutes to write the program, it took less than a thousandth of a second for the program to run the necessary $35$ iterations. In less than $8$ hundredths of a second, the zero was calculated to $100$ decimal places (with less than $200$ iterations).
\end{example}
