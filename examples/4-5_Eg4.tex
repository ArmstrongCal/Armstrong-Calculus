\begin{example}
The acceleration due to gravity of a falling object is $-32$ ft/s$^2$. At time $t=3$, a falling object had a velocity of $-10$ ft/s. Find the equation of the object's velocity.
We want to know a velocity function, $v(t)$. We know two things:
\begin{itemize}
\item The acceleration, i.e., $v'(t)= -32$, and
\item the velocity at a specific time, i.e., $v(3) = -10$.
\end{itemize}
Using the first piece of information, we know that $v(t)$ is an antiderivative of $v'(t)=-32$. So we begin by finding the indefinite integral of $-32$:
\[ \int (-32)\ dt = -32t+C=v(t). \]
Now we use the fact that $v(3)=-10$ to find $C$:
\begin{align*}
v(t) &= -32t+C \\
v(3) &= -10 \\
-32(3)+C &= -10\\
C &= 86
\end{align*}

Thus $v(t)= -32t+86$. We can use this equation to understand the motion of the object: when $t=0$, the object had a velocity of $v(0) = 86$ ft/s. Since the velocity is positive, the object was moving upward.

When did the object begin moving down? Immediately after $v(t) = 0$:
\[ -32t+86 = 0 \quad \Rightarrow \quad  t = \frac{43}{16}  \approx 2.69\text{ s}. \]
Recognize that we are able to determine quite a bit about the path of the object knowing just its acceleration and its velocity at a single point in time.
\end{example}