\begin{example} \label{eg:5.4.2} % EXAMPLE
Use partial fraction decomposition to integrate $\ds\int\frac{1}{(x-1)(x+2)^2}\ dx.$

\solution We decompose the integrand as follows:
$$\frac{1}{(x-1)(x+2)^2} = \frac{A}{x-1} + \frac{B}{x+2} + \frac{C}{(x+2)^2}.$$
To solve for $A$, $B$ and $C$, we multiply both sides by $(x-1)(x+2)^2$ and collect like terms:
\begin{align}
1 &= A(x+2)^2 + B(x-1)(x+2) + C(x-1)\label{eq:pf3}\\
	&= Ax^2+4Ax+4A + Bx^2 + Bx-2B + Cx-C \notag \\
	&= (A+B)x^2 + (4A+B+C)x + (4A-2B-C)\notag
\end{align}
Equation~\ref{eq:pf3} offers a direct route to finding the values of $A$, $B$ and $C$. Since the equation holds for all values of $x$, it holds in particular when $x=1$. However, when $x=1$, the right hand side simplifies to $A(1+2)^2 = 9A$. Since the left hand side is still $1$, we have $1 = 9A$. Hence $A = 1/9$.

Likewise, the equality holds when $x=-2$; this leads to the equation $1=-3C$. Thus $C = -1/3$.

We can find the value of $B$ by expanding the terms as shown in the example.

We have $$0x^2+0x+ 1 = (A+B)x^2 + (4A+B+C)x + (4A-2B-C)$$
leading to the equations 
$$A+B = 0, \quad 4A+B+C = 0 \quad \text{and} \quad 4A-2B-C = 1.$$
These three equations of three unknowns lead to a unique solution:
$$A = 1/9,\quad B = -1/9 \quad \text{and} \quad C = -1/3.$$
Thus 
$$\int\frac{1}{(x-1)(x+2)^2}\ dx = \int \frac{1/9}{x-1}\ dx + \int \frac{-1/9}{x+2}\ dx + \int \frac{-1/3}{(x+2)^2}\ dx.$$

Each can be integrated with a simple substitution with $u=x-1$ or $u=x+2$. The end result is
$$\int\frac{1}{(x-1)(x+2)^2}\ dx = \frac19\ln|x-1| -\frac19\ln|x+2| +\frac1{3(x+2)}+C.$$
\end{example}