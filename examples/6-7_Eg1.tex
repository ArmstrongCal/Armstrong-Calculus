%\begin{marginfigure} % MARGIN FIGURE
%\margingraphics{figures/6_4_DamEx.eps}
%\caption{A trapezoidal dam that is 25 feet tall, 60 feet wide at its base, 90 feet wide at its top, with the water line 5 feet down from the top of its face.} \label{F:6.4.DamEx}
%\end{marginfigure}

\begin{example} \label{eg:6.7.1} % EXAMPLE
Find all functions $y$ that are solutions to the differential equation 
$$\frac{dy}{dt}= \frac{t}{y^2}.$$

\solution We begin by separating the variables and writing
$$ y^2\frac{dy}{dt} = t. $$
Integrating both sides of the equation with respect to the independent variable $t$ shows that
$$ \int y^2\frac{dy}{dt}~dt = \int t~dt. $$
Next, we notice that the left-hand side allows us to change the variable of antidifferentiation from $t$ to $y$.  In particular, $dy = \frac{dy}{dt}~dt$, so we now have
$$ \int y^2 ~dy = \int t~dt. $$
This is why we required that the left-hand side be written as a product in which $dy/dt$ is one of the terms. This most recent equation says that two families of antiderivatives are equal to one another.  Therefore, when we find representative antiderivatives of both sides, we know they must differ by arbitrary constant $C$.  Antidifferentiating and including the integration constant $C$ on the right, we find that
$$ \frac{y^3}{3} = \frac{t^2}{2} + C. $$
    Again, note that it is not necessary to include an arbitrary constant on both sides 
    of the equation;  we know that $y^3/3$ and $t^2/2$ are in the same
    family of antiderivatives and must therefore differ by a single
    constant.

Finally, we may now solve the last equation above for $y$ as a function of $t$, which gives
    $$
    y(t) = \sqrt[3]{\frac 32 \thinspace t^2 + 3C}.
    $$
    Of course, the term $3C$ on the right-hand side represents
    $3$ times an unknown constant.  It is, therefore, still an unknown
    constant, which we will rewrite as $C$.  We thus conclude that the funtion
    $$
    y(t) = \sqrt[3]{\frac 32 \thinspace t^2 + C}
    $$
is a solution to the original differential equation for any value of $C$.

Notice that because this solution depends on the arbitrary constant $C$, we have found an infinite family of
solutions.  This makes sense because we expect to find a unique solution that corresponds to any given
 initial value.

For example, if we want to solve the initial value problem
$$
  \frac{dy}{dt} = \frac{t}{y^2}, \
  y(0) = 2,
$$
we know that the solution has the form $y(t) = \sqrt[3]{\frac32\thinspace
  t^2 + C}$ for some constant $C$.  We therefore must find the appropriate
value for $C$ that gives the initial value $y(0)=2$.  Hence,
$$
  2 = y(0)  \sqrt[3]{\frac 32 \thinspace 0^2 + C} = \sqrt[3]{C},
  $$
which shows that $C = 2^3 = 8$.  The solution to the initial value problem is then
$$
y(t) = \sqrt[3]{\frac32\thinspace t^2+8}.
$$
\end{example}