%\mtable{.55}{Plots of sequences in Example \ref{ex_seq7}.}{fig:seq7}{%
%\begin{tabular}{c}
%\myincludegraphics{figures/figseq7a}\\
%(a)\rule[-25pt]{0pt}{10pt}\\ 
%\myincludegraphics{figures/figseq7b}\\
%(b)\rule[-25pt]{0pt}{10pt}\\ 
%\myincludegraphics{figures/figseq7c}\\
%(c)\\
%\end{tabular}
%}
%\mfigure{.6}{A plot of $\{a_n\} = \{n^2/n!\}$ in Example \ref{ex_seq7}.}{fig:seq7d}{figures/figseq7d}

\begin{marginfigure}[3cm] % MARGIN FIGURE
\subfloat[]{\margingraphics{figures/figseq7a}}

\subfloat[]{\margingraphics{figures/figseq7b}}

\subfloat[]{\margingraphics{figures/figseq7c}}
\caption{Plots of sequences in Example~\ref{eg:7.1.6}.}
\label{fig:seq7}
\end{marginfigure}

\begin{example} \label{eg:7.1.6} % EXAMPLE
Determine the monotonicity of the following sequences.

\begin{enumerate}[1)]
\item $\ds \{a_n\} = \left\{\frac{n+1}n\right\}$
\item	$\ds \{a_n\} = \left\{\frac{n^2+1}{n+1}\right\}$	
\item $\ds \{a_n\} = \left\{\frac{n^2-9}{n^2-10n+26}\right\}$
\item	$\ds \{a_n\} = \left\{\frac{n^2}{n!}\right\}$	
\end{enumerate}

\solution In each of the following, we will examine $a_{n+1}-a_n$. If $a_{n+1}-a_n >0$, we conclude that $a_n<a_{n+1}$ and hence the sequence is increasing. If $a_{n+1}-a_n<0$, we conclude that $a_n>a_{n+1}$ and the sequence is decreasing. Of course, a sequence need not be monotonic and perhaps neither of the above will apply.

We also give a scatter plot of each sequence. These are useful as they suggest a pattern of monotonicity, but analytic work should be done to confirm a graphical trend.

\begin{enumerate}[1)]
\item	
\begin{align*}
a_{n+1}-a_n &= \frac{n+2}{n+1} - \frac{n+1}{n} \\		
&= \frac{(n+2)(n)-(n+1)^2}{(n+1)n} \\
&=	\frac{-1}{n(n+1)} \\
&<0 \quad\text{ for all $n$.}
\end{align*}
				
Since $a_{n+1}-a_n<0$ for all $n$, we conclude that the sequence is decreasing.

\item	
\begin{align*}	
a_{n+1}-a_n &= \frac{(n+1)^2+1}{n+2} - \frac{n^2+1}{n+1} \\		
&= \frac{\big((n+1)^2+1\big)(n+1)- (n^2+1)(n+2)}{(n+1)(n+2)}\\
&=	\frac{n^2+4n+1}{(n+1)(n+2)} \\
&> 0 \quad \text{ for all $n$.}
\end{align*}
					
Since $a_{n+1}-a_n>0$ for all $n$, we conclude the sequence is increasing.

\item We can clearly see in Figure \ref{fig:seq7}-(c), where the sequence is plotted, that it is not monotonic. However, it does seem that after the first 4 terms it is decreasing. To understand why, perform the same analysis as done before:
\begin{align*}
a_{n+1}-a_n &= \frac{(n+1)^2-9}{(n+1)^2-10(n+1)+26} - \frac{n^2-9}{n^2-10n+26} \\		
&= \frac{n^2+2n-8}{n^2-8n+17}-\frac{n^2-9}{n^2-10n+26}\\
&= \frac{(n^2+2n-8)(n^2-10n+26)-(n^2-9)(n^2-8n+17)}{(n^2-8n+17)(n^2-10n+26)}\\
&= \frac{-10n^2+60n-55}{(n^2-8n+17)(n^2-10n+26)}.
\end{align*}

We want to know when this is greater than, or less than, $0$, therefore we are only concerned with the numerator. Using the quadratic formula, we can determine that $-10n^2+60n-55=0$ when $n\approx 1.13, 4.87$. So for $n<1.13$, the sequence is decreasing. Since we are only dealing with the natural numbers, this means that $a_1 > a_2$.

Between $1.13$ and $4.87$, i.e., for $n=2$, $3$ and $4$, we have that $a_{n+1}>a_n$ and the sequence is increasing. (That is, when $n=2$, $3$ and $4$, the numerator $-10n^2+60n+55$ from the fraction above is $>0$.)

When $n> 4.87$, i.e, for $n\geq 5$, we have that $-10n^2+60n+55<0$, hence $a_{n+1}-a_n<0$, so the sequence is decreasing.

In short, the sequence is simply not monotonic. However, it is useful to note that for $n\geq 5$, the sequence is monotonically decreasing. 

\item Again, the plot in Figure \ref{fig:seq7d} shows that the sequence is not monotonic, but it suggests that it is monotonically decreasing after the first term. We perform the usual analysis to confirm this.
\begin{align*}	
a_{n+1}-a_n &= \frac{(n+1)^2}{(n+1)!} - \frac{n^2}{n!} \\
&= \frac{(n+1)^2-n^2(n+1)}{(n+1)!} \\
&=	\frac{-n^3+2n+1}{(n+1)!}
\end{align*}
					
When $n=1$, the above expression is $>0$; for $n\geq 2$, the above expression is $<0$. Thus this sequence is not monotonic, but it is monotonically decreasing after the first term.
\end{enumerate}
\end{example}

\begin{marginfigure}[-6cm] % MARGIN FIGURE
\margingraphics{figures/figseq7d}
\caption{Plots of sequences in Example~\ref{eg:7.1.6}.}
\label{fig:seq7d}
\end{marginfigure}