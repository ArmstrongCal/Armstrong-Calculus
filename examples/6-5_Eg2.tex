%\begin{marginfigure} % MARGIN FIGURE
%\margingraphics{figures/6_4_DamEx.eps}
%\caption{A trapezoidal dam that is 25 feet tall, 60 feet wide at its base, 90 feet wide at its top, with the water line 5 feet down from the top of its face.} \label{F:6.4.DamEx}
%\end{marginfigure}

\begin{example} \label{eg:6.5.2} % EXAMPLE
How much work is performed pulling a 60 m climbing rope up a cliff face, where the rope has a mass of 66 g/m?


\solution We need to create a force function $F(x)$ on the interval $[0,60]$. To do so, we must first decide what $x$ is measuring: it is the length of the rope still hanging or is it the amount of rope pulled in? As long as we are consistent, either approach is fine. We adopt for this example the convention that $x$ is the amount of rope pulled in. This seems to match intuition better; pulling up the first 10 meters of rope involves $x=0$ to $x=10$ instead of $x=60$ to $x=50$. 

As $x$ is the amount of rope pulled in, the amount of rope still hanging is $60-x$. This length of rope has a mass of 66 g/m, or $0.066$ kg/m. The the mass of the rope still hanging is $0.066(60-x)$ kg; multiplying this mass by the acceleration of gravity, 9.8 m/s$^2$, gives our variable force function $$F(x) = (9.8)(0.066)(60-x) = 0.6468(60-x).$$

Thus the total work performed in pulling up the rope is 
$$W = \int_0^{60} 0.6468(60-x)\ dx = 1,164.24\ \text{J}.$$

By comparison, consider the work done in lifting the entire rope 60 meters. The rope weights $60\times 0.066 \times 9.8 = 38.808$ N, so the work applying this force for 60 meters is $60\times 38.808 = 2,328.48$ J. This is exactly twice the work calculated before (and we leave it to the reader to understand why.)
\end{example}