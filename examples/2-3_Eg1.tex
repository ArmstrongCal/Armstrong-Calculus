\begin{example} \label{Ex:2.3.Eg1}
Let $f(x) = x^3+x^2-x+1$. Find intervals on which $f$ is increasing or decreasing.

\solution We first find the values where $f$ is neither increasing or decreasing. Given $f'(x) = 3x^2+2x-1 = (3x-1)(x+1)$, then  $f'(x) = 0$ when $x=-1$ and when $x=1/3$. Notice that $f'$ is never undefined.

Since an interval was not specified for us to consider, we consider the entire domain of $f$, which is $(-\infty,\infty)$. We thus break the whole real line into three subintervals based on the two values where the derivative is zero: $(-\infty,-1)$, $(-1,1/3)$ and $(1/3,\infty)$. This is shown in Figure \ref{F:2.3.Eg1-1}.

We now pick a value $p$ in each subinterval and find the sign of $f'(p)$. All we care about is the sign, so we do not actually have to fully compute $f'(p)$; pick "nice" values that make this simple.

\noindent\textbf{Subinterval 1}, $(-\infty,-1)$:\quad We (arbitrarily) pick $p=-2$. We can compute $f'(-2)$ directly: $f'(-2) = 3(-2)^2+2(-2)-1=7>0$. We conclude that $f$ is increasing on $(-\infty,-1)$.

Note we can arrive at the same conclusion without computation. For instance, we could choose $p=-100$. The first term in $f'(-100)$, i.e., $3(-100)^2$ is clearly positive and very large. The other terms are small in comparison, so we know $f'(-100)>0$. All we need is the sign.\\

\noindent\textbf{Subinterval 2}, $(-1,1/3)$:\quad We pick $p=0$ since that value seems easy to deal with. $f'(0) = -1<0$. We conclude $f$ is decreasing on $(-1,1/3)$.\\

\noindent\textbf{Subinterval 3}, $(1/3,\infty)$:\quad Pick an arbitrarily large value for $p>1/3$ and note that $f'(p) =3p^2+2p-1 >0$. We conclude that $f$ is increasing on $(1/3,\infty)$.

We can verify our calculations by considering Figure \ref{F:2.3.Eg1-2}, where $f$ is graphed in blue. The graph also presents $f'$ in red; note how $f'>0$ when $f$ is increasing and $f'<0$ when $f$ is decreasing.
\end{example}

\begin{marginfigure}[-15cm]
\margingraphics{figures/figincrline1} %APEX EX 84 
\caption{Number line for $f$ in Example \ref{Ex:2.3.Eg1}.}\label{F:2.3.Eg1-1}
\end{marginfigure}

\begin{marginfigure}[-6cm]
\margingraphics{figures/figincr1} %APEX EX 84
\caption{A graph of $f(x)$ in Example \ref{Ex:2.3.Eg1}, showing where $f$ is increasing and decreasing.}\label{F:2.3.Eg1-2}
\end{marginfigure}