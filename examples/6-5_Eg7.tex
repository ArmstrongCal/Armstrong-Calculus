
\begin{example} \label{eg:6.5.7} % EXAMPLE
A rectangular swimming pool is 20 ft wide and has a 3 ft ``shallow end'' and a 6 ft ``deep end.'' It is to have its water pumped out to a point 2 ft above the current top of the water. 
The cross--sectional dimensions of the water in the pool are given in Figure \ref{F:6.5.Pool1}; note that the dimensions are for the water, not the pool itself. Compute the amount of work performed in draining the pool.

\solution
For the purposes of this problem we choose to set $y=0$ to represent the bottom of the pool, meaning the top of the water is at $y=6$. 

Figure \ref{F:6.5.Pool2} shows the pool oriented with this $y$-axis, along with 2 differential elements as the pool must be split into two different regions. 

The top region lies in the $y$-interval of $[3,6]$, where the length of the differential element is $25$ ft as shown. As the pool is 20 ft wide, this differential element represents a this slice of water with volume $V(y) = 20\cdot25\cdot dy$.  The water is to be pumped to a height of $y=8$, so the height function is $h(y) = 8-y$. The work done in pumping this top region of water is 
$$W_t = 62.4\int_3^6 500(8-y)\ dy = 327,600 \text{ ft--lb}.$$

The bottom region lies in the $y$-interval of $[0,3]$; we need to compute the length of the differential element in this interval.

One end of the differential element is at $x=0$ and the other is along the line segment joining the points $(10,0)$ and $(15,3)$. The equation of this line is $y= 3/5(x-10)$; as we will be integrating with respect to $y$, we rewrite this equation as $x=5/3y+10$. So the length of the differential element is a difference of $x$-values: $x=0$ and $x=5/3y+10$, giving a length of $x=5/3y+10$. 

Again, as the pool is 20 ft wide, this differential element  represents a thin slice of water with volume $V(y) = 20\cdot(5/3y+10)\cdot dy$; the height function is the same as before at $h(y)=8-y$. The work performed in emptying this part of the pool is
$$W_b = 62.4\int_0^3 20(5/3y+10)(8-y)\ dy = 299,520\ \text{ft--lb}.$$
The total work in emptying the pool is 
$$W = W_b+W_t = 327,600+299,520 = 627,120\ \text{ft--lb}.$$		
Notice how the emptying of the bottom of the pool performs almost as much work as emptying the top. The top portion travels a shorter distance but has more water. In the end, this extra water produces more work.
\end{example}

\begin{marginfigure}[-24cm] % MARGIN FIGURE
\margingraphics{figures/figpump4}
\caption{The cross--section of a swimming pool filled with water in Example \ref{eg:6.5.7}.} \label{F:6.5.Pool1}
\end{marginfigure}

\begin{marginfigure}[-12cm] % MARGIN FIGURE
\margingraphics{figures/figpump4b}
\caption{Orienting the pool and showing differential elements for Example \ref{eg:6.5.7}.} \label{F:6.5.Pool2}
\end{marginfigure}
