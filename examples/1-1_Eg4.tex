\begin{marginfigure}
\margingraphics{figs/1/figSqueeze1.pdf}
\caption{The unit circle and related triangles.}\label{fig:1-1_Eg4}
\end{marginfigure}

\begin{example} \label{Ex:1.1.Eg4}
Use the Squeeze Theorem to show that
\[ \lim_{x \to 0} \frac{\sin(x)}{x} = 1 \]

\solution We begin by considering the unit circle. Each point on the unit circle has coordinates $(\cos (\theta),\sin (\theta))$ for some angle $\theta$ as shown in Figure~\ref{fig:1-1_Eg4}. Using similar triangles, we can extend the line from the origin through the point to the point $(1,\tan (\theta))$, as shown. (Here we are assuming that $0 \leq \theta \leq \pi/2$. Later we will show that we can also consider $\theta \leq 0$.)

The area of the large triangle is $\frac{1}{2} \tan(\theta)$; the area of the sector is $\theta/2$; the area of the triangle contained inside the sector is $\frac{1}{2} \sin(\theta)$. It is then clear from the diagram that 
\[ \frac{\tan (\theta)}{2} \geq \frac{\theta}{2} \geq \frac{\sin (\theta)}{2}.\]

Multiply all terms by $\ds \frac{2}{\sin(\theta)}$, giving 
\[ \frac{1}{\cos(\theta)} \geq \frac{\theta}{\sin (\theta)} \geq 1.\]

Taking reciprocals reverses the inequalities, giving 
\[ \cos (\theta) \leq \frac{\sin(\theta)}{\theta} \leq 1.\]
These inequalities hold for all values of $\theta$ near 0, even negative values, since $\cos (-\theta) = \cos(\theta)$ and $\sin (-\theta) = -\sin (\theta)$.

Now take the limit of everyting as $\theta \to 0$.

\[ \lim_{\theta \to 0} \cos (\theta) \leq \lim_{\theta \to 0} \frac{\sin(\theta)}{\theta} \leq \lim_{\theta \to 0}  1 \]
\[ \cos 0 \leq \lim_{\theta \to 0} \frac{\sin(\theta)}{\theta} \leq  1 \]
\[ 1 \leq \lim_{\theta \to 0} \frac{\sin(\theta)}{\theta} \leq  1 \]

By the Squeeze Theorem, $\ds \lim_{\theta \to 0} \frac{\sin(\theta)}{\theta}=1$.
\end{example}