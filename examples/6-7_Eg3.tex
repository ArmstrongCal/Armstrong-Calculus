\begin{marginfigure}[8cm] % MARGIN FIGURE
\margingraphics{figures/7_5_lake_michigan.eps}
\caption{Plot of $\frac{dP}{dt}$ vs. $P$. } \label{F:6.7.Ex3-1}
\end{marginfigure}

\begin{example} \label{eg:6.7.3} % EXAMPLE
In the Great Lakes region, rivers flowing into the lakes carry a great
deal of pollution in the form of small pieces of plastic averaging $1$
millimeter in diameter.  In order to understand how the amount of
plastic in Lake Michigan is changing, construct a model for how this type pollution has built up in the lake.


\solution First, some basic facts about Lake Michigan.
\begin{itemize}
  \item The volume of the lake is
    $5\cdot10^{12}$ cubic meters.
  \item Water flows into the lake at a rate of
    $5\cdot10^{10}$ cubic meters per year.  It flows out of the lake
    at the same rate.
  \item Each cubic meter flowing
    into the lake contains roughly $3\cdot10^{-8}$ cubic meters of
    plastic pollution.
\end{itemize}

Let's denote the amount of pollution in the lake by $P(t)$, where $P$
is measured in cubic meters of plastic and $t$ in years.  Our goal
is to describe the rate of change of this function;  in other
words, we want to develop a differential equation describing $P(t)$.

First, we will measure how $P(t)$ increases due to pollution flowing
into the lake.  We know that $5\cdot10^{10}$ cubic meters of water
enters the lake every year and each cubic meter of water contains
$3\cdot10^{-8}$ cubic meters of pollution.  Therefore, pollution
enters the lake at the rate of
$$ \left(5\cdot 10^{10} \frac{m^3 \mbox{\ water}}{\mbox{year}}\right) \cdot \left(3\cdot10^{-8} \frac{m^3 \mbox{\ plastic}}{m^3 \mbox{\ water}} \right) = 1.5\cdot 10^3$$
cubic meters of plastic per year.

Second, we will measure how $P(t)$ decreases due to pollution flowing
out of the lake.  If the total amount of pollution is $P$ cubic
meters and the volume of Lake Michigan is $5\cdot 10^{12}$ cubic
meters, then the concentration of plastic pollution in Lake Michigan is
$$
\frac{P}{5\cdot10^{12}} \quad \hbox{cubic meters of plastic per cubic meter of water}.
$$
Since $5\cdot10^{10}$ cubic meters of water flow out each year,and we assume that each cubic meter of water that flows out carries with it the plastic pollution it contains, then
the plastic pollution leaves the lake at the rate of
$$ \left(\frac{P}{5\cdot10^{12}} \frac{m^3 \mbox{\ plastic}}{m^3 \mbox{\ water}} \right) \cdot \left(5\cdot10^{10} \frac{m^3 \mbox{\ water}}{\mbox{year}} \right)=\frac{P}{100} $$
cubic meters of plastic per year.

The total rate of change of $P$ is thus the difference between the rate at which
pollution enters the lake minus the rate at which pollution leaves the
lake;  that is,
\begin{eqnarray*}
\frac{dP}{dt} & = &1.5\cdot10^{3}-\frac{P}{100} \\
                   & = & \frac{1}{100}(1.5\cdot10^{5} - P).
\end{eqnarray*}

We have now found a differential equation that describes the rate
at which the amount of pollution is changing.  To better understand the
behavior of $P(t)$, we now apply some
of the techniques we have recently developed.

Since this is an autonomous differential equation, we can sketch
$dP/dt$ as a function of $P$ and then construct a slope field, as shown in Figure~\ref{F:6.7.Ex3-1} and Figure \ref{F:6.7.Ex3-2}.

These plots both show that $P=1.5\cdot10^5$ is a stable equilibrium.  Therefore,
we should expect that the amount of pollution in Lake Michigan will
stabilize near $1.5\cdot10^5$ cubic meters of pollution.

Next, assuming that there is initially no pollution in the lake, we will
solve the initial value problem
$$
\frac{dP}{dt} = \frac{1}{100}(1.5\cdot10^{5} - P), \ P(0) = 0.
$$
Separating variables, we find that
$$
\frac1{1.5\cdot10^5-P} \frac{dP}{dt} = \frac1{100}.
$$
Integrating with respect to $t$, we have 
$$  \int \frac1{1.5\cdot10^5-P} \frac{dP}{dt}~dt = \int \frac1{100}~dt,$$
and thus changing variables on the left and antidifferentiating on both sides, we find that
\begin{eqnarray*}
  \int \frac{dP}{1.5\cdot10^5-P} &=& \int \frac1{100}~dt \\
  -\ln|1.5\cdot10^5 - P| & = & \frac1{100}t + C
\end{eqnarray*}
Finally, multiplying both sides by $-1$ and using the definition of the logarithm, we find that
\begin{equation} \label{E:7.5.Ex1C}  1.5\cdot10^5 - P = C e^{-t/100}.
\end{equation}
This is a good time to determine the constant $C$.  Since $P =
0$ when $t=0$, we have
$$
1.5\cdot 10^5 - 0 = Ce^0 = C.
$$
In other words, $C=1.5\cdot10^5$. 

Using this value of $C$ in Equation~(\ref{E:7.5.Ex1C}) and solving for $P$, we arrive at the solution
$$ P(t) = 1.5\cdot10^5(1-e^{-t/100}).$$
Superimposing the graph of $P$ on the slope field we saw in Figure \ref{F:6.7.Ex3-1} and Figure \ref{F:6.7.Ex3-2}, we see, as shown in Figure~\ref{F:6.7.Ex3-3}.

We see that, as expected, the amount of plastic pollution stabilizes around
$1.5\cdot10^5$ cubic meters.
\end{example}

\begin{marginfigure}[-16cm] % MARGIN FIGURE
\margingraphics{figures/7_5_slope_field.eps}
\caption{The slope field for the differential equation $\frac{dP}{dt} = \frac{1}{100}(1.5\cdot10^{5} - P)$.} \label{F:6.7.Ex3-2}
\end{marginfigure}

\begin{marginfigure}[-4cm] % MARGIN FIGURE
\margingraphics{figures/7_5_solution.eps}
\caption{The solution $P(t)$ and the slope field for the differential equation $\frac{dP}{dt} = \frac{1}{100}(1.5\cdot10^{5} - P)$.} \label{F:6.7.Ex3-3}
\end{marginfigure}