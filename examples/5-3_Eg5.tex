\begin{example} \label{eg:5.3.5} % EXAMPLE

Evaluate $\ds\int_0^5\frac{x^2}{\sqrt{x^2+25}}\ dx$.

\solution
Using \ref{C.5.3-tan}, we set $x=5\tan(\theta)$, $dx = 5\sec^2(\theta)\ d\theta$, and note that $\sqrt{x^2+25} = 5\sec(\theta)$. As we substitute, we can also change the bounds of integration.

The lower bound of the original integral is $x=0$. As $x=5\tan(\theta)$, we solve for $\theta$ and find $\theta = \arctan(x/5)$. Thus the new lower bound is $\theta = \arctan(0) = 0$. The original upper bound is $x=5$, thus the new upper bound is $\theta = \arctan(5/5) = \pi/4$. 

Thus we have 
\begin{align*}
\int_0^5\frac{x^2}{\sqrt{x^2+25}}\ dx &= \int_0^{\pi/4} \frac{25\tan^2(\theta)}{5\sec(\theta)}5\sec^2(\theta)\ d\theta\\
		&= 25\int_0^{\pi/4} \tan^2(\theta)\sec(\theta)\ d\theta.
\end{align*}
We encountered this indefinite integral in Example~\ref{eg:5.3.3} where we found 
$$\int \tan^2(\theta)\sec(\theta) \ d\theta = \frac12\big(\sec(\theta)\tan(\theta)-\ln|\sec(\theta)+\tan(\theta)|\big).$$
So
\begin{align*}
25\int_0^{\pi/4} \tan^2(\theta)\sec(\theta)\ d\theta &= \frac{25}2\big(\sec(\theta)\tan(\theta)-\ln|\sec(\theta)+\tan(\theta)|\big)\Bigg|_0^{\pi/4}\\
&= \frac{25}2\big(\sqrt2-\ln(\sqrt2+1)\big)
%&\approx 6.661.
\end{align*}

\end{example}