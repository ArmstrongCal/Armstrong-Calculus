\begin{example}
Suppose a ball is thrown straight up with velocity given by $v(t) = -32t+20$ ft/s, where $t$ is measured in seconds. Find, and interpret, $\ds \int_0^1 v(t) \ dt.$

\solution Using the second part of the Fundamental Theorem of Calculus, we have 
\begin{align*}
\int_0^1 v(t)\ dt &= \int_0^1 (-32t+20)\ dt \\
			&= -16t^2 + 20t\Big|_0^1 \\
			&= 4
\end{align*}
Thus if a ball is thrown straight up into the air with velocity $v(t) = -32t+20$, the height of the ball, $1$ second later, will be $4$ feet above the initial height. (Note that the ball has \textit{traveled} much farther. It has gone up to its peak and is falling down, but the difference between its height at $t=0$ and $t=1$ is $4$ feet.)
\end{example}