\begin{example} \label{Ex:2.1.Eg1}
For the function given by $f(x) = x - x^2$, use the limit definition of the derivative to compute $f'(2)$.  In addition, discuss the meaning of this value and draw a labeled graph that supports your explanation. 

\solution From the limit definition, we know that
$$f'(2) = \lim_{h \to 0} \frac{f(2+h)-f(2)}{h}.$$
Now we use the rule for $f$, and observe that $f(2) = 2 - 2^2 = -2$ and $f(2+h) = (2+h) - (2+h)^2.$  Substituting these values into the limit definition, we have that
$$f'(2) = \lim_{h \to 0} \frac{(2+h) - (2+h)^2 -  (-2)}{h}.$$
Observe that with $h$ in the denominator and our desire to let $h \to 0$, we have to wait to take the limit (that is, we wait to actually let $h$ approach $0$).  Thus, we do additional algebra.  Expanding and distributing in the numerator,
$$f'(2) = \lim_{h \to 0} \frac{2+h - 4 - 4h - h^2 + 2}{h}.$$
Combining like terms, we have
$$f'(2) = \lim_{h \to 0} \frac{ -3h - h^2}{h}.$$
Next, we observe that there is a common factor of $h$ in both the numerator and denominator, which allows us to simplify and find that
$$f'(2) = \lim_{h \to 0} (-3-h).$$
Finally, we are able to take the limit as $h \to 0$, and thus conclude that $f'(2) = -3$.

Now, we know that $f'(2)$ represents the slope of the tangent line to the curve $y = x - x^2$ at the point $(2,-2)$; $f'(2)$ is also the instantaneous rate of change of $f$ at the point $(2,-2)$.  Graphing both the function and the line through $(2,-2)$ with slope $m = f'(2) = -3$, we indeed see that by calculating the derivative, we have found the slope of the tangent line at this point, as shown in Figure~\ref{fig:2.1.Eg1}.
\end{example}

\begin{marginfigure}[-6cm]
\margingraphics{figs/2/2-1_Eg1.pdf} %Tangent line ex 1.3 active
\caption{The tangent line to $y = x - x^2$ at the point $(2,-2)$.}\label{fig:2.1.Eg1}
\end{marginfigure}