\begin{marginfigure}[4cm] % MARGIN FIGURE
\margingraphics{figures/figseries3}
\caption{Scatter plots relating to the series in Example \ref{eg:7.2.4}.} \label{F:eg:7.2.4}
\end{marginfigure}

\begin{example} \label{eg:7.2.4} % EXAMPLE
Evaluate the sum $\ds \sum_{n=1}^\infty \left(\frac1n-\frac1{n+1}\right)$.
\index{series!telescoping}\index{telescoping series}}

\solution
It will help to write down some of the first few partial sums of this series.
\begin{align*}
S_1 &=	\frac11-\frac12 & & = 1-\frac12\\
S_2 &=	\left(\frac11-\frac12\right) + \left(\frac12-\frac13\right) & & = 1-\frac13\\
S_3 &=	\left(\frac11-\frac12\right) + \left(\frac12-\frac13\right)+\left(\frac13-\frac14\right) & &= 1-\frac14\\
S_4 &=	\left(\frac11-\frac12\right) + \left(\frac12-\frac13\right)+\left(\frac13-\frac14\right) +\left(\frac14-\frac15\right)& &= 1-\frac15
\end{align*}
Note how most of the terms in each partial sum are canceled out! In general, we see that $\ds S_n = 1-\frac{1}{n+1}$. The sequence $\{S_n\}$ converges,  as $\ds \lim_{n\to\infty}S_n = \lim_{n\to\infty}\left(1-\frac1{n+1}\right) = 1$, and so we conclude that $\ds \sum_{n=1}^\infty \left(\frac1n-\frac1{n+1}\right) = 1$. Partial sums of the series are plotted in Figure \ref{F:eg:7.2.4}.
\mfigure{.75}{Scatter plots relating to the series of Example \ref{ex_series3}.}{fig:series3}{figures/figseries3}
\end{example}