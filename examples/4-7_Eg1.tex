\begin{marginfigure} % MARGIN FIGURE
\margingraphics{figures/4_4_TCTEx.eps}
\caption{The rate $r(t)$ of pollution leaking from a tank, measured in gallons per day.} \label{F:4.4.TCTEx}
\end{marginfigure} 

\begin{example} \label{eg:4.7.1} % EXAMPLE
Suppose that pollutants are leaking out of an underground storage tank at a rate of $r(t)$ gallons/day, where $t$ is measured in days.  It is conjectured that $r(t)$ is given by the formula $r(t) = 0.0069t^3 -0.125t^2+11.079$ over a certain 12-day period.  The graph of $y=r(t)$ is given in Figure~\ref{F:4.4.TCTEx}.  What is the meaning of $\int_4^{10} r(t) \, dt$ and what is its value?  What is the average rate at which pollutants are leaving the tank on the time interval $4 \le t \le 10$?

\solution We know that since $r(t) \ge 0$, the value of $\int_4^{10} r(t) \, dt$ is the area under the curve on the interval $[4,10]$.  If we think about this area from the perspective of a Riemann sum, the rectangles will have heights measured in gallons per day and widths measured in days, thus the area of each rectangle will have units of
\[ \frac{\mbox{gallons}}{\mbox{day}} \cdot \mbox{days} = \mbox{gallons}. \]
Thus, the definite integral tells us the total number of gallons of pollutant that leak from the tank from day 4 to day 10.  The Total Change Theorem tells us the same thing:  if we let $R(t)$ denote the function that measures the total number of gallons of pollutant that have leaked from the tank up to day $t$, then $R'(t) = r(t)$, and 
\[ \int_4^{10} r(t) \, dt = R(10) - R(4), \]
which is the total change in the function that measures total gallons leaked over time, thus the number of gallons that have leaked from day 4 to day 10.

To compute the exact value, we use the Fundamental Theorem of Calculus.  Antidifferentiating $r(t) = 0.0069t^3 -0.125t^2+11.079$, we find that
\begin{eqnarray*}
\int_4^{10} (0.0069t^3 -0.125t^2+11.079) \, dt & = & \left. \left( 0.0069 \cdot \frac{1}{4} \, t^4 - 0.125 \cdot \frac{1}{3} t^3 + 11.079t \right) \right|_4^{10} \\
& = & \left( 0.0069 \cdot \frac{1}{4} \, (10)^4 - 0.125 \cdot \frac{1}{3} (10)^3 + 11.079(10) \right) - \\
& \ &  \left( 0.0069 \cdot \frac{1}{4} \, (4)^4 - 0.125 \cdot \frac{1}{3} (4)^3 + 11.079(4) \right) \\
& \approx & 44.282. 
\end{eqnarray*}
Thus, approximately 44.282 gallons of pollutant leaked over the six day time period.

To find the average rate at which pollutant leaked from the tank over $4 \le t \le 10$, we want to compute the average value of $r$ on $[4,10]$.  Thus,
\[ r_{\mbox{\tiny{AVG}}[4,10]} = \frac{1}{10-4} \int_4^{10} r(t) \ dt \approx \frac{44.282}{6} = 7.380, \]
which has its units measured in gallons per day.
\end{example}