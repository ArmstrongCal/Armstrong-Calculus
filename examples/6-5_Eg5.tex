\begin{marginfigure}[6cm] % MARGIN FIGURE
\margingraphics{figures/figwork_pump1}
\caption{Illustrating a water tank in order to compute the work required to empty it in Example~\ref{eg:6.5.5}.} \label{F:6.5.Cylinder}
\end{marginfigure}

\begin{example} \label{eg:6.5.5} % EXAMPLE
A cylindrical storage tank with a radius of 10 ft and a height of 30 ft is filled with water, which weighs approximately 62.4 lb/ft$^3$. Compute the amount of work performed by pumping the water up to a point 5 feet above the top of the tank.

\solution
We will refer often to Figure \ref{F:6.5.Cylinder} which illustrates the salient aspects of this problem.

We start as we often do: we partition an interval into subintervals. We orient our tank vertically since this makes intuitive sense with the base of the tank at $y=0$. Hence the top of the water is at $y=30$, meaning we are interested in subdividing the $y$-interval $[0,30]$ into $n$ subintervals as 
$$0 = y_1 < y_2 < \cdots < y_{n+1} = 30.$$
Consider the work $W_i$ of pumping only the water residing in the $i\,^\text{th}$ subinterval, illustrated in Figure \ref{F:6.5.Cylinder}. The force required to move this water is equal to its weight which we calculate as volume $\times $ density. The volume of water in this subinterval is 
$V_i = 10^2\pi \Delta y_i$; its density is $62.4$ lb/ft$^3$. Thus the required force is $6240\pi\Delta y_i$ lb.

We approximate the distance the force is applied by using any $y$-value contained in the $i\,^\text{th}$ subinterval; for simplicity, we arbitrarily use $y_i$ for now (it will not matter later on). The water will be pumped to a point 5 feet above the top of the tank, that is, to the height of $y=35$ ft. Thus the distance the water at height $y_i$ travels is $35-y_i$ ft. 

In all, the approximate work $W_i$ performed in moving the water in the $i\,^\text{th}$ subinterval to a point 5 feet above the tank is 
$$W_i \approx 6240\pi\Delta y_i(35-y_i),$$
and the total work performed is
$$W \approx \sum_{i=1}^n W_i = \sum_{i=1}^n 6240\pi\Delta y_i(35-y_i).$$
This is a Riemann sum. Taking the limit as the subinterval length goes to 0 gives 
\begin{align*}
W 	&=	\int_0^{30} 6240\pi(35-y)\ dy \\
		&=  (6240\pi\left(35y-1/2y^2\right)\Big|_0^{30} \\
		&= 	11,762,123 \ \text{ft--lb}\\
		&\approx  1.176 \times 10^7 \ \text{ft--lb}.
\end{align*}

\end{example}