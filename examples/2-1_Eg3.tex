\begin{marginfigure}[1cm]
\margingraphics{figures/figtangent1} %FIG EXAMPLE 33 apex
\caption{A graph of $f(x) = 3x^2+5x-7$ and its tangent lines at $x=1$ and $x=3$.}\label{fig:2.1.Eg3}
\end{marginfigure}

\begin{example} \label{Ex:2.1.Eg3}
Let $f(x) = 3x^2+5x-7$. Find: 
\begin{enumerate*}[1)]
\item $f'(1)$; 	
\item the equation of the tangent line to the graph of $f$ at $x=1$; 
\item $f'(3)$; and 
\item the equation of the tangent line to the graph $f$ at $x=3$.
\end{enumerate*}
	
\solution
\begin{enumerate}[1)]
\item We compute this directly using the definition of the derivative at a point.
\begin{align*}
f'(1) &= \lim_{h \to 0} \frac{f(1+h)-f(1)}{h} \\
&=	\lim_{h \to 0} \frac{3(1+h)^2+5(1+h)-7 - (3(1)^2+5(1)-7)}{h}\\
&=	\lim_{h\to 0} \frac{3h^2+11h}{h}\\
&= 	\lim_{h\to 0} 3h+11=11.
%&= 11.
\end{align*}

\item The tangent line at $x=1$ has slope $f'(1)$ and goes through the point $(1,f(1)) = (1,1)$. Thus the tangent line has equation, in point-slope form, $y = 11(x-1) + 1$. In slope-intercept form we have $y = 11x-10$.
	
\item Again, using the definition,
\begin{align*}
f'(3) &=	\lim_{h \to 0} \frac{f(3+h)-f(3)}{h} \\
&=	\lim_{h \to 0} \frac{3(3+h)^2+5(3+h)-7 - (3(3)^2+5(3)-7)}{h} \\
&=	\lim_{h \to 0} \frac{3h^2+23h}{h}\\
&= \lim_{h \to 0} 3h+23 = 23
%&= 23.
\end{align*}
	
\item The tangent line at $x=3$ has slope $23$ and goes through the point $(3,f(3)) = (3,35)$. Thus the tangent line has equation $y=23(x-3)+35 = 23x-34$.
\end{enumerate}
\end{example}