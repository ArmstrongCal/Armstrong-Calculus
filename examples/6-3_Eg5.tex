\begin{example} \label{eg:6.3.5} % EXAMPLE
Find the surface area of the solid formed by revolving the curve $y=x^2$ on $[0,1]$ about:
\begin{enumerate}[1)]
\item the $x$-axis
\item	 the $y$-axis.
\end{enumerate}

\solution
\begin{enumerate}[1)]
\item The integral is straightforward to setup:
\begin{align*}
SA &= 2\pi\int_0^1 x^2\sqrt{1+(2x)^2}\ dx.
\intertext{Like the integral in Example \ref{eg:6.3.4}, this requires Trigonometric Substitution.}
&= \left.\frac{\pi}{32}\left(2(8x^3+x)\sqrt{1+4x^2}-\sinh^{-1}(2x)\right)\right|_0^1\\
&=\frac{\pi}{32}\left(18\sqrt{5}-\sinh^{-1}2\right)\\
&\approx 3.81\ \text{units}^2.
\end{align*}
The solid formed by revolving $y=x^2$ around the $x$-axis is graphed in Figure \ref{F:6.3.Ex5}-(a).
	
\item	 Since we are revolving around the $y$-axis, the ``radius'' of the solid is not $f(x)$ but rather $x$. Thus the integral to compute the surface area is:
\begin{align*}
SA &= 2\pi\int_0^1x\sqrt{1+(2x)^2}\ dx.
\intertext{This integral can be solved using substitution. Set $u=1+4x^2$; the new bounds are $u=1$ to $u=5$. We then have }
&=	\frac{\pi}4\int_1^5 \sqrt{u}\ du \\
&= \left.\frac{\pi}{4}\frac23 u^{3/2}\right|_1^5\\
&= \frac{\pi}6\left(5\sqrt{5}-1\right)\\
&\approx 5.33\ \text{units}^2.
\end{align*}
 The solid formed by revolving $y=x^2$ about the $y$-axis is graphed in Figure \ref{F:6.3.Ex5}-(b).	
\end{enumerate}
\end{example}

\begin{marginfigure}[-15cm] %MARGIN FIGURE
%\captionsetup[subfigure]{labelformat=empty}
\subfloat[]{\margingraphics{figures/figsa2a}}

\vspace{1cm}

\subfloat[]{\margingraphics{figures/figsa2b}}

\caption{The solids used in Example \ref{eg:6.3.5}} \label{F:6.3.Ex5}
\end{marginfigure}



