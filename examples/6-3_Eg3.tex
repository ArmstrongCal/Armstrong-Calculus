\begin{margintable}[3cm]
\begin{center}
\scalebox{1.15}{
\begin{tabular}{c | r} 
 $x$ & $\sqrt{1+\cos^2x}$ \\ \hline
 $0$ & $\sqrt{2}$ \\
 $\pi/4$ & $\sqrt{3/2}$ \\
 $\pi/2$ & $1$ \\
 $3 \pi/4$ & $\sqrt{3/2}$ \\
 $\pi$  & $\sqrt{2}$ \\
\end{tabular}}
\end{center}
\caption{A table of values of $y=\sqrt{1+\cos^2x}$ to evaluate a definite integral in Example \ref{eg:6.3.3}.} \label{T:6.3.Ex3}
\end{margintable}

\begin{example} \label{eg:6.3.3} % EXAMPLE
Find the length of the sine curve from $x=0$ to $x=\pi$.

\solution This is somewhat of a mathematical curiosity; in Activity \ref{A:4.5.1} (b) we found the area under one ``hump'' of the sine curve is 2 square units; now we are measuring its arc length.

The setup is straightforward: $f(x) = \sin x$ and $\fp(x) = \cos x$. Thus 
$$L = \int_0^\pi \sqrt{1+\cos^2x}\ dx.$$
This integral \textit{cannot} be evaluated in terms of elementary functions so we will approximate it with Simpson's Method with $n=4$. 

Table~\ref{T:6.3.Ex3} gives $\sqrt{1+\cos^2x}$ evaluated at 5 evenly spaced points in $[0,\pi]$. Simpson's Rule then states that 
\begin{align*}
\int_0^\pi \sqrt{1+\cos^2x}\ dx &\approx	\frac{\pi-0}{4\cdot 3}\left(\sqrt{2}+4\sqrt{3/2}+2(1)+4\sqrt{3/2}+\sqrt{2}\right) \\
			&=3.82918.
\end{align*}
Using a computer with $n=100$ the approximation is $L\approx 3.8202$; our approximation with $n=4$ is quite good.
\end{example}