\begin{example} \label{eg:5.4.5} % EXAMPLE
Decompose $\ds f(x)=\frac{1}{(x+5)(x-2)^3(x^2+2x+1)(x^2+x+7)^2}$ without solving for the resulting coefficients.

\solution
The denominator is already factored; we need to decompose $f(x)$ properly. Since $(x+5)$ is a linear term that divides the denominator, there will be a $\ds\frac{A}{x+5}$ term in the decomposition.

As $(x-2)^3$ divides the denominator, we will have the following terms in the decomposition:
$$\frac{B}{x-2},\quad \frac{C}{(x-2)^2}\quad \text{and}\quad \frac{D}{(x-2)^3}.$$

The $x^2+2x+1$ term in the denominator results in a $\ds\frac{Ex+F}{x^2+2x+1}$ term.

Finally, the $(x^2+x+7)^2$ term results in the terms $$\frac{Gx+H}{x^2+x+7}\quad \text{and}\quad \frac{Ix+J}{(x^2+x+7)^2}.$$
All together, we have $\ds \frac{1}{(x+5)(x-2)^3(x^2+2x+1)(x^2+x+7)^2} =$\footnotesize

\[ \frac{A}{x+5} + \frac{B}{x-2}+ \frac{C}{(x-2)^2}+\frac{D}{(x-2)^3}+ \frac{Ex+F}{x^2+2x+1}+ \frac{Gx+H}{x^2+x+7}+\frac{Ix+J}{(x^2+x+7)^2}\]\small

Solving for the coefficients $A$, $B \ldots J$ would be a bit tedious but not ``hard.''

\end{example}