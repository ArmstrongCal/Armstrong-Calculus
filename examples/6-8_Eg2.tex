\begin{example} \label{eg:6.8.2} % EXAMPLE
Evaluate the following derivatives and integrals.

\begin{enumerate*}[1)]
\item	 $\ds\frac{d}{dx}\big(\cosh (2x)\big)$\hspace{.5cm}
\item	 $\ds\int \sech^2(7t-3)\ dt$\hspace{.5cm}
\item	 $\ds \int_0^{\ln (2)} \cosh (x)\ dx$
\end{enumerate*}

\solution
\begin{enumerate}[1)]
\item Using the Chain Rule directly, we have $\frac{d}{dx} \big(\cosh (2x)\big) = 2\sinh (2x)$.

Just to demonstrate that it works, let's also use the Basic Identity $\cosh (2x) = \cosh^2(x)+\sinh^2(x)$.
\begin{align*}
\frac{d}{dx}\big(\cosh (2x)\big) &= \frac{d}{dx}\big(\cosh^2(x)+\sinh^2(x)\big) \\
&= 2\cosh (x)\sinh (x)+ 2\sinh (x)\cosh (x)\\ 
&= 4\cosh (x)\sinh (x).
\end{align*}
Using another Basic Identity, we can see that $4\cosh (x)\sinh (x) = 2\sinh (2x)$. We get the same answer either way.

\item	  We employ substitution, with $u = 7t-3$ and $du = 7dt$. Then we have:
$$ \int \sech^2 (7t-3)\ dt = \frac17\tanh (7t-3) + C.$$

\item	 $\ds \int_0^{\ln (2)} \cosh (x)\ dx = \sinh (x)\Big|_0^{\ln (2)} = \sinh (\ln (2)) - \sinh (0) = \sinh(\ln (2)).$
\noindent We can simplify this last expression as $\sinh x$ is based on exponentials:
$$\sinh(\ln (2)) = \frac{e^{\ln (2)}-e^{-\ln (2)}}2 = \frac{2-1/2}{2} = \frac34.$$
\end{enumerate}

\end{example}