\begin{marginfigure}
\margingraphics{figures/figextval4} %APEX ex78 
\caption{A graph of $f(x) = 2x^3+3x^2-12x$ on $[0,3]$ as in Example \ref{Ex:3.3.Eg2}. } \label{F:3.3.Ex2}
\end{marginfigure}

\begin{margintable}
\begin{center}
\scalebox{1.25}{
\begin{tabular}{cc} 
$x$ & $f(x)$ \\ \hline \rule{0pt}{10pt}
 $0$ & $0$ \\ 
 $1$ & $-7$\\
 $3$ & $45$ 
\end{tabular}
}% end scalebox
\end{center}
\caption{Finding the extreme values of $f$ in Example \ref{Ex:3.3.Eg2}.} \label{T:3.3.Ex2}
\end{margintable}

\begin{example} \label{Ex:3.3.Eg2}
Find the extreme values of $f(x) = 2x^3+3x^2-12x$ on $[0,3]$, graphed in Figure~\ref{F:3.3.Ex2}.

\solution
We follow the steps outlined. We first evaluate $f$ at the endpoints: $$f(0) = 0 \quad \text{and}\quad f(3) =45.$$

Next, we find the critical values of $f$ on $[0,3]$. $\fp(x) = 6x^2+6x-12 = 6(x+2)(x-1)$; therefore the critical values of $f$ are $x=-2$ and $x=1$. Since $x=-2$ does not lie in the interval $[0,3]$, we ignore it. Evaluating $f$ at the only critical number in our interval gives: $f(1) = -7$. 

The table in Table~\ref{T:3.3.Ex2} gives $f$ evaluated at the ``important'' $x$ values in $[0,3]$. We can easily see the maximum and minimum values of $f$: the maximum value is $45$ and the minimum value is $-7$.


\end{example}