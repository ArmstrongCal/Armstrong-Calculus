

\begin{example} \label{Ex:3.5.Eg1}
Consider $f(x) = x^3+5x+5$ on $[-3,3]$. Find $c$ in $[-3,3]$ that satisfies the Mean Value Theorem.

\solution
The average rate of change of $f$ on $[-3,3]$ is:
		$$\frac{f(3)-f(-3)}{3-(-3)} = \frac{84}{6} = 14.$$
		
We want to find $c$ such that $\fp(c) = 14$. We find $\fp(x) = 3x^2+5$. We set this equal to $14$ and solve for $x$. 
		\begin{align*}
		3x^2 +5 &= 14\\
		x^2  &= 3\\
		x &= \pm \sqrt{3} \approx \pm 1.732
		\end{align*}
		
We have found $2$ values $c$ in $[-3,3]$ where the instantaneous rate of change is equal to the average rate of change; the Mean Value Theorem guaranteed at least one. In Figure~\ref{F:3.5.Ex1} $f$ is graphed with a dashed line representing the average rate of change; the lines tangent to $f$ at $x=\pm \sqrt{3}$ are also given. Note how these lines are parallel (i.e., have the same slope) as the dashed line.

\end{example}

\begin{marginfigure}[-8cm]
\margingraphics{figures/figmvt4} %APEX ex83
\caption{Demonstrating the Mean Value Theorem in Example~\ref{Ex:3.5.Eg1}}\label{F:3.5.Ex1}
\end{marginfigure}